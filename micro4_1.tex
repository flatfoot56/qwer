\documentclass[a4paper]{article}
\usepackage[14pt]{extsizes} % 
\usepackage[utf8]{inputenc}
\usepackage{setspace,amsmath}
\usepackage{mathtools}
\usepackage{pgfplots}
\usepackage{titlesec}
\usepackage{pdfpages}
\usepackage[shortlabels]{enumitem}
\usepackage{tikz}
\usetikzlibrary{angles,quotes}
\usepackage{graphicx}
\usepackage{amssymb}
\usepackage{float}
\usepackage[section]{placeins}
\usepackage[makeroom]{cancel}
\usepackage{mathrsfs} % 
\newcommand\numberthis{\addtocounter{equation}{1}\tag{\theequation}}
%\addto\captionsrussian{\renewcommand{\figurename}{Fig.}}
\usepackage{amsmath,amsfonts,amssymb,amsthm,mathtools} 
\newcommand*{\hm}[1]{#1\nobreak\discretionary{}
{\hbox{$\mathsurround=0pt #1$}}{}}
\usepackage{graphicx}  % 
\graphicspath{{images/}{images2/}}  % 
\setlength\fboxsep{3pt} %  \fbox{} 
\setlength\fboxrule{1pt} % \fbox{}
\usepackage{wrapfig} % 
\newcommand{\prob}{\mathbb{P}}
\newcommand{\norma}{\mathscr{N}}
\newcommand{\expect}{\mathbb{E}}
\newcommand{\summa}{\sum_{i=1}^n}
\usepackage[left=7mm, top=20mm, right=15mm, bottom=20mm, nohead, footskip=10mm]{geometry} % 
\usepackage{tikz} % 
\def\myrad{2cm}% radius of the circle
\def\myanga{45}% angle for the arc
\def\myangb{195}
\begin{document} % 
	\begin{flushright}
	\begin{tabular}{r}
		Danil Fedchenko, MAE 2020, group A \\
	\end{tabular}
\end{flushright}


\begin{center}
	Microeconomics 4. Problem Set 1.
\end{center}
\section*{1\ Strategic entry deterrence}
A market is open for two periods, and demand is initially uncertain. With probability $p$,
demand is high $(H)$ in both periods, and with probability $1-p$, demand is low $(L)$ in both
periods. In period 1, a single firm (the incumbent) is alone in the market. In period 2,
the incumbent remains in the market; another firm (the entrant) decides whether or not
to enter and can, if it wishes, base its decision upon the incumbent's first-period profits.
The incumbent's profitability in each period is limited by the monopoly profit (which is
$M^H$ if demand is high and $M^L < M^H$ if demand is low); however, the incumbent may
decide to earn a smaller profit in period 1, by secretly slacking. If the entrant enters,
both firms' second-period duopoly profit is $D^H$ if demand is high and $D^L$
if demand
is low, where $D^L < 0 < D^H < M^L$. If entry does not occur, the entrant obtains 0
in the second period, while the incumbent earns $M^L$ or $M^H$, depending upon demand.
The incumbent's objective at the start of the game is to maximize the sum of first- and
second-period profit
\begin{enumerate}
	\item  Assume demand is publicly observable. Explain why, in equilibrium, the incumbent
	always chooses to earn the monopoly profit in the first period and the entrant enters
	if and only if demand is high.
	\item Assume instead, form now on, that the incumbent observes the level of demand at
	the start of period 1 but that the entrant does not observe it until after making
	the entry decision in period 2. Explain why, under this assumption of asymmetric information, there does not exist a perfect Bayesian equilibrium in which the
	incumbent uses the same strategy as in the equilibrium in question 1.
	\item Suppose $p D^H + (1-p)D^L < 0$. Describe the strategies and beliefs in one of the
	perfect Bayesian equilibria in this signalling model. State whether your equilibrium
	is pooling or separating
\end{enumerate}


\textbf{Solution}
\begin{enumerate}
	\item The tree of this game is depicted below.
		\begin{figure}[H]
		\centering
		\includegraphics[width=0.8\textwidth]{plotdraft}
		\caption{}\label{fig1}
	\end{figure}
Since $D^L < 0 < D^H$ the entrant in case of low demand will always choose not enter, and in case of high demand will always choose enter. The incumbent hence should solve the following optimization problems:
\begin{align*}
\underset{m \in [0, M^H]}{\max}\ m+D^H\ \text{ if demand is high}\\
\underset{m \in [0, M^L]}{\max}\ m+M^L\ \text{ if demand is low}\\
\end{align*}
That's why the optimal strategy for incumbent is to choose $M^H$ and $M^L$ in case of high and low demand respectively.
\item The tree of this game is depicted below
		\begin{figure}[H]
	\centering
	\includegraphics[width=0.8\textwidth]{plotdraft}
	\caption{}\label{fig2}
\end{figure}
Denote the belief of the entrant that he is at the high demand case as $\omega$. If incumbent chooses $M^H$ as a profit for high demand case and $M^L$ as a profit for the low demand case, then $\omega = 1$, that means that the strategy of the entrant consistent with such a belief is choosing not enter if $m = M^H$ and enter if $m = M^L$. But since $D^H < M^L$ the incumbent has incentive to deviate and choose $M^L$ if demand is high in order to prevent the entrant from entry and obtain $M^L + M^H$ instead of $M^H + D^H$. Thus, the "honest" strategy is not an equilibrium one for the incumbent.
\item The tree of the game is depicted below
\begin{figure}[H]
	\centering
	\includegraphics[width=0.8\textwidth]{plotdraft}
	\caption{}\label{fig3}
\end{figure}
Denote $\omega$ is a condition belief of the entrant that the demand is high. Assume $\omega = p: pD^H + (1-p)D^L < 0$. For such a belief the optimal strategy for the entrant is enter once the incumbent's profit if $M^H$ and not enter otherwise (once it is $M^L$). The incumbent's strategy hence is to choose $M^L$ in both cases because
\begin{align*}
M^H + D^H < M^H + M^L
\end{align*}
Moreover $\omega = p$ is consistent with such a strategy of the incumbent. Thus, the equilibrium is:
\begin{align*}
&\text{Incumbent : } \left\{M^L, M^L\right\}\\
&\text{Entrant : } \left\{\text{enter if incumbent's profit is } M^H \text{ not enter if it is } M^L\right\}\\
&\text{Entrant's belief : } (p, 1-p)
\end{align*}
This is a pooling equilibrium.
\end{enumerate}
\section*{2\ Beer and Quiche}
Nick wakes up and feels like fighting. He goes to a local pub in a hope of finding a target
there. Indeed, there he sees a man (called Paul) ordering a breakfast. Nick has seen Paul
before, but never talked to him. Nick knows that Paul can be of two types: Wimp or
Surly. Surly type is always eager to fight, and Nick does not want to fight with a Surly
type. In turn, Wimp is weak and easy to fight with, so Nick would be eager to release
his anger on a Wimp type. However, Nick does not know Paul's type exactly, he only
knows that the probability of each type is 0.5. Further, Nick has read in a book called
"Real Men Don't Eat Quiche" by Bruce Feirstein that surly people typically have beer
for breakfast, while wimp guys prefer ordering a quiche. Nick observes what Paul has
for breakfast, and decides whether to fight or pass. The payoffs are summarized in the
matrices below, where the strategies of Nick are in the columns, and the ones of Paul are
in the rows

\begin{figure}[H]
	\centering
	\includegraphics[width=0.8\textwidth]{surlywimp}
	\label{fig4}
\end{figure}
where $c$ is an additional cost for a wimp type to drink beer in the morning.
\begin{enumerate}
\item What conditions on $c$ guarantee the existence of a separating PBE? Fully describe
such an equilibrium.
\item For what values of $c$ is there a pooling equilibrium in which both Surly and Wimp
Paul gets beer for breakfast? Fully describe such an equilibrium.
\item Does the equilibrium in question 2 satisfy the intuitive criterion?
\end{enumerate}

\textbf{Solution}

\begin{enumerate}
	\item In the separating PBE the surly type chooses $beer$ and the wimp type chooses $quiche$. Since at the separating equilibrium Nick chooses to fight once Paul gets quiche for breakfast and to pass otherwise, getting a beer for wimp type should be such costly as even being aware of Nick's strategy Paul would not have a profitable deviation. That is, $1 - c < -1$ hence $c > 2$.
	
	If we denote belief for Paul of coming across the surly type condition on the fact that Paul gets the beer as $\omega$ and  belief for Paul of coming across the wimp type condition on the fact that Paul gets the quiche as $\lambda$, then the separating equilibrium is:
	\begin{align*}
	&\text{Paul : } \{beer, quiche\}\\
	&\text{Nick : } \{pass, fight\}\\
	& \omega = 1, \lambda = 0
	\end{align*}
	(first action of Paul corresponds to surly type and the second to wimp type, while first action of Nick corresponds to Paul orders beer and the second corresponds to Paul orders quiche)
	\item Assume $\omega = \frac{1}{2} > \frac{1}{3}$ in this case
	\begin{align*}
	2\omega + 1 - \omega < 0
	\end{align*}
	that means that Nich chooses $pass$ once Paul gets beer. If $\lambda > \frac{1}{3}$ then $-2\lambda + 1 - \lambda < 0$ which means that Nick chooses to pass even if Paul gets quiche. Since $c > 0$, $1 - c < 1$ and Paul should choose quiche for wimp type. In case $\lambda < \frac{1}{3}$ Nick chooses to fight once Paul gets quiche, as a result if $-1 \le 1 - c$ i.e. $c \le 2$ even wimp type will choose beer.
	Thus, the equilibrium is:
	\begin{align*}
	&\text{Paul : } \{beer, beer\}\\
	&\text{Nick : } \{pass, fight\}\\
	& \omega = \frac{1}{2}, \lambda < \frac{1}{3}
	\end{align*}
	(again, first action of Paul corresponds to surly type and the second to wimp type, while first action of Nick corresponds to Paul orders beer and the second corresponds to Paul orders quiche)
	\item It does not satisfy the intuitive criterion. It makes no sense for wimpish Paul to deviate, if $c \le 2$, whereas a surly Paul is indifferent between beer and quiche and in principle can deviate, that means that Nick should believe that if Paul gets quiche, he is surly and as a result he should avoid fighting, but being aware of that wimpish Paul chooses $quiche$, and the equilibrium is broken.
\end{enumerate}
\section*{3\ Education}
Consider Spence's job-marker signalling model. There are two firms on the market, and
two types of worker, $\theta_H$ and $\theta_L$; $\theta_H > \theta_L > 0$. The probability of type $\theta_H$ is $Prob(\theta =
\theta_H) = \lambda$, and is common knowledge. The worker knows his productivity and chooses
education $e \ge 0$. Education is productive: the productivity of a worker of type $\theta_i$ with
education $e$ is given by:
\begin{align*}
\theta_i + e
\end{align*}
The firms observe $e$ and simultaneously make the worker a wage offer. Then the worker
accepts the highest of the wages, $w$. The payoff of the worker of productivity $\theta_i, i = L, H,$ is:
\begin{align*}
w - c(\theta_i, e)
\end{align*}
where $c(\theta_i, e)$ is the cost of education $e$ for the worker of ability $\theta_i$,
\begin{align*}
c(\theta_i, e) = \frac{e^2}{2\theta_i}
\end{align*}
The payoff of the firm that hires the worker of type $\theta_i$ with education $e$ at the wage $w$ is
\begin{align*}
\theta_i + e - w
\end{align*}
Assume also that the firms have the same beliefs about the worker's productivity off equilibrium path. 
\begin{enumerate}
\item Consider the full information case, in which firms observe the type of the worker. What are the wages $w^*_L, w^*_H$, and the levels of education $e^*_L, e^*_H$ under this scenario?


Now assume that the firms observe the educational choices of the workers but not the
types.
\item  Can this game have a PBE which replicates the full information outcome in terms of
wages and educational levels? If yes, find the corresponding restrictions on the set
of parameters of the problem $(\theta_H; \theta_L; \lambda)$, and suggest an intuitive explanation for
your answer. If no, explain why not. Illustrate your answer graphically in $(w; e)$
space (do not worry about the precision of your graph).
\item Now consider pooling equilibria in this model:
\begin{enumerate}[(a)]
\item Are there situations in which $e_p = 0$ cannot be supported as a pooling equilibrium? If yes, describe the set of parameters $(\theta_H; \theta_L; \lambda)$ that corresponds to such a situation, and suggest an intuitive explanation for your answer. If no,
explain why not. Illustrate your answer graphically in $(w; e)$ space .
\item Can the pooling equilibria in this model be fully Pareto-ordered? Explain your
answer.
\item Formulate the dominance requirement
for PBE refinement. Assume that
the parameters are such that the set $\Omega$
of pooling equilibria surviving the
dominance requirement is non-empty. Does $\Omega$
contain the pooling equilibrium
which delivers the highest (pooling equilibrium) payoff to $\theta_H$?
\end{enumerate}
\end{enumerate}


\textbf{Solution}

\begin{enumerate}
	\item Competition requires that firms have zero profit, that means that
	\begin{align*}
	w_L &= \theta_L + e_L\\
	w_H &= \theta_H + e_H
	\end{align*}
	Thus, choosing the level of education workers of each type maximize:
	\begin{align*}
	\theta_L &+ e_L - \frac{e^2_L}{2\theta_L}, \text{ if } \theta = \theta_L\\
	\theta_H &+ e_H - \frac{e^2_H}{2\theta_H}, \text{ if } \theta = \theta_H
	\end{align*}
	hence $e^*_H = \theta_H, e^*_L = \theta_L$ and $w^*_L = 2\theta_L, w^*_H = 2\theta_H$.
	\item In this case, firms can perfectly distinguish the type of workers by their education level. Hence the beliefs should be $\mu(e_H) = 1, \mu(e_L) = 0$ where
	\begin{align*}
	\mu(e) = \prob(\text{the worker is of high type}|\text{the education level } e \text{ is observed})
	\end{align*}
	let us specify the beliefs off the equilibrium path as follows:
	\begin{align*}
	\mu(e) = \begin{cases}
	0, e < e_H\\
	1, e \ge e_H
	\end{cases}
	\end{align*}
	Hence in order for the low type workers do not have a profitable deviation the following condition on parameters should hold:
	\begin{align*}
	2\theta_H - \frac{\theta_H^2}{2\theta_L} &\le 2\theta_L - \frac{\theta_L^2}{2\theta_L}\\
	2(\theta_H - \theta_L) - \frac{1}{2\theta_L}(\theta_H^2 - \theta^2_L) &\le 0\\
	\frac{(\theta_H - \theta_L)(3\theta_L - \theta_H)}{2\theta_L} &\le 0\\
	\theta_L &\le \frac{\theta_H}{3}
	\end{align*}
	So, when the productivity of the low type is very low compare to that of high type, costs of education are very high for low type workers and it is not optimal for them to pretend to be a high type, which means that low type workers choose $e^*_L = \theta_L$ and high type workers choose $e^*_H = \theta_H$ and firms perfectly distinguish between types.
	\item 
	\begin{enumerate}[(a)]
	\item At the pooling equilibrium both types of workers choose
	\begin{align*}
	e^*_H = e^*_L = e_p
	\end{align*}
	and $\mu(e_p) = \lambda$. The expected profit of the firm, given an education level $e$ is
	\begin{align*}
	\mu(e) \theta_H + (1 - \mu(e)) \theta_L + e
	\end{align*}
	that means that the equilibrium wages should be
	\begin{align*}
	w(e_p) =  \lambda \theta_H + (1 - \lambda) \theta_L + e_p
	\end{align*} 
	Let us define beliefs off the equilibrium path as follows:
	\begin{align*}
	\mu(e) = \begin{cases}
	\lambda, e=e_p = 0\\
	0, e \neq e_p
	\end{cases}
	\end{align*}
	in this case
	\begin{align*}
	w(e) = \begin{cases}
	\lambda \theta_L + (1 - \lambda) \theta_H, e = 0\\
	\theta_L + e, e > 0
	\end{cases}
	\end{align*}
	Then the high type is solving the following optimization problem:
	\begin{align*}
	\max\left\{ \underset{e > 0}{\max}\ \theta_L + e - \frac{e^2}{2\theta_H}, \lambda \theta_L + (1 - \lambda) \theta_H \right\} = \max\left\{\theta_L + \frac{\theta_H}{2}, \lambda \theta_L + (1 - \lambda) \theta_H \right\}
	\end{align*}
	If $\lambda \le \frac{1}{2}$ then $\theta_L + \frac{\theta_H}{2} > \lambda \theta_L + (1 - \lambda)\theta_H$ and that means that high type will deviate and choose positive $e^*_H = \theta_H$ level of education, i.e. $e_p = 0$ is not supportable.
	\item No, they cannot be. Firstly, obviously that the greatest amount of efforts the workers can spend on education at any pooling equilibria cannot exceed $\bar{e}$ (see Fig. \ref{fig5}) because for any other education level $e > \bar{e}$ and respective equilibrium wage level $w(e) = \lambda \theta_H + (1 - \lambda)\theta_L + e$ the low type worker prefers $e = 0$.
	\begin{figure}[H]
		\centering
		\includegraphics[width=0.8\textwidth]{plotdraft}
		\caption{}\label{fig5}
	\end{figure}
Suppose now that the pooling equilibrium is attained at education level $e' < \bar{e}$ (see Fig. \ref{fig5}). The corresponding indifference curves and wage schedule is depicted on the Fig. \ref{fig5}. Both types are maximizing their utilities, given the wage schedule and equilibrium wages are consistent with beliefs. And this pooling equilibrium Pareto dominates the one with $e = \bar{e}$ (and also each equilibrium with $e' < e < \bar{e}$) because firms have the same zero expected profit, but both types attain higher indifference curves and consequently higher utility levels at $e'$. However, there always exists an education level $\underline{e} = \theta_H$ such that indifference curve of high type is tangent to the line $\lambda \theta_H + (1 - \lambda)\theta_L + e$ at this point (see Fig. \ref{fig5}). And each pooling  equilibrium level of education $e'' < \underline{e}$ will yield the high type worker strictly less utility. Since each point at line $\lambda \theta_H + (1 - \lambda) \theta_L + e$ for $e < \underline{e}$ corresponds to lower indifference curve for high type. Similarly for low type workers there exists an education level $\underline{\underline{e}} = \theta_L$ such that indifference curve of low type is tangent to the line $\lambda \theta_H + (1 - \lambda)\theta_L + e$ at this point (see Fig. \ref{fig5}). And each pooling  equilibrium level of education $e''' < \underline{\underline{e}}$ will yield the low type worker strictly less utility. Since each point at line $\lambda \theta_H + (1 - \lambda) \theta_L + e$ for $e < \underline{\underline{e}}$ corresponds to lower indifference curve for low type. As a result, there is a full Pareto order for pooling equilibrium for $e \in [\theta_H, \bar{e}]$ and for $e \in [0, \theta_L]$, for $e \in (\theta_L, \theta_H)$ there is no Pareto order, since some pooling equilibria are better for low type, but are worse for high type and vice-versa.
\item Highest pooling equilibrium payoff for the high type is delivered at pooling equilibrium education level $e^* = \theta_H$
	\end{enumerate}
\end{enumerate}
\end{document}