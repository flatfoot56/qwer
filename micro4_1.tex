\documentclass[a4paper]{article}
\usepackage[14pt]{extsizes} % 
\usepackage[utf8]{inputenc}
\usepackage{setspace,amsmath}
\usepackage{mathtools}
\usepackage{pgfplots}
\usepackage{titlesec}
\usepackage{pdfpages}
\usepackage[shortlabels]{enumitem}
\usepackage{tikz}
\usetikzlibrary{angles,quotes}
\usepackage{graphicx}
\usepackage{amssymb}
\usepackage{float}
\usepackage[section]{placeins}
\usepackage[makeroom]{cancel}
\usepackage{mathrsfs} % 
\newcommand\numberthis{\addtocounter{equation}{1}\tag{\theequation}}
%\addto\captionsrussian{\renewcommand{\figurename}{Fig.}}
\usepackage{amsmath,amsfonts,amssymb,amsthm,mathtools} 
\newcommand*{\hm}[1]{#1\nobreak\discretionary{}
{\hbox{$\mathsurround=0pt #1$}}{}}
\usepackage{graphicx}  % 
\graphicspath{{images/}{images2/}}  % 
\setlength\fboxsep{3pt} %  \fbox{} 
\setlength\fboxrule{1pt} % \fbox{}
\usepackage{wrapfig} % 
\newcommand{\prob}{\mathbb{P}}
\newcommand{\norma}{\mathscr{N}}
\newcommand{\expect}{\mathbb{E}}
\newcommand{\summa}{\sum_{i=1}^n}
\usepackage[left=7mm, top=20mm, right=15mm, bottom=20mm, nohead, footskip=10mm]{geometry} % 
\usepackage{tikz} % 
\def\myrad{2cm}% radius of the circle
\def\myanga{45}% angle for the arc
\def\myangb{195}
\begin{document} % 
	\begin{flushright}
	\begin{tabular}{r}
		Danil Fedchenko, MAE 2020, group A \\
	\end{tabular}
\end{flushright}


\begin{center}
	Microeconomics 4. Problem Set 1.
\end{center}
\section*{1\ Strategic entry deterrence}
A market is open for two periods, and demand is initially uncertain. With probability $p$,
demand is high $(H)$ in both periods, and with probability $1-p$, demand is low $(L)$ in both
periods. In period 1, a single firm (the incumbent) is alone in the market. In period 2,
the incumbent remains in the market; another firm (the entrant) decides whether or not
to enter and can, if it wishes, base its decision upon the incumbent's first-period profits.
The incumbent's profitability in each period is limited by the monopoly profit (which is
$M^H$ if demand is high and $M^L < M^H$ if demand is low); however, the incumbent may
decide to earn a smaller profit in period 1, by secretly slacking. If the entrant enters,
both firms' second-period duopoly profit is $D^H$ if demand is high and $D^L$
if demand
is low, where $D^L < 0 < D^H < M^L$. If entry does not occur, the entrant obtains 0
in the second period, while the incumbent earns $M^L$ or $M^H$, depending upon demand.
The incumbent's objective at the start of the game is to maximize the sum of first- and
second-period profit
\begin{enumerate}
	\item  Assume demand is publicly observable. Explain why, in equilibrium, the incumbent
	always chooses to earn the monopoly profit in the first period and the entrant enters
	if and only if demand is high.
	\item Assume instead, form now on, that the incumbent observes the level of demand at
	the start of period 1 but that the entrant does not observe it until after making
	the entry decision in period 2. Explain why, under this assumption of asymmetric information, there does not exist a perfect Bayesian equilibrium in which the
	incumbent uses the same strategy as in the equilibrium in question 1.
	\item Suppose $p D^H + (1-p)D^L < 0$. Describe the strategies and beliefs in one of the
	perfect Bayesian equilibria in this signalling model. State whether your equilibrium
	is pooling or separating
\end{enumerate}


\textbf{Solution}
\begin{enumerate}
	\item The tree of this game is depicted below.
		\begin{figure}[H]
		\centering
		\includegraphics[width=0.8\textwidth]{plotdraft}
		\caption{}\label{fig1}
	\end{figure}
Since $D^L < 0 < D^H$ the entrant in case of low demand will always choose not enter, and in case of high demand will always choose enter. The incumbent hence should solve the following optimization problems:
\begin{align*}
\underset{m \in [0, M^H]}{\max}\ m+D^H\ \text{ if demand is high}\\
\underset{m \in [0, M^L]}{\max}\ m+M^L\ \text{ if demand is low}\\
\end{align*}
That's why the optimal strategy for incumbent is to choose $M^H$ and $M^L$ in case of high and low demand respectively.
\item The tree of this game is depicted below
		\begin{figure}[H]
	\centering
	\includegraphics[width=0.8\textwidth]{plotdraft}
	\caption{}\label{fig2}
\end{figure}
Denote the belief of the entrant that he is at the high demand case as $\omega$. If incumbent chooses $M^H$ as a profit for high demand case and $M^L$ as a profit for the low demand case, then $\omega = 1$, that means that the strategy of the entrant consistent with such a belief is choosing not enter if $m = M^H$ and enter if $m = M^L$. But since $D^H < M^L$ the incumbent has incentive to deviate and choose $M^L$ if demand is high in order to prevent the entrant from entry and obtain $M^L + M^H$ instead of $M^H + D^H$. Thus, the "honest" strategy is not an equilibrium one for the incumbent.
\item The tree of the game is depicted below
\begin{figure}[H]
	\centering
	\includegraphics[width=0.8\textwidth]{plotdraft}
	\caption{}\label{fig3}
\end{figure}
Denote $\omega$ is a condition belief of the entrant that the demand is high. Assume $\omega = p: pD^H + (1-p)D^L < 0$. For such a belief the optimal strategy for the entrant is enter once the incumbent's profit is $M^H$ and not enter otherwise (once it is $M^L$). The incumbent's strategy hence is to choose $M^L$ in both cases because
\begin{align*}
M^H + D^H < M^H + M^L
\end{align*}
Moreover $\omega = p$ is consistent with such a strategy of the incumbent. Thus, the equilibrium is:
\begin{align*}
&\text{Incumbent : } \left\{M^L, M^L\right\}\\
&\text{Entrant : } \left\{\text{enter if incumbent's profit is } M^H \text{ not enter if it is } M^L\right\}\\
&\text{Entrant's belief : } (p, 1-p)
\end{align*}
This is a pooling equilibrium.
\end{enumerate}
\section*{2\ Beer and Quiche}
Nick wakes up and feels like fighting. He goes to a local pub in a hope of finding a target
there. Indeed, there he sees a man (called Paul) ordering a breakfast. Nick has seen Paul
before, but never talked to him. Nick knows that Paul can be of two types: Wimp or
Surly. Surly type is always eager to fight, and Nick does not want to fight with a Surly
type. In turn, Wimp is weak and easy to fight with, so Nick would be eager to release
his anger on a Wimp type. However, Nick does not know Paul's type exactly, he only
knows that the probability of each type is 0.5. Further, Nick has read in a book called
"Real Men Don't Eat Quiche" by Bruce Feirstein that surly people typically have beer
for breakfast, while wimp guys prefer ordering a quiche. Nick observes what Paul has
for breakfast, and decides whether to fight or pass. The payoffs are summarized in the
matrices below, where the strategies of Nick are in the columns, and the ones of Paul are
in the rows

\begin{figure}[H]
	\centering
	\includegraphics[width=0.8\textwidth]{surlywimp}
	\label{fig4}
\end{figure}
where $c$ is an additional cost for a wimp type to drink beer in the morning.
\begin{enumerate}
\item What conditions on $c$ guarantee the existence of a separating PBE? Fully describe
such an equilibrium.
\item For what values of $c$ is there a pooling equilibrium in which both Surly and Wimp
Paul gets beer for breakfast? Fully describe such an equilibrium.
\item Does the equilibrium in question 2 satisfy the intuitive criterion?
\end{enumerate}

\textbf{Solution}

\begin{enumerate}
	\item In the separating PBE the surly type chooses $beer$ and the wimp type chooses $quiche$. Since at the separating equilibrium Nick chooses to fight once Paul gets quiche for breakfast and to pass otherwise, getting a beer for wimp type should be such costly as even being aware of Nick's strategy Paul would not have a profitable deviation. That is, $1 - c < -1$ hence $c > 2$.
	
	If we denote belief for Paul of coming across the surly type condition on the fact that Paul gets the beer as $\omega$ and  belief for Paul of coming across the wimp type condition on the fact that Paul gets the quiche as $\lambda$, then the separating equilibrium is:
	\begin{align*}
	&\text{Paul : } \{beer, quiche\}\\
	&\text{Nick : } \{pass, fight\}\\
	& \omega = 1, \lambda = 0
	\end{align*}
	(first action of Paul corresponds to surly type and the second to wimp type, while first action of Nick corresponds to Paul orders beer and the second corresponds to Paul orders quiche)
	\item Assume $\omega = \frac{1}{2} > \frac{1}{3}$ in this case
	\begin{align*}
	2\omega + 1 - \omega < 0
	\end{align*}
	that means that Nick chooses $pass$ once Paul gets beer. If $\lambda > \frac{1}{3}$ then $-2\lambda + 1 - \lambda < 0$ which means that Nick chooses to pass even if Paul gets quiche. Since $c > 0$, $1 - c < 1$ and Paul should choose quiche for wimp type. In case $\lambda < \frac{1}{3}$ Nick chooses to fight once Paul gets quiche, as a result if $-1 \le 1 - c$ i.e. $c \le 2$ even wimp type will choose beer.
	Thus, the equilibrium is:
	\begin{align*}
	&\text{Paul : } \{beer, beer\}\\
	&\text{Nick : } \{pass, fight\}\\
	& \omega = \frac{1}{2}, \lambda < \frac{1}{3}
	\end{align*}
	(again, first action of Paul corresponds to surly type and the second to wimp type, while first action of Nick corresponds to Paul orders beer and the second corresponds to Paul orders quiche)
	\item It does not satisfy the intuitive criterion. It makes no sense for wimpish Paul to deviate, if $c \le 2$, whereas a surly Paul is indifferent between beer and quiche and in principle can deviate, that means that Nick should believe that if Paul gets quiche, he is surly and as a result he should avoid fighting, but being aware of that wimpish Paul chooses $quiche$, and the equilibrium is broken.
\end{enumerate}
\section*{3\ Education}
Consider Spence's job-marker signalling model. There are two firms on the market, and
two types of worker, $\theta_H$ and $\theta_L$; $\theta_H > \theta_L > 0$. The probability of type $\theta_H$ is $Prob(\theta =
\theta_H) = \lambda$, and is common knowledge. The worker knows his productivity and chooses
education $e \ge 0$. Education is productive: the productivity of a worker of type $\theta_i$ with
education $e$ is given by:
\begin{align*}
\theta_i + e
\end{align*}
The firms observe $e$ and simultaneously make the worker a wage offer. Then the worker
accepts the highest of the wages, $w$. The payoff of the worker of productivity $\theta_i, i = L, H,$ is:
\begin{align*}
w - c(\theta_i, e)
\end{align*}
where $c(\theta_i, e)$ is the cost of education $e$ for the worker of ability $\theta_i$,
\begin{align*}
c(\theta_i, e) = \frac{e^2}{2\theta_i}
\end{align*}
The payoff of the firm that hires the worker of type $\theta_i$ with education $e$ at the wage $w$ is
\begin{align*}
\theta_i + e - w
\end{align*}
Assume also that the firms have the same beliefs about the worker's productivity off equilibrium path. 
\begin{enumerate}
\item Consider the full information case, in which firms observe the type of the worker. What are the wages $w^*_L, w^*_H$, and the levels of education $e^*_L, e^*_H$ under this scenario?


Now assume that the firms observe the educational choices of the workers but not the
types.
\item  Can this game have a PBE which replicates the full information outcome in terms of
wages and educational levels? If yes, find the corresponding restrictions on the set
of parameters of the problem $(\theta_H; \theta_L; \lambda)$, and suggest an intuitive explanation for
your answer. If no, explain why not. Illustrate your answer graphically in $(w; e)$
space (do not worry about the precision of your graph).
\item Now consider pooling equilibria in this model:
\begin{enumerate}[(a)]
\item Are there situations in which $e_p = 0$ cannot be supported as a pooling equilibrium? If yes, describe the set of parameters $(\theta_H; \theta_L; \lambda)$ that corresponds to such a situation, and suggest an intuitive explanation for your answer. If no,
explain why not. Illustrate your answer graphically in $(w; e)$ space .
\item Can the pooling equilibria in this model be fully Pareto-ordered? Explain your
answer.
\item Formulate the dominance requirement
for PBE refinement. Assume that
the parameters are such that the set $\Omega$
of pooling equilibria surviving the
dominance requirement is non-empty. Does $\Omega$
contain the pooling equilibrium
which delivers the highest (pooling equilibrium) payoff to $\theta_H$?
\end{enumerate}
\end{enumerate}


\textbf{Solution}

\begin{enumerate}
	\item Competition requires that firms have zero profit, that means that
	\begin{align*}
	w_L &= \theta_L + e_L\\
	w_H &= \theta_H + e_H
	\end{align*}
	Thus, choosing the level of education workers of each type maximize:
	\begin{align*}
	\theta_L &+ e_L - \frac{e^2_L}{2\theta_L}, \text{ if } \theta = \theta_L\\
	\theta_H &+ e_H - \frac{e^2_H}{2\theta_H}, \text{ if } \theta = \theta_H
	\end{align*}
	hence $e^*_H = \theta_H, e^*_L = \theta_L$ and $w^*_L = 2\theta_L, w^*_H = 2\theta_H$.
	\item In this case, firms can perfectly distinguish the type of workers by their education level. Hence the beliefs should be $\mu(e_H) = 1, \mu(e_L) = 0$ where
	\begin{align*}
	\mu(e) = \prob(\text{the worker is of high type}|\text{the education level } e \text{ is observed})
	\end{align*}
	let us specify the beliefs off the equilibrium path as follows:
	\begin{align*}
	\mu(e) = \begin{cases}
	0, e < e_H\\
	1, e \ge e_H
	\end{cases}
	\end{align*}
	Hence in order for the low type workers do not have a profitable deviation the following condition on parameters should hold:
	\begin{align*}
	2\theta_H - \frac{\theta_H^2}{2\theta_L} &\le 2\theta_L - \frac{\theta_L^2}{2\theta_L}\\
	2(\theta_H - \theta_L) - \frac{1}{2\theta_L}(\theta_H^2 - \theta^2_L) &\le 0\\
	\frac{(\theta_H - \theta_L)(3\theta_L - \theta_H)}{2\theta_L} &\le 0\\
	\theta_L &\le \frac{\theta_H}{3}
	\end{align*}
	So, when the productivity of the low type is very low compare to that of high type, costs of education are very high for low type workers and it is not optimal for them to pretend to be a high type, which means that low type workers choose $e^*_L = \theta_L$ and high type workers choose $e^*_H = \theta_H$ and firms perfectly distinguish between types. On the figure below two examples are provided where the PBE which replicates the full information outcome is possible and where not.
	\begin{figure}[H]
		\centering
		\includegraphics[width=0.8\textwidth]{plotdraft}
		\caption{Left: low type has profitable deviation; Right: low type does not have a profitable deviation}\label{fig7}
	\end{figure}
	\item 
	\begin{enumerate}[(a)]
	\item At the pooling equilibrium both types of workers choose
	\begin{align*}
	e^*_H = e^*_L = e_p
	\end{align*}
	and $\mu(e_p) = \lambda$. The expected profit of the firm, given an education level $e$ is
	\begin{align*}
	\mu(e) \theta_H + (1 - \mu(e)) \theta_L + e
	\end{align*}
	that means that the equilibrium wages should be
	\begin{align*}
	w(e_p) =  \lambda \theta_H + (1 - \lambda) \theta_L + e_p
	\end{align*} 
	Let us define beliefs off the equilibrium path as follows:
	\begin{align*}
	\mu(e) = \begin{cases}
	\lambda, e=e_p = 0\\
	0, e \neq e_p
	\end{cases}
	\end{align*}
	in this case
	\begin{align*}
	w(e) = \begin{cases}
	\lambda \theta_H + (1 - \lambda) \theta_L, e = 0\\
	\theta_L + e, e > 0
	\end{cases}
	\end{align*}
	Then the high type is solving the following optimization problem:
	\begin{align*}
	\max\left\{ \underset{e > 0}{\max}\ \theta_L + e - \frac{e^2}{2\theta_H}, \lambda \theta_H + (1 - \lambda) \theta_L \right\} = \max\left\{\theta_L + \frac{\theta_H}{2}, \lambda \theta_H + (1 - \lambda) \theta_L \right\}
	\end{align*}
	If $0 \le \lambda \le \frac{1}{2}$ then $\theta_L + \frac{\theta_H}{2} > \lambda \theta_H + (1 - \lambda)\theta_L$ and that means that high type will deviate and choose positive $e^*_H = \theta_H$ level of education, i.e. $e_p = 0$ is not supportable. Figure below illustrates the case.
	\begin{figure}[H]
		\centering
		\includegraphics[width=0.8\textwidth]{plotdraft}
		\caption{Unsupportable $e_p = 0$}\label{fig8}
	\end{figure}
	\item No, they cannot be. The levels of education which maximize workers' utilities provided that firms pay a wage $\lambda \theta_H + (1 - \lambda) \theta_l + e$ are $e_H = \theta_H, e_L = \theta_L$. Since $\theta_H > \theta_L$ all pooling equilibria with education level above $\theta_H$ are Pareto ordered because decreasing of $e$ benefits both types of workers and firm still earn zero profit. However, equilibria with education level between $\theta_L$ and $\theta_H$ are not Pareto ordered because once $e^*_p \to \theta_L$ the high type becomes worse-off and the low type becomes better-off.
\item Dominance requirement requires firms to prescribe positive $\mu(e)$ only to such $e$ that is not a strictly dominated action for high type. In the sense that there is no action $e'$ which yields the high type strictly greater payoff regardless of what equilibrium wages are. Assumption that $\Omega$ in non-empty means that firstly $\theta_H < \bar{e}$ (where $\bar{e}$ is the highest possible amount of education the low type worker obtains in any pooling equilibria) and on the top of that the indifference curve of high type, which goes through $(\bar{e}, \theta_H + \bar{e})$ lies below the indifference curve which is tangent to the line $\lambda \theta_H + (1 - \lambda) \theta_L$ at the point $\theta_H$ (see Fig. \ref{fig6}) because otherwise, the high type chooses the level $\max\ \{\theta_H, \bar{e}\}$ and low type chooses $0$ and $\Omega$, the set of pooling equilibria, is empty.
	\begin{figure}[H]
	\centering
	\includegraphics[width=0.8\textwidth]{plotdraft}
	\caption{}\label{fig6}
\end{figure} 
The highest pooling equilibrium payoff for the high type is delivered at the education level $e = \theta_H$. In our case this equilibrium belongs to $\Omega$ because neither of types have $e = \theta_H$ as a strictly dominated action. On the Fig. \ref{fig6} there is a wage schedule for which $e_H = e_L = \theta_H$ is the best response for both types (i.e. there is no action which strictly dominates this action).
	\end{enumerate}
\end{enumerate}
\section*{4\ Exams}
Karen would like her friend Tim to pass his exams. Tim may choose to work at a
private cost of $c$. If he slacks, he fails. If he works, he passes with probability $\theta$. Hence $\theta$
is Tim's $ability$. If he passes, Karen receives a private benefit of $V > 0$ and Tim receives
a private benefit of $U > 0$.
Karen offers to pay a reward $b \in [0; V]$ to Tim if and only if he passes. Hence Karen
receives a payoff $V-b$ if Tim passes, and zero otherwise. Tim receives a payoff of zero if
he slacks, $-c$ if he works and fails, and $U + b-c$ if he works and passes. For simplicity,
assume that Tim will choose to work whenever he is otherwise indifferent between working
and slacking.
Tim's ability is initially unknown; $\theta_H$ with probability $\alpha$ and $\theta_L$ otherwise where
\begin{align*}
1 > \theta_H > \theta_L > 0, [\alpha \theta_H + (1-\alpha)\theta_L] U < c < \theta_L(U + V ) \text{ and } c < \theta_H U
\end{align*}
We refer to Tim's beliefs about $\theta$ as his "confidence". Begin by assuming that neither
player observes $\theta$. Hence Karen begins by choosing a bonus $b$. Tim observes this bonus,
and then chooses whether to work
\begin{enumerate}
	\item What bonus will Karen choose to offer?
	
	
	Karen now observes Tim's ability $\theta$, but Tim does not. Hence, Nature begins by
	choosing $\theta \in \{\theta_H; \theta_L\}$. Next, Karen observes $\theta$ and offers a bonus $b_H$ (if $\theta = \theta_H$) or $b_L$
	(if $\theta = \theta_L$). Finally, Tim observes $b$, but not $\theta$, and then chooses whether to work. We seek pure strategy perfect Bayesian Nash equilibria.
	\item Show that there is no separating equilibrium, and hence $b^*_H = b^*_L$ in any equilibrium.
	\item Does the standard sorting (single crossing) condition hold in part 2?
	
	
	Suppose now that Karen offers Tim an unconditional gift $w \ge 0$ as well as the bonus
	$b \ge 0$. Hence she pays $w+b$ if Tim passes, and $w$ if he fails. So Karen offers $(w_H; b_H)$ when
	she observes $\theta_H$ and $(w_L; b_L)$ when she observes $\theta_L$. We seek a separating equilibrium.
	
	
	\item Show that $w^*_L = 0$ and $b^*_L = \frac{c}{\theta_L} - U$ in any separating equilibrium. Show that
	together with $w^*_H = c - \theta_LU$ and $b_H^* = 0$ this constitutes an equilibrium.
	\item Briefly interpret this equilibrium
\end{enumerate}


\textbf{Solution}

\begin{enumerate}
	\item Since Karen gets 0 if Tim slacks and if he works she in the worst case gets 0, she should offer such a $b$ that Tim would be indifferent between work and slack, and as a result chooses to work.
	\begin{align*}
	\alpha(\theta_H(U + b - c) &- (1 - \theta_H)c) + (1 - \alpha)(\theta_L(U + b - c) - (1 - \theta_L)c) = 0\\
	&(U+b)(\alpha\theta_H + (1 - \alpha)\theta_L) = c\\
	&b = \frac{c}{\alpha\theta_H + (1 - \alpha)\theta_L} - U
	\end{align*}
	\item In separating equilibrium Tim can perfectly distinguish whether $\theta$ is high or low, hence in case of high $\theta$ he will work only if:
	\begin{align*}
	(U + b_H)\theta_H - c \ge 0\\
	b_H \ge \frac{c}{\theta_H} - U < 0
	\end{align*}
	i.e. if $\theta = \theta_H$ Karen's optimal offer is $b_H = 0$. If $\theta = \theta_L$ then Tim will work only if
	\begin{align*}
	b_L \ge \frac{c}{\theta_L} - U
	\end{align*}
	and hence optimal bonus which Karen offers is:
	\begin{align*}
	b_L = \max\ \left\{\frac{c}{\theta_l} - U, 0\right\}
	\end{align*}
	but if $b_L = 0$ then equilibrium becomes pooling and Tim cannot distinguish between states. If $\frac{c}{\theta_l} - U > 0$ then Karen will want to deviate and offer $b_L = 0$ in order to convince Tim that he is of high type. Hence at each equilibrium $b^*_H = b^*_L$.
	\item Single crossing condition means that once cost of work increases the low type Tim needs higher bonus compensation to preserve the same utility level than the high type Tim. Expected utility levels of both types are the following:
	\begin{align*}
	u_H = \begin{cases}
	(U + b) \theta_H - c, \text{ if Tim works}\\
	0, \text{if Tim slacks}
	\end{cases}\\
	u_L = \begin{cases}
	(U + b) \theta_L - c, \text{ if Tim works}\\
	0, \text{if Tim slacks}
	\end{cases}\\
	\end{align*}
	Since $\frac{1}{\theta_H} < \frac{1}{\theta_L}$ then for high levels of utility single crossing condition holds. However, for example for utility level $0$ and for $c < U \theta_L$ (see Fig. \ref{fig9})as we can see utility curves coincide, i.e. the single crossing condition does not hold.
	\begin{figure}[H]
		\centering
		\includegraphics[width=0.8\textwidth]{plotdraft}
		\caption{}\label{fig9}
	\end{figure}
	\item  As always at separating equilibrium Tim can perfectly distinguish between states. That means that at low state he will work only if
	\begin{align*}
	\theta_L(U + b_L) + w_L - c \ge w_L
	\end{align*}
	hence Karen should offer such $(b_L, w_L)$ that:
	\begin{align*}
	\theta_L b_L = c - \theta_LU
	\end{align*}
	In high state it is profitable for Tim to accept any positive offer $(b_H, w_H)$ and it would be optimal for Karen to offer $(0, 0)$ but it cannot be a separating equilibrium because at low state Karen will have a profitable deviation, namely $(0, 0)$ and equilibrium becomes pooling. So, at the separating equilibrium $(w_H, b_H)$ should be such that Karen does not want to deviate at low state. That is:
	\begin{align*}
	\theta_L(V - b_H - w_H) + (1 - \theta_L)(-w_H) &\le \theta_L(V - b_L - w_L) + (1 - \theta_L)(-w_L)\\
	\theta_L V - \theta_Lb_H - w_H &\le \theta_L(V + U) - c - w_L
	\end{align*}
	hence
	\begin{align}\label{eq1}
	\theta_L b_H + w_H = c +w_L - \theta_L U
	\end{align}
	and finally Karen should choose such $(b_H, w_H), (b_L, w_L)$ which maximize her expected payoff. At low state she should maximize:
	\begin{align*}
	\theta_L (V - b_L) - w_L = \theta_L V - c + \theta_LU - w_L \to \underset{w_L}{\max}
	\end{align*}
	obviously the solution is $w_L = 0$. Similarly, using \eqref{eq1}, at high state she should maximize:
	\begin{align*}
	\theta_HV - \theta_H b_H - w_H = \theta_HV - \theta_Hb_H - c + \theta_LU + \theta_L b_H \to \underset{b_H}{\max}\\
	\theta_H V + \theta_l U - c - (\theta_H - \theta_L)b_H \to \underset{b_H}{\max}
	\end{align*}
	the solution is $b_H = 0$. Thus, the equilibrium is
	\begin{align*}
	w^*_L = 0,\ b^*_L = \frac{c}{\theta_L} - U\\
	w^*_H = c - \theta_LU,\ b^*_H = 0
	\end{align*}
	as we sought.
	\item The interpretation is straightforward. Without unconditional gift Karen pays her bonus ex post, as a result she had incentives to deviate and underpay Tim, to cheat him that he is of high type. Whereas the introduction of the gift which Karen pays ex ante can be viewed as a commitment that Karen will not cheat Tim and will not make his expected payoff negative when Tim is of low type.
\end{enumerate}
\section*{5\ Beliefs}
The definition of the term $arbitrary$ is the following:
\begin{align*}
\text{based on random choice or personal whim, rather than any reason or system}
\end{align*}First of all such beliefs are not fully arbitrary, they should be consistent with strategies, players play. That is, given those beliefs and actions the player should undertake according to them, players do not have profitable unilateral deviations from the equilibrium path. But even if we actually can arbitrary (to the aforementioned extent) choose beliefs for the off equilibrium path info sets, conclusions are not arbitrary in the slightest. They are fully consistent with this particular concept of equilibrium. We do not conclude that tomorrow will be rain because some information set at the game we study is reached with zero probability at the equilibrium. It is the example of arbitrary conclusion. But those conclusions we usually draw are correct for the model and concept of equilibrium we chose. If we do not like these conclusions we can only choose another concept of equilibrium, use some refinements for example, but we cannot say that conclusions are arbitrary, they are valid, correct for this concept and not random.
\section*{6\ Wage Inequality}
Suppose access to a university education becomes less dependent on wealth or social class.
Using a signalling model of education, what predictions would you make about how wage
inequality within age cohorts would change?

\textbf{Solution}

There can be three types of predictions: wage inequality can increase, decrease or remains unchanged. And in principle all three types can be possible, depending on parameters of the model and concept of equilibrium we use (PBE or PBE+intuitive criterion refinement)


Assume for simplicity that we have only two types of workers (of high productivity and of low productivity) and that education is useless, in the sense that it does not increase productivity (a plausible assumption for Russian educational system).

In addition assume that we have a separating equilibrium i.e. the wage inequality is $\theta_H - \theta_L$ and moreover let us use the intuitive criterion refinement which eliminates all pooling equilibria. Since indifference curves of low type workers become flatter, it is now becomes profitable for them to deviate and to pretend to be workers of high productivity. In this case either firms become unable to distinguish between the types and offer a single wage for all types (pooling equilibrium) and as a result, inequality decreases to 0, or workers of high type just increase their level of education (starting to pursue PhD for example) and firm still are able to distinguish between types and offer different wages, hence the wage inequality remains unchanged.
\section*{7 Equilibrium Refinements}
	There are many different criteria that can be used to refine the set of equilibria in a given
	game. One of them, which you have already discussed, is the intuitive criterion. Think about
	the logic it tries to capture. Can you think of other ways to select reasonable out-of-equilibrium path beliefs? Suggest one such criterion and explain intuitively why it is reasonable. You may
	refer to Beer and Quiche or Education signalling or your own set up to illustrate the logic behind
	your criterion.
	
	
	\textbf{Solution}
	
	Intuitive criterion and other criteria try to refine unreasonable out-off equilibrium beliefs. For example, in $Beer-and-Quiche$ set-up (Problem 2). Intuitive criterion requires $\lambda$ (the conditional probability of coming across surly type, condition on the fact that Paul has ordered quiche) being equal to 1, because it is unreasonable for wimpish Paul to deviate from equilibrium. It is a good logic, but as for me it does not fully capture reality. In reality there is always a some probability of mistake, moreover Nick cannot for sure know the cost of drinking beer for wimp Paul. That's why I think it would be more reasonable to consider beliefs in unreachable information set as a random variable distributed on $[0, 1]$. It can be beta $B(a, b)$ distribution. Where parameters $a$ and $b$ are chosen taking into account possible payoffs of players. For instance $c$ is very close to 2 then Paul can not care a lot about his order and the probability of mistake becomes higher. However even if $c$ is small Nick cannot rule out the chance that apart from being wimpish Paul is also stupid and orders $quiche$. Thus, the distribution of belief $\lambda \sim B(a, b)$ where parameters are chosen in a special way (if this problem costs more than 5 point maybe I will derive some possible formulas for $a$ and $b$) capture the reality more precisely, in my opinion.
\end{document}