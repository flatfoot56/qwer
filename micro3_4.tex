\documentclass[a4paper]{article}
\usepackage[14pt]{extsizes} % 
\usepackage[utf8]{inputenc}
\usepackage{setspace,amsmath}
\usepackage{mathtools}
\usepackage{pgfplots}
\usepackage{titlesec}
\usepackage{pdfpages}
\usepackage[shortlabels]{enumitem}
\usepackage{tikz}
\usetikzlibrary{angles,quotes}
\usepackage{graphicx}
\usepackage{amssymb}
\usepackage{float}
\usepackage[section]{placeins}
\usepackage[makeroom]{cancel}
\usepackage{mathrsfs} % 
\newcommand\numberthis{\addtocounter{equation}{1}\tag{\theequation}}
%\addto\captionsrussian{\renewcommand{\figurename}{Fig.}}
\usepackage{amsmath,amsfonts,amssymb,amsthm,mathtools} 
\newcommand*{\hm}[1]{#1\nobreak\discretionary{}
{\hbox{$\mathsurround=0pt #1$}}{}}
\usepackage{graphicx}  % 
\graphicspath{{images/}{images2/}}  % 
\setlength\fboxsep{3pt} %  \fbox{} 
\setlength\fboxrule{1pt} % \fbox{}
\usepackage{wrapfig} % 
\newcommand{\prob}{\mathbb{P}}
\newcommand{\norma}{\mathscr{N}}
\newcommand{\expect}{\mathbb{E}}
\newcommand{\summa}{\sum_{i=1}^n}
\usepackage[left=7mm, top=20mm, right=15mm, bottom=20mm, nohead, footskip=10mm]{geometry} % 
\usepackage{tikz} % 
\def\myrad{2cm}% radius of the circle
\def\myanga{45}% angle for the arc
\def\myangb{195}
\begin{document} % 
	\begin{flushright}
	\begin{tabular}{r}
		Danil Fedchenko, MAE 2020, group A \\
	\end{tabular}
\end{flushright}


\begin{center}
	Microeconomics 3. Problem Set 4.
\end{center}
\section*{Problem 1}
Consider the exchange economy depicted in Figure 15.B.10(a). Show that the conclusion
of Proposition 17.B.2(v) fails in this example. Which property of preferences (or the lack
thereof) is responsible from this?



\textbf{Solution}


Assume $p^n = \left(1\ \frac{1}{n}\right) \underset{n \to \infty}{\to} (1\ 0)$. Then 
\begin{align*}
x_2(p^n) = \begin{pmatrix}
\bar{\omega_1}\\
0
\end{pmatrix},\ x_1(p^n) = \begin{pmatrix}
x_{11}(p^n)\\
x_{21}(p^n)
\end{pmatrix}
\end{align*}
are demand functions of both consumers, where $x_{11}(p^n)$ and $x_{21}(p^n)$ are the first consumer preferences maximizers over her budget set:
\begin{align*}
\left\{(x_{11}, x_{21}) \in [0, \bar{\omega_2}] \times [\bar{\omega_1}, 0]: x_{11} + \frac{x_{12}}{n} \le \frac{\bar{\omega_2}}{n} \right\}
\end{align*}
Then the excess demand function is:
\begin{align*}
z(p^n) = \begin{pmatrix}
x_{11}(p^n)\\
x_{21}(p^n) - \bar{\omega_2}
\end{pmatrix}
\end{align*}
Since the first consumer strictly prefers any positive amount of good 1 to her initial endowment of this good - 0, hence $x_{11}(p^n) > 0\ \forall\ n$. Moreover, from the budget constraint:
\begin{align*}
x_{11} + \frac{x_{12} - \bar{\omega_2}}{n} \le 0
\end{align*}
that means that $x_{12} - \bar{\omega_2} \le 0$, i.e.
\begin{align*}
\max \left\{z_1(p^n), z_2(p^n) \right\} = x_{11}(p^n)
\end{align*}
$\frac{x_{12} - \bar{\omega_2}}{n} \underset{n \to \infty}{\to} 0$ therefore:
\begin{align*}
B(p^n) \underset{n \to \infty}{\to} \left\{(x_{11}, x_{21}) \in [0, \bar{\omega_2}] \times [\bar{\omega_1}, 0]: x_{11} \le 0 \right\}
\end{align*}
hence maximizer $x_{11}(p^n)^* \to 0$.
Thus, 
\begin{align*}
\max\left\{z_1(p^n), z_2(p^n)\right\} \underset{p^n \to (1\ 0)}{\to} 0 \neq +\infty
\end{align*}
i.e. the conclusion of the proposition fails. It has happend because of lack of strictly monotonicity of preferences of the second consumer.
\section*{Problem 2}
Suppose $\pi_{11} > \pi_{12}$ in Example 19.C.1 in the textbook. Show that an interior Pareto
efficient allocation $x$ lies below the diagonal of the Edgeworth box, i.e., 
\begin{align*}
\frac{x_{11}}{x_{21}} > 1 > \frac{x_{12}}{x_{22}}
\end{align*}
What
else can you say about the relation between the ratios $\frac{\pi_{11}}{\pi_{21}}$, $\frac{\pi_{12}}{\pi_{22}}$ and $\frac{p_1}{p_2}$ in an
Arrow-Debreu equilibrium?



\textbf{Solution}

\begin{align*}
U_1(x_{11}, x_{21}) = \pi_{11}u_1(x_{11}) + (1 - \pi_{11}) u_1(x_{21})\\
U_2(x_{12}, x_{22}) = \pi_{12}u_2(x_{12}) + (1 - \pi_{12}) u_2(x_{22})\\
\end{align*}
Necessary (and sufficient for concave functions) condition for the interior Pareto efficiency is the following:
\begin{align*}
\frac{\partial U_1/\partial x_{11}}{\partial U_1/\partial x_{21}} &= \frac{\partial U_2/\partial x_{12}}{\partial U_1/\partial x_{22}}\\
\frac{\pi_{11}u'_1(x_{11})}{(1-\pi_{11})u'_1(x_{21})} &= \frac{\pi_{12}u'_2(x_{12})}{(1-\pi_{12})u'_2(x_{22})}
\end{align*}
The market clearing conditions imply that:
\begin{align*}
x_{11} + x_{12} = \bar{\omega_1} = 1\\
x_{21} + x_{22} = \bar{\omega_2} = 1\\
\end{align*}
Thus, 
\begin{align*}
\frac{\pi_{11}u'_1(x_{11})}{(1-\pi_{11})u'_1(x_{21})} = \frac{\pi_{12}u'_2(1 - x_{11})}{(1-\pi_{12})u'_2(1-x_{21})}
\end{align*}
If $\pi_{11} > \pi_{12}$ then 
\begin{align}\label{eq1}
\frac{u'_1(x_{11})}{u'_1(x_{21})} < \frac{u'_2(1-x_{11})}{u'_2(1 - x_{21})}
\end{align}
Assume by contradiction that $x_{11} \le x_{21}$ hence $1-x_{11} \ge 1 - x_{21}$ by strict concavity of $u_1(\cdot), u_2(\cdot)$:
\begin{align*}
\frac{u_1'(x_{11})}{u'_1(x_{21})} \ge 1\\
\frac{u'_2(1 - x_{11})}{u_2'(1 - x_{21})} \le 1
\end{align*}
but it contradicts to \eqref{eq1}. Hence $x_{11} > x_{21}$ and $1 - x_{11} = x_{21} < x_{22} = 1 - x_{21}$.


Since at the interior equilibrium:
\begin{align*}
\frac{\partial U_1/\partial x_{11}}{\partial U_1/\partial x_{21}} &= \frac{\partial U_2/\partial x_{12}}{\partial U_1/\partial x_{22}} = \frac{p_1}{p_2}
\end{align*}
that means that
\begin{align*}
\frac{\pi_{11}u'_1(x_{11})}{(1-\pi_{11})u'_1(x_{21})} = \frac{\pi_{12}u'_2(1 - x_{11})}{(1-\pi_{12})u'_2(1-x_{21})} = \frac{p_1}{p_2}\\
x_{11} > x_{21} \to \frac{u_1'(x_{11})}{u_1'(x_{21})} < 1 \to \frac{\pi_{11}}{1 - \pi_{11}} > \frac{p_1}{p_2}
\end{align*}
Similarly, 
\begin{align*}
\frac{u_2'(1 - x_{11})}{u_2'(1 - x_{21})} > 1 \to \frac{\pi_{12}}{1 - \pi_{12}} < \frac{p_1}{p_2}
\end{align*}
i.e.
\begin{align*}
\frac{\pi_{12}}{1 - \pi_{12}} < \frac{p_1}{p_2} < \frac{\pi_{11}}{1 - \pi_{21}}
\end{align*}
\section*{Problem 3}
Solve Exercise 19.C.4 from the textbook. (Note: There is an error in the statement,
only $u_2$ is strictly concave.)


\textbf{Solution}

\begin{enumerate}[(a)]
	\item As in the previous problem, at the interior equilibrium:
	\begin{align*}
	\frac{\pi_{11}u'_1(x_{11})}{(1-\pi_{11})u'_1(x_{21})} = \frac{\pi_{12}u'_2(x_{12})}{(1-\pi_{12})u'_2(x_{22})}
	\end{align*}
	here, however $\pi_{11} = \pi_{12}$ and moreover the first consumer is risk-neutral (i.e. $u_1'(x) = const\ \forall\ x$) hence
	\begin{align*}
	1 = \frac{u'_2(x_{12})}{u'_2(x_{22})}
	\end{align*}
	by stictly concavity of $u_2(\cdot)$ it follows that
	\begin{align*}
	x_{12} = x_{22}
	\end{align*}
	i.e. the second consumer insures completely.
	\item Assume wlog that $\pi_{11} > \pi_{12}$ then
	\begin{align*}
	\frac{\pi_{11}}{1 - \pi_{11}} > \frac{\pi_{12}}{1 - \pi_{12}}\ \to 1 < \frac{u'_2(x_{12})}{u'_2(x_{22})}\ \to x_{12} < x_{22}
	\end{align*}
	i.e. the second consumer does not insure completely, namely:
	\begin{align*}
	\begin{cases}
	x_{12} < x_{22}, \pi_{12} < \pi_{11}\\
	x_{12} > x_{22}, \pi_{12} > \pi_{11}
	\end{cases}
	\end{align*} 
	Without trade the risk-neutral consumer gets:
	\begin{align*}
	\pi_{11} \frac{\bar{\omega_1}}{2} + (1 - \pi_{11})\frac{\bar{\omega_2}}{2}
	\end{align*}
	If the trade occurs the first consumer should maxmize his preferences over his budget set, i.e. solve the following optimization problem:
	\begin{align*}
	\underset{x_{11}, x_{21} \ge 0}{\max}\ \pi_{11}x_{11} + (1 - \pi_{21})x_{21}\\
	s.t.\ p_1x_{11} + p_2x_{22} \le p_1\frac{\bar{\omega_1}}{2} + p_2 \frac{\bar{\omega_2}}{2}
	\end{align*}
	If $\pi_{11} > 0.5$ then solution is:
	\begin{align*}
	x_{11}^* = \frac{\bar{\omega_1}}{2} + \frac{p_2}{p_1}\frac{\bar{\omega}_2}{2}\\
	x_{21}^* = 0
	\end{align*}
	in that case the consumer gets:
	\begin{align*}
	\pi_{11} \left(\frac{\bar{\omega_1}}{2} + \frac{p_2}{p_1}\frac{\bar{\omega}_2}{2}\right) = \pi_{11} \frac{\bar{\omega_1}}{2} + \pi_{11}\frac{1 - \pi_{11}}{\pi_{11}} \frac{\bar{\omega_2}}{2} = \pi_{11} \frac{\bar{\omega_1}}{2} + (1 - \pi_{11})\frac{\bar{\omega_2}}{2}
	\end{align*}
	which is exactly the same value he gets without trade. (the case $\pi_{11} \le 0.5$ is analogous).
\end{enumerate}

\section*{Problem 4}
Consider an Arrow-Debreu exchange economy with two physical goods, two states and
two consumers. Consumers' Bernoulli utility functions are state independent and identical:

\begin{align*}
u_1(x_{s1}) = \frac{1}{2} \ln x_{1s1} + \frac{1}{2}\ln x_{2s1}\ u_2(x_{s2}) = \frac{1}{2} \ln x_{1s2} + \frac{1}{2}\ln x_{2s2}
\end{align*}
Here, $x_{si} = (x_{1si}; x_{2si})$ stands for the consumption bundle of consumer $i$ in state $s$. Consumer
1 assigns probability $1/4$ to state 1 while consumer 2 assigns probability $3/4$ to state 1.
Consumers' endowment vectors are as follows:

\begin{align*}
\omega_1 = (\omega_{111}; \omega_{211}; \omega_{121}; \omega_{221}) = (12; 4; 0; 0);\  \omega_{2} = (\omega_{112}; \omega_{212}; \omega_{122}; \omega_{222}) = (0; 0; 4; 12)
\end{align*}
As usual, $\omega_{lsi}$ stands for consumer $i's$ endowment of good $l$ in state $s$. Find an Arrow-Debreu
equilibrium of this model.


\textbf{Solution}

Consumers are solving the following optimization problems:
\begin{align*}
\underset{x_{111}, x_{211}, x_{121}, x_{221} \ge 0}{\max}\ \ \ \frac{1}{8} (\ln x_{111}+ \ln x_{211}) + \frac{3}{8}(\ln x_{121} + \ln x_{221})\\
s.t.\ \  p_{11}x_{111} + p_{21}x_{211} + p_{12}x_{121} + p_{22}x_{221} \le 12p_{11} + 4p_{21}\\
\\
\underset{x_{112}, x_{212}, x_{122}, x_{222} \ge 0}{\max}\ \ \ \frac{3}{8}(\ln x_{112} + \ln x_{212}) + \frac{1}{8}(\ln x_{122} + \ln x_{222})\\
s.t.\ \  p_{11}x_{112} + p_{21}x_{212} + p_{12}x_{122} + p_{22}x_{222} \le 12p_{22} + 4p_{12}\\
\end{align*}
And moreover market clearing conditions imply:
\begin{align*}
x_{111} + x_{112} = 12 \nonumber\\
x_{211} + x_{212} = 4\nonumber\\
x_{121} + x_{122} = 4\\
x_{221} + x_{222} = 12\nonumber
\end{align*}
Since derivatives of objective functions tend to infinity while some of $x$ tend to 0 hence the solution should lie at the interior. Thus, FOCs are:
\begin{align*}
\begin{cases}
\frac{1}{8x_{111}} - \lambda p_{11} = 0\\
\frac{1}{8x_{211}} - \lambda p_{21} = 0\\
\frac{3}{8x_{121}} - \lambda p_{12} = 0\\
\frac{3}{8x_{221}} - \lambda p_{22} = 0\\
p_{11}x_{111} + p_{21}x_{211} + p_{12}x_{121} + p_{22}x_{221} = 12p_{11} + 4p_{21}
\end{cases} \to \begin{cases}
x_{111} = \frac{12p_{11} + 4p_{21}}{8p_{11}}\\
x_{211} = \frac{12p_{11} + 4p_{21}}{8p_{21}}\\
x_{121} = \frac{3(12p_{11} + 4p_{21})}{8p_{12}}\\
x_{221} = \frac{3(12p_{11} + 4p_{21})}{8p_{22}}
\end{cases}
\end{align*}
Similarly for the second consumer:
\begin{align*}
\begin{cases}
x_{112} = \frac{3(12p_{22} + 4p_{12})}{8p_{11}}\\
x_{212} = \frac{3(12p_{22} + 4p_{12})}{8p_{21}}\\
x_{122} = \frac{12p_{22} + 4p_{12}}{8p_{12}}\\
x_{222} = \frac{12p_{22} + 4p_{12}}{8p_{22}}
\end{cases}
\end{align*}
Plugging market clearing conditions into FOCs:
\begin{align*}
\begin{cases}
\frac{12p_{11} + 4p_{21}}{8p_{11}} + \frac{3(12p_{22} + 4p_{12})}{8p_{11}} = 12\\
\frac{12p_{11} + 4p_{21}}{8p_{21}} + \frac{3(12p_{22} + 4p_{12})}{8p_{21}} = 4\\
\frac{3(12p_{11} + 4p_{21})}{8p_{12}} + \frac{12p_{22} + 4p_{12}}{8p_{12}} = 4\\
\frac{3(12p_{11} + 4p_{21})}{8p_{22}} + \frac{12p_{22} + 4p_{12}}{8p_{22}} = 12
\end{cases}\\
\begin{cases}
-21p_{11} + p_{21} + 3p_{12} + 9p_{22} = 0\\
3p_{11} - 7p_{21} + 3p_{12} + 9p_{22} = 0\\
9p_{11} +3p_{21} - 7p_{12} + 3p_{22} = 0\\
9p_{11} + 3p_{21} + p_{12} - 21p_{22} = 0
\end{cases}
\end{align*}
The solution is:
$p_{11} = p_{22}, p_{21} = p_{12} = 3p_{22}$. If we normalize $p_{11} = 1$ then $p_{22} = 1, p_{12} = p_{21} = 3$.
That is, the equilibrium allocation is:
\begin{align*}
x_1 = \begin{pmatrix}
3\\
1\\
3\\
9
\end{pmatrix},\ x_2 = \begin{pmatrix}
9\\
3\\
1\\
3
\end{pmatrix}
\end{align*}
\section*{Problem 5}
Consider an Arrow-Debreu exchange economy with two physical goods, two states, and
two consumers. Consumers' Bernoulli utility functions are state independent:

\begin{align*}
u_1(x_{s1}) = \ln x_{1s1} + 2 \ln x_{2s1}\ \  u_2(x_{s2}) = 2 \ln x_{1s2} + \ln x_{2s2}
\end{align*}
Both consumers think that the states are equally likely. Consumers' endowment vectors are
equal to each other:
\begin{align*}
\omega_i = (\omega_{11i}; \omega_{21i}; \omega_{12i};\omega_{22i}) = (6; 6; 3; 3) \text{ for } i = 1; 2,
\end{align*}
Find an Arrow-Debreu
equilibrium of this model.


\textbf{Solution}


Consumers are solving the following optimization problems:
\begin{align*}
&\underset{x_{111}, x_{211}, x_{121}, x_{221} \ge 0}{\max}\ \ \ \frac{1}{2}(\ln x_{111} + 2\ln x_{211}) + \frac{1}{2}(\ln x_{121} + 2\ln x_{221})\\
s.t.\ \  p_{11}x_{111}& + p_{21}x_{211} + p_{12}x_{121} + p_{22}x_{221} \le 6p_{11} + 6p_{21} + 3p_{12} + 3p_{22}\\
\\
&\underset{x_{112}, x_{212}, x_{122}, x_{222} \ge 0}{\max}\ \ \ \frac{1}{2}(2\ln x_{112} + \ln x_{212}) + \frac{1}{2}(2\ln x_{122} + \ln x_{222})\\
s.t.\ \  p_{11}x_{112}& + p_{21}x_{212} + p_{12}x_{122} + p_{22}x_{222} \le 6p_{11} + 6p_{21} + 3p_{12} + 3p_{22}\\
\end{align*}
And moreover market clearing conditions imply:
\begin{align*}
x_{111} + x_{112} = 12 \nonumber\\
x_{211} + x_{212} = 12\nonumber\\
x_{121} + x_{122} = 6\\
x_{221} + x_{222} = 6\nonumber
\end{align*}
Since derivatives of objective functions tend to infinity while some of $x$ tend to 0 hence the solution should lie at the interior. Thus, FOCs are:
\begin{align*}
&\begin{cases}
\frac{1}{2x_{111}} - \lambda p_{11} = 0\\
\frac{1}{x_{211}} - \lambda p_{21} = 0\\
\frac{1}{2x_{121}} - \lambda p_{12} = 0\\
\frac{1}{x_{221}} - \lambda p_{22} = 0\\
p_{11}x_{111} + p_{21}x_{211} + p_{12}x_{121} + p_{22}x_{221} \le 6p_{11} + 6p_{21} + 3p_{12} + 3p_{22}
\end{cases} \\
&\begin{cases}
x_{111} = \frac{6p_{11} + 6p_{21} + 3p_{12} + 3p_{22}}{6p_{11}}\\
x_{211} = \frac{6p_{11} + 6p_{21} + 3p_{12} + 3p_{22}}{3p_{21}}\\
x_{121} = \frac{6p_{11} + 6p_{21} + 3p_{12} + 3p_{22}}{6p_{12}}\\
x_{221} = \frac{6p_{11} + 6p_{21} + 3p_{12} + 3p_{22}}{3p_{22}}
\end{cases}
\end{align*}
Similarly for the second consumer:
\begin{align*}
x_{112} = \frac{6p_{11} + 6p_{21} + 3p_{12} + 3p_{22}}{3p_{11}}\\
x_{212} = \frac{6p_{11} + 6p_{21} + 3p_{12} + 3p_{22}}{6p_{21}}\\
x_{122} = \frac{6p_{11} + 6p_{21} + 3p_{12} + 3p_{22}}{3p_{12}}\\
x_{222} = \frac{6p_{11} + 6p_{21} + 3p_{12} + 3p_{22}}{6p_{22}}
\end{align*}
From market clearing conditions:
\begin{align*}
\begin{cases}
-6p_{11} + 2p_{21} + p_{12} + p_{22} = 0\\
2p_{11} - 6p_{21} + p_{12} + 3p_{22} = 0\\
2p_{11} + 2p_{21} - 3p_{12} + p_{22} = 0\\
2p_{11} + 2p_{21} + p_{12} - 3p_{22} = 0
\end{cases}
\end{align*}
The solution is:
\begin{align*}
2p_{11} = 2p_{21} = p_{22} = p_{12}
\end{align*}
Normalize $p_{11} = 1$ then $p_{11} = p_{21} = 1, p_{22} = p_{12} = 2$. And the equilibrium allocation is:
\begin{align*}
x_1 = \begin{pmatrix}
4\\
8\\
2\\
4
\end{pmatrix},\ x_2 = \begin{pmatrix}
8\\
4\\
4\\
2
\end{pmatrix}
\end{align*}
\end{document}