\documentclass[a4paper]{article}
\usepackage[14pt]{extsizes} % 
\usepackage[utf8]{inputenc}
\usepackage{setspace,amsmath}
\usepackage{mathtools}
\usepackage{pgfplots}
\usepackage{titlesec}
\usepackage{pdfpages}
\usepackage[shortlabels]{enumitem}
\usepackage{tikz}
\usetikzlibrary{angles,quotes}
\usepackage{graphicx}
\usepackage{amssymb}
\usepackage{float}
\usepackage[section]{placeins}
\usepackage[makeroom]{cancel}
\usepackage{mathrsfs} % 
\newcommand\numberthis{\addtocounter{equation}{1}\tag{\theequation}}
%\addto\captionsrussian{\renewcommand{\figurename}{Fig.}}
\usepackage{amsmath,amsfonts,amssymb,amsthm,mathtools} 
\newcommand*{\hm}[1]{#1\nobreak\discretionary{}
{\hbox{$\mathsurround=0pt #1$}}{}}
\usepackage{graphicx}  % 
\graphicspath{{images/}{images2/}}  % 
\setlength\fboxsep{3pt} %  \fbox{} 
\setlength\fboxrule{1pt} % \fbox{}
\usepackage{wrapfig} % 
\newcommand{\prob}{\mathbb{P}}
\newcommand{\norma}{\mathscr{N}}
\newcommand{\expect}{\mathbb{E}}
\newcommand{\summa}{\sum_{i=1}^n}
\usepackage[left=7mm, top=20mm, right=15mm, bottom=20mm, nohead, footskip=10mm]{geometry} % 
\usepackage{tikz} % 
\def\myrad{2cm}% radius of the circle
\def\myanga{45}% angle for the arc
\def\myangb{195}
\begin{document} % 
	\begin{flushright}
	\begin{tabular}{r}
		Danil Fedchenko, MAE 2020, group A \\
	\end{tabular}
\end{flushright}


\begin{center}
	Microeconomics 3. Problem Set 4.
\end{center}
\section*{Problem 1}
Consider the exchange economy depicted in Figure 15.B.10(a). Show that the conclusion
of Proposition 17.B.2(v) fails in this example. Which property of preferences (or the lack
thereof) is responsible from this?



\textbf{Solution}


Assume $p^n = \left(1\ \frac{1}{n}\right) \underset{n \to \infty}{\to} (1\ 0)$. Then 
\begin{align*}
x_2(p^n) = \begin{pmatrix}
\bar{\omega_1}\\
0
\end{pmatrix},\ x_1(p^n) = \begin{pmatrix}
x_{11}(p^n)\\
x_{21}(p^n)
\end{pmatrix}
\end{align*}
are demand functions of both consumers, where $x_{11}(p^n)$ and $x_{21}(p^n)$ are the first consumer preferences maximizers over her budget set:
\begin{align*}
\left\{(x_{11}, x_{21}) \in [0, \bar{\omega_2}] \times [\bar{\omega_1}, 0]: x_{11} + \frac{x_{12}}{n} \le \frac{\bar{\omega_2}}{n} \right\}
\end{align*}
Then the excess demand function is:
\begin{align*}
z(p^n) = \begin{pmatrix}
x_{11}(p^n)\\
x_{21}(p^n) - \bar{\omega_2}
\end{pmatrix}
\end{align*}
Since the first consumer strictly prefers any positive amount of good 1 to her initial endowment of this good - 0, hence $x_{11}(p^n) > 0\ \forall\ n$. Moreover, from the budget constraint:
\begin{align*}
x_{11} + \frac{x_{12} - \bar{\omega_2}}{n} \le 0
\end{align*}
that means that $x_{12} - \bar{\omega_2} \le 0$, i.e.
\begin{align*}
\max \left\{z_1(p^n), z_2(p^n) \right\} = x_{11}(p^n)
\end{align*}
$\frac{x_{12} - \bar{\omega_2}}{n} \underset{n \to \infty}{\to} 0$ therefore:
\begin{align*}
B(p^n) \underset{n \to \infty}{\to} \left\{(x_{11}, x_{21}) \in [0, \bar{\omega_2}] \times [\bar{\omega_1}, 0]: x_{11} \le 0 \right\}
\end{align*}
hence maximizer $x_{11}(p^n)^* \to 0$.
Thus, 
\begin{align*}
\max\left\{z_1(p^n), z_2(p^n)\right\} \underset{p^n \to (1\ 0)}{\to} 0 \neq +\infty
\end{align*}
i.e. the conclusion of the proposition fails. It has happend because of lack of strictly monotonicity of preferences of the second consumer.
\section*{Problem 2}
Suppose $\pi_{11} > \pi_{12}$ in Example 19.C.1 in the textbook. Show that an interior Pareto
efficient allocation $x$ lies below the diagonal of the Edgeworth box, i.e., 
\begin{align*}
\frac{x_{11}}{x_{21}} > 1 > \frac{x_{12}}{x_{22}}
\end{align*}
What
else can you say about the relation between the ratios $\frac{\pi_{11}}{\pi_{21}}$, $\frac{\pi_{12}}{\pi_{22}}$ and $\frac{p_1}{p_2}$ in an
Arrow-Debreu equilibrium?



\textbf{Solution}

\begin{align*}
U_1(x_{11}, x_{21}) = \pi_{11}u_1(x_{11}) + (1 - \pi_{11}) u_1(x_{21})\\
U_2(x_{12}, x_{22}) = \pi_{12}u_2(x_{12}) + (1 - \pi_{12}) u_2(x_{22})\\
\end{align*}
Necessary (and sufficient for concave functions) condition for the interior Pareto efficiency is the following:
\begin{align*}
\frac{\partial U_1/\partial x_{11}}{\partial U_1/\partial x_{21}} &= \frac{\partial U_2/\partial x_{12}}{\partial U_1/\partial x_{22}}\\
\frac{\pi_{11}u'_1(x_{11})}{(1-\pi_{11})u'_1(x_{21})} &= \frac{\pi_{12}u'_2(x_{12})}{(1-\pi_{12})u'_2(x_{22})}
\end{align*}
The market clearing conditions imply that:
\begin{align*}
x_{11} + x_{12} = \bar{\omega_1} = 1\\
x_{21} + x_{22} = \bar{\omega_2} = 1\\
\end{align*}
Thus, 
\begin{align*}
\frac{\pi_{11}u'_1(x_{11})}{(1-\pi_{11})u'_1(x_{21})} = \frac{\pi_{12}u'_2(1 - x_{11})}{(1-\pi_{12})u'_2(1-x_{21})}
\end{align*}
If $\pi_{11} > \pi_{12}$ then 
\begin{align}\label{eq1}
\frac{u'_1(x_{11})}{u'_1(x_{21})} < \frac{u'_2(1-x_{11})}{u'_2(1 - x_{21})}
\end{align}
Assume by contradiction that $x_{11} \le x_{21}$ hence $1-x_{11} \ge 1 - x_{21}$ by strict concavity of $u_1(\cdot), u_2(\cdot)$:
\begin{align*}
\frac{u_1'(x_{11})}{u'_1(x_{21})} \ge 1\\
\frac{u'_2(1 - x_{11})}{u_2'(1 - x_{21})} \le 1
\end{align*}
but it contradicts to \eqref{eq1}. Hence $x_{11} > x_{21}$ and $1 - x_{11} = x_{21} < x_{22} = 1 - x_{21}$.



\end{document}