\documentclass[a4paper]{article}
\usepackage[14pt]{extsizes} % 
\usepackage[utf8]{inputenc}
\usepackage{setspace,amsmath}
\usepackage{mathtools}
\usepackage{pgfplots}
\usepackage{titlesec}
\usepackage{pdfpages}
\usepackage[shortlabels]{enumitem}
\usepackage{tikz}
\usetikzlibrary{angles,quotes}
\usepackage{graphicx}
\usepackage{amssymb}
\usepackage{float}
\usepackage[section]{placeins}
\usepackage[makeroom]{cancel}
\usepackage{mathrsfs} % 
\newcommand\numberthis{\addtocounter{equation}{1}\tag{\theequation}}
%\addto\captionsrussian{\renewcommand{\figurename}{Fig.}}
\usepackage{amsmath,amsfonts,amssymb,amsthm,mathtools} 
\newcommand*{\hm}[1]{#1\nobreak\discretionary{}
{\hbox{$\mathsurround=0pt #1$}}{}}
\usepackage{graphicx}  % 
\graphicspath{{images/}{images2/}}  % 
\setlength\fboxsep{3pt} %  \fbox{} 
\setlength\fboxrule{1pt} % \fbox{}
\usepackage{wrapfig} % 
\newcommand{\prob}{\mathbb{P}}
\newcommand{\norma}{\mathscr{N}}
\newcommand{\expect}{\mathbb{E}}
\newcommand{\summa}{\sum_{i=1}^n}
\usepackage[left=7mm, top=20mm, right=15mm, bottom=20mm, nohead, footskip=10mm]{geometry} % 
\usepackage{tikz} % 
\def\myrad{2cm}% radius of the circle
\def\myanga{45}% angle for the arc
\def\myangb{195}
\begin{document} % 
	\begin{flushright}
	\begin{tabular}{r}
		Danil Fedchenko, MAE 2020, group A \\
	\end{tabular}
\end{flushright}


\begin{center}
	Microeconomics 4. Problem Set 1.
\end{center}
\section*{1\ The Doctor and the Patient}
Consider the cheap talk signalling model a-la Crawford and Sobel (1982). Suppose that
there is a patient that does not feel well and goes to see a doctor. The doctor should
decide which treatment $a$ the patient needs. The type of treatment will depend on
the health state of the patient $\theta$, which is only privately observed by the patient. The
doctor's prior about the patient's health is $\theta \sim U[0; 1]$. The patient can send a costless,
non-binding and non-verifiable signal $s \in [0; 1]$ about his health state (i.e. explain how
he feels). The doctor cares about her reputation, so she wants to prescribe a appropriate
treatment for the patient's health: $U
D(a; \theta) = -(a-\theta)^2$. Instead, the patient, that is
very hypochondriac, wants to be overtreated: $U^P(a; \theta) = -(a - (\theta + b))^2$, where $b > 0$ is
the bias towards excessive medication.
\begin{enumerate}
	\item Assume that b is such that there exist a 2-partitional equilibrium $(p = 2)$. Characterize this equilibrium.
	\item Under which condition will such an equilibrium exist?
	\item Is this the only equilibrium that exists? If yes, argue why. If no, characterize the
	other(s) equilibrium (equilibria).
	\item Compute the expected payoff of the doctor and the patient. How do the payoffs
	change with $b$?
\end{enumerate}

\textbf{Solution}

\begin{enumerate}
	\item Assume that for $\theta < \theta_1$ the patient sends $s_1$ e.g. $s_1 = 0$, while for $\theta \ge \theta_1$ the patient sends $s_2 > s_1$ e.g. $s_2 = \theta_1$. Then the doctor chooses treatment as a maximizer of:
	\begin{align*}
	\begin{cases}
	\underset{a}{\max}\ \int_{0}^{\theta_1} -(a - \theta)^2d\theta, s < \theta_1\\
	\underset{a}{\max}\ \int_{\theta_1}^{1} -(a - \theta)^2d\theta, s \ge \theta_1\\
	\end{cases}
	\end{align*}
	and the solution is obviously
	\begin{align*}
	a(s) = \begin{cases}
	\frac{\theta_1}{2}, s < \theta_1\\
	\frac{1 + \theta_1}{2}, s \ge \theta_1
	\end{cases}
	\end{align*}
	The patient should send:
	\begin{align*}
	\begin{cases}s_1, -\left(\frac{\theta_1}{2} - \theta - b\right)^2 > -\left(\frac{\theta_1 + 1}{2} - \theta - b\right)^2\\
	s_2, \text{ otherwise}
	\end{cases}
	\end{align*}
	That means that $\theta_1$ should be such that the patient becomes indifferent between sending $s_1$ and $s_2$ i.e.
	\begin{align*}
	\frac{\frac{\theta_1}{2} - b + \frac{\theta_1+1}{2} - b}{2} &= \theta_1\\
	\theta_1 &= \frac{1}{2} - 2b
	\end{align*}
	\item 
	\begin{align*}
	\frac{1}{2} - 2b \ge 0\\
	b \le \frac{1}{4}
	\end{align*}
	\item No, there always exists a "babbling" equilibrium. In which the patient always sends the same, uninformative signal and doctor just chooses action which maximize his expected payoff taking in account doctor's prior beliefs about type. Given that, patient indifferent between any signal (he cannot change anything) hence he has not profitable deviation.
	\item The expected payoff of the doctor is:
	\begin{align*}
	\pi^D = \int_{0}^{\frac{1}{2} - 2b} - \left(\frac{1}{4} - b - \theta\right)^2d\theta + \int_{\frac{1}{2} - 2b}^{1}- \left(\frac{3}{4} - b - \theta\right)^2d\theta = -2\frac{\left(\frac{1}{4} - b\right)^3}{3} -2\frac{\left(\frac{1}{4} + b\right)^3}{3}
	\end{align*}
	The expected payoff of the patient is:
	\begin{align*}
	\pi^P = \int_{0}^{\frac{1}{2} - 2b} - \left(\frac{1}{4} - 2b - \theta\right)^2d \theta + \int_{\frac{1}{2} - 2b}^1 - \left(\frac{3}{4} - 2b - \theta\right)^2d \theta = -\frac{1}{3 \cdot 2^5} + \\
	+\frac{1}{3} \left(\left(\frac{3}{4} - 2b\right)^3 + \left(\frac{1}{4} - 2b\right)\right)
	\end{align*}
	\begin{align*}
	&\frac{\partial \pi^D}{\partial b} = 2\left(\frac{1}{4} - b - \frac{1}{4} - b\right)\left(\frac{1}{4} - b + \frac{1}{4} + b\right) = -2b < 0\\
	&\frac{\partial \pi^P}{\partial b} = - 2 \left(\frac{3}{4} - 2b - \frac{1}{4} + 2b\right)\left(\frac{3}{4} - 2b + \frac{1}{4} - 2b\right) = -1 + 4b < 0
	\end{align*}
	That is payoffs goes down once $b$ increases.
\end{enumerate}
\section*{ 2 Labour Market}
There are two types of workers at the labor market, $i = 1, 2$. A worker of type $i$ has a utility function
\begin{align*}
u_i(w, x) = w - c_i(x)
\end{align*}
where $w$ is the wage paid to the worker, $x$ is the output he produces, and $c_i(x)$ is a convex
and increasing cost function, $c'_1(x) < c'_2(x)$, $c_i(0) = 0$, $c'_i(0) = 0$, $c'_i(\infty) = \infty$ for $i = 1, 2$.
The share of workers of the first type is given by $\lambda$.

\begin{enumerate}
	\item Assume that there is a single hiring firm at the market, and it does not observe the
	workers' types. The profit of the firm is given by the value of output less the wage
	payment
	\begin{align*}
	\pi = x - w
	\end{align*}
	Assume that the reservation wage of both workers is zero. Characterize the monopolistic screening contracts $(w_i, x_i)$ offered by the firm.
	\item Now assume that there are multiple firms in the market, but none of them observes
	the workers' types. The firms are competing in hiring workers. Characterize the
	competitive screening SPNE.
	\item How would your answers to (1) and (2) change if the firms can observe the types
	of the workers? Why can the first best be (or not be) achieved in each of these
	scenarios?
	\item Assume that $c_i(x) = \frac{i}{2}x^2,\ i = 1,\ 2$ and $\lambda = 1/2$. Derive the levels of output and the
	wages in your monopolistic and competitive screening problems with unobservable
	and observable worker types (i.e., in (1)-(3))
\end{enumerate}


\textbf{Solution}

\begin{enumerate}
	\item The monopoly's problem is:
	\begin{align*}
	\underset{(x_1, w_1), (x_2, w_2)}{\max}\ \lambda(x_1 - w_1) + (1 - \lambda)(x_2 - w_2)
	\end{align*}
	Since outside options of workers are 0, the optimal contracts monopoly proposes to workers should be such that
	\begin{align*}
	w_1^* = c_1(x_1^*)\\
	w_2^* = c_2(x_2^*)
	\end{align*}
	Assuming interior solution FOC is:
	\begin{align*}
	\begin{cases}
	1 - c_1'(x_1^*) = 0\\
	1 - c_2'(x_2^*) = 0
	\end{cases}\\
	\end{align*}
	\begin{align*}
	c_1'(x_1^*) = c_2'(x_2^*) \to x_1^* > x_2^* \to c_1(x_1^*) > c_1(x_2^*) \to w_2^* - c_1(x_2^*) > w_1^* - c_1(x_1^*) = 0
	\end{align*}
	That means that at any interior solution high type has incentive to deviate and mimic to low type. Hence we should add incentive compatibility constraints. The IC constraints are:
	\begin{align*}
	\begin{cases}
	w_1^* - c_2(x_1^*) \le w_2^* - c_2(x_2^*)\\
	w_2^* - c_1(x_2^*) \le w_1^* - c_1(x_1^*)
	\end{cases}\\
	\begin{cases}
	c_1(x_1^*) - c_2(x_1^*) \le 0\\
	c_2(x_2^*) - c_1(x_2^*) \le 0
	\end{cases}
	\end{align*}
	Since $c_2(x) > c_1(x)\ \forall\ x > 0$ ($g(x) = c_2(x) - c_1(x)$, $g(0) = 0, g'(x) = c'_2(x) - c_1'(x) > 0$ hence $g(x)$ is strictly increasing, since $g(0) = 0$ then $g(x)>0\ \forall\ x > 0$) it follows that $x_2^* = 0$. That means that monopoly should offer the following contracts:
	\begin{align*}
	&(0, 0)\ \text{ which is supposed to be for low type}\\
	&(w_1 = c_1(c_1'^{-1}(1)), x_1 = c_1'^{-1}(1))\ \text{ which is supposed to be for high type}
	\end{align*}
	\item At the competitive equilibrium firms should earn 0 profit. That means that if firms can observe types they should offer the following contracts to both types of workers:
	\begin{align*}
	(x_1, x_1), (x_2, x_2)
	\end{align*}
	Since firms are competing for workers, at the equilibrium both firms should offer workers such contracts that would maximize their utility (otherwise, if any of firm offer another contract, the rival firm can deviate, offer maximizing utility contract, and hire all workers). Of course if firms offer maximizing utility contracts workers have no incentives to deviate and choose a contract which was supposed to be for another type. As a result the firms should propose contracts:
	\begin{align*}
	(c_1'^{-1}(1), c_1'^{-1}(1))\\
	(c_2'^{-1}(1), c_2'^{-1}(1))
	\end{align*}
	\item If monopoly can distinguish between workers, it should propose each type a contract which would maximize its profit and leave each type of workers with zero utility. That is, the contracts should be:
	\begin{align*}
	(w_1 = c_1(c_1'^{-1}(1)), x_1 = c_1'^{-1}(1))\ \text{ high type}\\
	(w_2 = c_2(c_2'^{-1}(1)), x_2 = c_2'^{-1}(1))\ \text{ low type}
	\end{align*}
	If competitive firms can observe types of workers they anyway have to propose such contracts which would maximize workers' utility because otherwise the rival firm hire all workers. That means that firms should propose the same contracts as in the case of unobservable types. That is, first best is achieved under competition and is not achieved under monopoly.
	\item \begin{itemize}
		\item Monopoly, observable types:
		\begin{align*}
		\underset{x_1, x_2}{\max}\ \frac{1}{2}\left(x_1 -\frac{x_1^2}{2}\right) + \frac{1}{2} \left(x_2 - x_2^2\right)\\
		x_1^* = 1,\ x_2^* = \frac{1}{2}\\
		w_1^* = \frac{1}{2},\ w_2^* = \frac{1}{4}
		\end{align*}
		\item Monopoly, unobservable types:
		\begin{align*}
		\underset{x_1, x_2}{\max}\ \frac{1}{2}\left(x_1 -\frac{x_1^2}{2}\right) + \frac{1}{2} \left(x_2 - x_2^2\right)\\
		s.t.\ 	\begin{cases}
		\frac{x_1^2}{2} - x_1^2 \le 0\\
		x_2^2 - \frac{x_2^2}{2} \le 0
		\end{cases}\\
		x_2^* = 0, x_1^* = 1\\
		w_2 = 0, w_1 = \frac{1}{2}
		\end{align*}
		\item Competition, observable types:
		\begin{align*}
		x_1^* = 1, w_1^* = 1\\
		x_2^* = \frac{1}{2}, w_2^* = \frac{1}{2}
		\end{align*}
		\item Competition, unobservable types:
		\begin{align*}
		x_1^* = 1, w_1^* = 1\\
		x_2^* = \frac{1}{2}, w_2^* = \frac{1}{2}
		\end{align*}
	\end{itemize}
\end{enumerate}
\section*{3 Insurance}
Consider the competitive screening model of the insurance market with two types of
consumers $H$ and $L$. All consumers have initial wealth $W > 0$ and may face a loss $d > 0$,
$d < W$. The probability of facing the loss is $p_L$ for the $L$ type of the consumer, and
$p_H$ for the $H$ type, where $p_H > p_L$. Multiple competitive insurance firms offer insurance
contracts against the loss to either type of the consumer. The probabilities of loss and the
utility functions for both types of consumers are common knowledge to all participants.
For parts (1) and (2), assume that all consumers have the same strictly concave
(Bernoulli) utility of wealth $u(W)$ with $u'(W) > 0$ and $u''(W) < 0$. That is, the two types
of consumers are distinguished only by the difference between $p_L$ and $p_H$
\begin{enumerate}
	\item  Assume that the types of the consumers are publicly observable. Draw a graph
	representing the equilibrium outcome of this model. Explain. Derive the utility levels offered to either type of the consumer in this equilibrium.
	\item Now assume that the types of the consumers are only known to the consumers themselves. Draw a graph to represent the separating equilibrium of this model. Explain the reasoning behind your graph. Compare the outcome to (1) and comment on
	the source of difference/similarity.
	\item Now suppose the $H$ types are more risk averse - with utility function $v(W)$ instead
	of $u(W)$. As above, probabilities of loss and the utility functions for both types of
	consumers are common knowledge, but the types of the consumers are only known
	to the consumers themselves. Draw a new graph to show the effect of this change in
	utility function on the contracts in the separating equilibrium. Suggest an intuitive
	explanation for why this change from $u(W)$ to $v(W)$ increases or reduces the utility
	for good types in the separating equilibrium.
\end{enumerate}


\textbf{Solution}

\begin{enumerate}
	\item Both consumers can happen to be in two states of the world:
	\begin{align*}
	W_g &= W - t\\
	W_b &= W - d + r - t
	\end{align*}
	And they derive the following expected utility function:
	\begin{align*}
	V(W_b, W_g) = p_i U(W_b) + (1 - p_i)U(W_g),\ i \in \{L, H\}
	\end{align*}
	Insurance companies' profit is equal to:
	\begin{align*}
	\pi = t - pr
	\end{align*}
	and should be equal to zero. That means that
	\begin{align*}
	W - W_g - p(W_b + d - W_g) = 0\\
	(1 - p)W_g + pW_b = W - pd
	\end{align*}
	Hence break-even lines for high and low types are given as:
	\begin{align*}
	W_b = -\frac{1-p_H}{p_H}W_g + \frac{W}{p_H} -d\\
	W_b = -\frac{1-p_L}{p_L}W_g + \frac{W}{p_L} -d\\	
	\end{align*}
	At the same time, slopes of indifferent curves of consumers are given by:
	\begin{align*}
	\frac{dW_b}{dW_g} = - \frac{\frac{dV}{dW_g}}{\frac{dV}{dW_b}} = -\frac{1-p_i}{p_i}\frac{U'(W_g)}{U'(W_b)}
	\end{align*}
	In complete information case insurance companies can observe the type, hence equilibrium $(W_g^H, W_b^H), (W_g^L, W_b^L)$ should lie on the break-even lines. In other words,
	\begin{align*}
	\frac{dW_b}{dW_g} = - \frac{1-p_i}{p_i}\frac{U'(W_g)}{U'(W_b)} = -\frac{1 - p_i}{p_i}
	\end{align*}
	hence by properties of the utility function $W_g = W_b$ for both types. Thus, both types get full insurance in complete information case. The figure below illustrates the case.
	\begin{figure}[H]
		\centering
		\includegraphics[width=0.8\textwidth]{plotdraft}
		\caption{}\label{fig1}
	\end{figure}
Since $W_g^H = W_b^H, W_g^L = W_b^L$ that means that both types at the equilibrium derive the following utility levels:
\begin{align*}
p_iU(W_b^i) + (1 - p_i)U(W_b^i) = U(W_b^i) = U(W - p_id),\ i \in \{H, L\}
\end{align*}
\item In case of incomplete information at the equilibrium among other things the incentive compatibility constraints should hold. That means that 
\begin{align*}
p_L U(W - d + r_L - t_L) + (1 - p_L)U(W - t_L) \ge p_L U(W - d + r_H - t_H) + \\
+(1 - p_L)U(W - t_H)\\
p_H U(W - d + r_H - t_H) + (1 - p_H)U(W - t_H) \ge p_H U(W - d + r_L - t_L) + \\
+ (1 - p_H)U(W - t_L)\\
\end{align*}
If we assume that as in complete information case $W_g^H = W_b^H, W_g^L = W_b^L$ then IC imply:
\begin{align*}
U(W^L_g) \ge U(W^H_g)\\
U(W^H_g) \ge U(W^L_g)\\
\end{align*}
but by properties of utility functions this implies that $W^L_g = W^H_g$ which is not a separating equilibrium. Thus, at any separating equilibrium the high type should be offered his complete information contract $(W_g^H, W_g^H)$ while the incentive compatibility constraints imply that:
\begin{align*}
p_HU(W_b^H) + (1 - p_H)U(W_g^H) = p_H U(W_b^L) + (1 - p_H)U(W_g^L)\\
U(W - p_Hd) = p_HU(W_b^L) + (1 -p_H)U(W_g^L)
\end{align*}
This separating equilibrium is illustrated below
	\begin{figure}[H]
	\centering
	\includegraphics[width=0.8\textwidth]{plotdraft}
	\caption{}\label{fig2}
\end{figure}
The high risk type obtains the same utility level as in complete information case, whereas the low risk type becomes strictly worse-off, since $W^g_L \neq W_b^L$. The source of this is the existence of high risk individual who wants to mimic to low risk type and make insurance companies to lose money.
\item If high type is more risk averse, then for a given wealth level in good state he will demand more wealth in bad state comparing with utility $u(W)$ to maintain the same utility level. That means that new separating equilibrium looks like as follows:
	\begin{figure}[H]
	\centering
	\includegraphics[width=0.8\textwidth]{plotdraft}
	\caption{}\label{fig3}
\end{figure}
As we can see, in new equilibrium low type obtain higher level of utility, however still less than in the complete information case. Intuitively it happens because once the high risk type becomes more risk averse, the amount of riskier (with not complete insurance) contracts which attract this type decreases. As a result, now the low type becomes happier, obtaining the contract which is closer to complete insurance. 
\end{enumerate}
\end{document}