\documentclass[a4paper]{article}
\usepackage[14pt]{extsizes} % 
\usepackage[utf8]{inputenc}
\usepackage{setspace,amsmath}
\usepackage{mathtools}
\usepackage{pgfplots}
\usepackage{titlesec}
\usepackage{pdfpages}
\usepackage{makecell}
\usepackage[shortlabels]{enumitem}
\usepackage{tikz}
\usepackage{multirow}
\usetikzlibrary{angles,quotes}
\usepackage{graphicx}
\usepackage{xcolor,colortbl}
\usepackage{amssymb}
\usepackage{float}
\usepackage[section]{placeins}
\usepackage[makeroom]{cancel}
\usepackage{mathrsfs} % 
\newcommand\numberthis{\addtocounter{equation}{1}\tag{\theequation}}
%\addto\captionsrussian{\renewcommand{\figurename}{Fig.}}
\usepackage{amsmath,amsfonts,amssymb,amsthm,mathtools} 
\newcommand*{\hm}[1]{#1\nobreak\discretionary{}
{\hbox{$\mathsurround=0pt #1$}}{}}
\usepackage{graphicx}  % 
\graphicspath{{images/}{images2/}}  % 
\setlength\fboxsep{3pt} %  \fbox{} 
\setlength\fboxrule{1pt} % \fbox{}
\usepackage{wrapfig} % 
\newcommand{\prob}{\mathbb{P}}
\newcommand{\norma}{\mathscr{N}}
\newcommand{\expect}{\mathbb{E}}
\newcommand{\summa}{\sum_{i=1}^n}
\usepackage[left=7mm, top=20mm, right=15mm, bottom=20mm, nohead, footskip=10mm]{geometry} % 
\usepackage{tikz} % 
\begin{document} % 
\section{Enough undeclared money to pay the worker under the table}
This is the case once $e\theta - y' \ge \beta$.
\subsection{Enforceability problem}
\begin{align*}
&\underset{\beta', e, y'}{\max}\ \pi = y'(1 - t) - \beta'(1-t)(1 - \gamma) + \gamma[e\theta - y' - c(e)] - pk(e\theta - y')\\
s.t.\ &\frac{\delta \pi}{1 - \delta} \ge (1-t)\beta' + \gamma \beta\ (DE) \\
&e\theta - y' \ge \beta
\end{align*}
Using the (IC), (DE) becomes
\begin{align*}
\delta y'(1 - t - \gamma) + \delta \gamma e \theta - \delta p k(e \theta - y') \ge \gamma c(e) + \beta'(1-t)(1 - \gamma)
\end{align*}
Then the problem is
\begin{align*}
\underset{e, y', \beta'}{\max}\ &y'(1 - t) - \beta'(1-t)(1 - \gamma) + \gamma[e\theta - y' - c(e)] - pk(e\theta - y')\\
s.t.\ &\delta y'(1 - t - \gamma) + \delta \gamma e \theta - \delta p k(e \theta - y') \ge \gamma c(e) + \beta'(1-t)(1 - \gamma)\ \to \lambda \\
& e\theta - y' \ge c(e) - (1 - t)\beta'\ \to \mu \\
& \beta', y', e \ge 0\ \to \psi_1, \psi_2, \psi_3\ \\
\end{align*}

Then FOCs are 
\begin{align}
\frac{\partial \pi}{\partial e} &= \gamma \theta - \gamma c'(e)- \theta p k'(e \theta - y') + \lambda \delta \gamma \theta - \lambda \delta \theta pk'(e \theta - y') - \lambda \gamma c'(e) + \mu \theta - \mu c'(e) + \psi_3= 0\label{eq1}\\
\frac{\partial \pi}{\partial y'} &= (1 - t - \gamma)+pk'(e\theta - y') + \lambda \delta (1 - t - \gamma) +\lambda \delta pk'(e\theta - y') - \mu + \psi_2 = 0\label{eq2}\\
\frac{\partial \pi}{\partial \beta'} &= -(1-t)(1-\gamma) - \lambda (1-t)(1-\gamma) + \mu(1 - t) + \psi_1= 0\label{eq3}\\
&\lambda, \mu, \psi_1, \psi_2, \psi_3 \ge 0\nonumber\\
&+\text{ slackness complementary conditions}\nonumber
\end{align}

\subsubsection{The declared bonus is zero}
It is clear that if $e \theta - y' \ge c(e)$ then $\beta' = 0$. Indeed, assume the contrary, $\beta' > 0$ then the principal can slightly decrease $\beta'$. This cannot violate the (DE) constraint. Additionally, if $e\theta - y' \ge c(e)$ then it is for sure $\ge c(e) - (1 - t)\beta'$ for $\beta'>0$, thus, this constraint cannot be violated either. But the profit increases. Hence, $\beta'>0$ is not optimal. Intuitively: if there is enough undeclared money to fully compensate the worker's cost, it is optimal to pay this compensation under the table, since this income is tax free for the worker. The question is, what if it is not optimal for the firm to make the difference $e\theta - y'$ too large (e.g. more than $c(e)$) because the punishment for tax evasion depends positively upon it?


Let us firstly consider the case $\beta'=0$. Moreover, suppose that $e \theta - y' > c(e)\ \to \mu = 0$, and that the (DE) constraint does not bind. This case is equivalent to the case without the enforceability problem, thus, the optimal level of effort and the amount of declared revenue are determined by
\begin{align}
c'(e) &= \frac{(1 - t)\theta}{\gamma}\label{eq4}\\
pk'(e\theta - y') &= - (1 - t - \gamma) \label{eq5}
\end{align}
Assuming that the (DE) binds, FOCs become
\begin{align*}
\frac{\partial \pi}{\partial e} &= (1 + \lambda \delta)(\gamma \theta - \theta p k'(e \theta - y')) -(1+ \lambda) \gamma c'(e)= 0\\
\frac{\partial \pi}{\partial y'} &= (1 + \lambda \delta)(1 - t - \gamma+pk'(e\theta - y')) = 0\\
&\text{if }1-t < \gamma \text{ then }c'(e) = \frac{(1+\lambda \delta)\theta(1-t)}{\gamma (1 + \lambda)} \underset{\delta < 1}{<} \frac{\theta(1-t)}{\gamma}\\
&pk'(e \theta - y') = -(1 - t - \gamma)
\end{align*}
(Note that $\lambda$ can be found from the DE constraint which is binding). Comparative statics are the following:
\begin{itemize}
	\item An increase in $\theta$ increases $e \theta$ directly and indirectly (through FOC), however, now the indirect influence is smaller than in case of the absence of the enforceability problem. This is because the increase in $e$ tightens the (DE) constraint, so it is harder for the firm to incentivize efforts. Also, $y'$ should increase to keep the FOC satisfied.
	\item An increase in $\gamma$ (or $t$) decreases $e$ through FOC, however, now the decrease is smaller than in case of the absence of the enforceability problem. $y'$ should also decrease to keep FOC satisfied.
	\item An increase in $\lambda$ also decreases $e$: once the "shadow" price of relaxing the (DE) constraint goes up - it becomes more costly to incentivize effort. Furthermore, comparing cases $\lambda = 0$ and $\lambda > 0$ we can conclude that if the latter case implies less effort, but in both cases $pk'(e\theta - y') = -(1 - t - \gamma)$ then, other things equal, $y'$ in the latter case should also be smaller. That is the presence of the (DE) constraint not only decreases the effort level but also decreases the declared revenue.
	\item An increase in $p$ does not affect $e$, but increases $e \theta - y'$, which means that $y'$ increases.
\end{itemize}
Assume that $\mu > 0$. That means that $e\theta - y' = c(e)$. If the (DE) constraint is not binding then FOCs become
\begin{align}
\frac{\partial \pi}{\partial e} &= \gamma \theta- \gamma c'(e) - \theta p k'(e \theta - y') + \mu \theta - \mu c'(e) = 0\nonumber\\
\frac{\partial \pi}{\partial y'} &= (1 - t - \gamma)+pk'(e\theta - y') - \mu  = 0\nonumber\\
c'(e) &= \frac{\theta(1 - t)}{\mu + \gamma}\label{eq6}\\
pk'(e\theta - y') &= \mu - (1 - t - \gamma)\label{eq7}
\end{align}
if the (DE) constraint is binding then 
\begin{align*}
\frac{\partial \pi}{\partial e} &= (1 + \lambda \delta)(\gamma \theta - \theta p k'(e \theta - y')) - (1+\lambda) \gamma c'(e)+ \mu \theta - \mu c'(e) = 0\\
\frac{\partial \pi}{\partial y'} &= (1+\lambda \delta)(1 - t - \gamma+pk'(e\theta - y')) - \mu = 0\\
c'(e) &= \frac{\theta(1-t)(1+\lambda \delta)}{(1 + \lambda)\gamma + \mu}\\
pk'(e\theta - y') &= \frac{\mu}{1+\lambda \delta} - (1 - t- \gamma)
\end{align*}

Denote $e_{\mu = 0}(e_{\mu > 0}), y'_{\mu=0}(y'_{\mu>0})$ the effort level and the declared revenue in case $\mu = 0\ (\mu > 0)$. Then from \eqref{eq4} and \eqref{eq6} it follows that $e_{\mu >0} \le e_{\mu = 0}$. However from \eqref{eq5} and \eqref{eq7}:
\begin{align*}
c(e_{\mu > 0}) = e_{\mu>0}\theta - y'_{\mu>0} \ge e_{\mu=0}\theta - y'_{\mu=0} \ge c(e_{\mu=0})
\end{align*}
which means that $e_{\mu>0} \ge e_{\mu = 0}$ which is impossible unless $\mu \neq 0$. Thus, $\underline{\mu = 0}$.
\subsubsection{The declared bonus is positive}
If $c(e) - (1-t)\beta' \le e \theta - y' < c(e)$ then $\beta'$ can (and should) be positive.
Assuming in addition $y' > 0, e > 0$ then $\psi_1 = 0, \psi_2 = 0, \psi_3= 0$ and hence from \eqref{eq3}:
\begin{align*}
(1 + \lambda)(1 - \gamma) = \mu > 0\ \to e \theta - y' = c(e) - (1 - t)\beta'
\end{align*}
Then if the (DE) constraint does not bind, $\lambda = 0, \mu = 1 - \gamma$ and FOCs imply:
\begin{align*}
\frac{\partial \pi}{\partial e} &= \theta - c'(e) - \theta p k'(e\theta - y') = 0\\
\frac{\partial \pi}{\partial y'} &= (1 - t - \gamma)+pk'(e\theta - y') - (1-\gamma) = 0
\end{align*}
\begin{align*}
c'(e) &= \theta(1 - t)\\
pk'(e\theta - y') &= t\\
\beta' &= \frac{c(e) - (e\theta - y')}{1-t}
\end{align*}


Denote this effort level and corresponding declared revenue as $e_{\beta' >0}$ and $y'_{\beta'>0}$ respectively. Also denote the effort level and corresponding revenue from \eqref{eq4} and \eqref{eq5} as $e_{\beta'=0}, y'_{\beta'=0}$. It follows that
\begin{align*}
e_{\beta'=0} &> e_{\beta'>0}\\
e_{\beta'>0}\theta - y'_{\beta'>0} &> e_{\beta'=0}\theta - y'_{\beta'=0} \ge c(e_{\beta'=0})
\end{align*}
which is possible only if $e_{\beta'>0}\theta - y'_{\beta'>0} \ge c(e_{\beta'>0})$ but this case has been considered above and implies that $\beta' = 0$.


Assume that (DE) does bind. That means that $\lambda > 0$.
\begin{align*}
\frac{\partial \pi}{\partial e} &= (1 + \lambda \delta)(\gamma \theta - \theta p k'(e \theta - y')) - (1+\lambda)c'(e) + (1+\lambda)(1-\gamma)\theta = 0\\
\frac{\partial \pi}{\partial y'} &= (1+\lambda \delta)(1 - t - \gamma+pk'(e\theta - y')) - (1+\lambda)(1-\gamma) = 0\\
c'(e) &= \frac{(1+\lambda \delta)\theta(\gamma - pk'(e \theta - y'))+ (1+\lambda)(1-\gamma)\theta}{1+\lambda} = \frac{\theta(1-t)(1 + \lambda \delta)}{1+\lambda}\\
pk'(e\theta - y') &= \frac{(1+\lambda)(1-\gamma)}{1+\lambda \delta} - (1 - t - \gamma)
\end{align*}


Using the same logic, one can show that this case also implies $e \theta - y' \ge c(e)$ which, in turns, means that $\underline{\beta' = 0}$.

In conclusion, if there is enough money to pay the worker under the table, the declared bonus is zero.

The results of this section are summarized by the table below
\begin{center} 
	\begin{tabular}{ |c|c|c|c|c|c| } 
		\hline 
		& \makecell{$\lambda>0$}&\makecell{$\lambda=0$} \\ 
		\hline 
		$\beta' = 0$ & \makecell{$c'(e) = \frac{(1+\lambda\delta)\theta(1-t)}{\gamma(1 + \lambda)}$\\\\$pk'(e\theta - y') =$\\$ -(1 - t - \gamma)$} & \makecell{$c'(e) = \frac{\theta(1-t)}{\gamma}$\\\\$pk'(e\theta - y') =$\\$ -(1 - t - \gamma)$}\\
		\hline
	\end{tabular} 
\end{center}
\section{Not enough undeclared money}
Consider the case $e\theta - y' < \beta = c(e) - (1-t)\beta'$. Since it is less costly to pay the worker out of undeclared money, it is worth paying as much as possible out of undeclared money and then pay the remaining amount out of declared profit. That means that 
\begin{align*}
e\theta - y' = (1 - \alpha) \beta\\
\alpha = \left(1 - \frac{e \theta - y'}{\beta}\right)\cdot\mathbb{I}_{\{e \theta - y' < \beta\}}
\end{align*}
where
\begin{align*}
\mathbb{I}_{\{e \theta - y' < \beta\}} = \begin{cases}
1, e\theta - y' < \beta\\
0, \text{ otherwise}
\end{cases}
\end{align*}
The firm's profit becomes:
\begin{align*}
\pi = (y' - \beta')(1 - t) - \beta\mathbb{I}_{\{e \theta - y' < \beta\}} + (e \theta - y')\mathbb{I}_{\{e \theta - y' < \beta\}} + \gamma(e\theta - y' - \beta + \beta \mathbb{I}_{\{e \theta -y' < \beta\}} -\\ -\mathbb{I}_{\{e \theta - y' < \beta\}}(e\theta - y')) - pk(e \theta - y')
\end{align*}
Using (IC) $\beta = c(e) - \beta'(1-t)$ the profit function is
\begin{align*}
\pi = (y'-\beta')(1-t) +(\mathbb{I}_{\{e \theta - y' < \beta\}} + \gamma(1-\mathbb{I}_{\{e \theta - y' < \beta\}}))(e \theta -y' - c(e) + (1-t)\beta')-pk(e \theta - y')
\end{align*}
Denote $\mathbb{I}_{\{e \theta - y' < \beta\}} + \gamma(1-\mathbb{I}_{\{e \theta - y' < \beta\}})$ as $\xi$. Note that
\begin{align*}
\xi = \begin{cases}
1, e\theta - y' < \beta\\
\gamma, \text{ otherwise}
\end{cases}
\end{align*}
\subsection{No credibility problems}
Without enforceability problems, FOCs are
\begin{align*}
\frac{\partial \pi }{\partial e} &= -\xi c'(e) + \xi \theta - \theta p k'(e\theta - y') = 0\\
\frac{\partial \pi}{\partial y'} &= 1-t - \xi + pk'(e \theta - y') = 0\\
c'(e) &= \theta(1-t)\\
pk'(e \theta -y') &= -(1 - t - \xi)
\end{align*}
\subsection{Enforceability problems}
The (DE) constraint is
\begin{align*}
\frac{\delta \pi}{1 - \delta} \ge (1 - t)\beta' + \alpha \beta + \gamma(1 - \alpha) \beta
\end{align*}
\begin{align*}
\delta (y'-\beta')(1-t) +\delta \xi(e \theta -y' - c(e) + (1-t)\beta')-\delta pk(e \theta - y') \ge (1 - \delta)(1 - t)\beta' +\\
+ (1-\delta)(\alpha + \gamma(1 - \alpha))\beta\\
\delta y'(1-t) +\delta \xi(e \theta -y')-\delta pk(e \theta - y') \ge ((1-\delta)(\alpha + \gamma(1 - \alpha)) +\delta \xi)c(e) +\\ +(1 - \delta \xi + (1 - \delta)(\alpha + \gamma(1 - \alpha)))(1-t)\beta'
\end{align*}
The problem becomes
\begin{align*}
\max\ &(y'-\beta')(1-t) +\xi(e \theta -y' - c(e) + (1-t)\beta')-pk(e \theta - y')\\
s.t.\ &\delta y'(1-t) +\delta \xi(e \theta -y')-\delta pk(e \theta - y') \ge ((1-\delta)(\alpha + \gamma(1 - \alpha)) +\delta \xi)c(e) +\\&+(1 - \delta \xi + (1 - \delta)(\alpha + \gamma(1 - \alpha)))(1-t)\beta'\to \lambda \ (DE)\\
& e \theta - y' < c(e) - (1-t)\beta'\ \to \mu \\
& e\theta - y' \ge 0\ \to \eta\\
& \beta' \ge 0, y' \ge 0, e \ge 0\ \to \psi_1, \psi_2,\psi_3
\end{align*}
\begin{align}
\frac{\partial \pi}{\partial y'} &= 1-t-\xi + pk'(e \theta - y') + \lambda \delta (1 - t - \xi) + \lambda \delta pk'(e \theta - y') + \mu - \eta + \psi_2 = 0 \label{eq8}\\
\frac{\partial \pi}{\partial e} &= -\xi c'(e)+\delta \xi \theta - \theta pk'(e\theta - y')- \lambda (\delta\xi + (1-\delta)(\alpha + \gamma(1 - \alpha))) c'(e) + \lambda \delta \theta - \nonumber\\ 
&-\lambda \delta \theta pk'(e \theta - y') - \mu \theta + \mu c'(e) + \eta \theta+ \psi_3 = 0\label{eq9}\\
\frac{\partial \pi}{\partial \beta'} &= -(1-\xi)(1-t) - \lambda(1 - \delta \xi + (1 - \delta)(\alpha + \gamma(1 - \alpha)))(1-t) - \mu(1-t) + \psi_1 = 0\label{eq10}\\
& \lambda, \mu, \eta, \psi_1, \psi_2, \psi_3 \ge 0 \label{eq11}\\
&+\text{ slackness complementary conditions}\nonumber
\end{align}
Assume firstly that $\beta', y', e >0$, also suppose that $e\theta - y' < \beta \to \xi = 1$, that means $\psi_1 = \psi_2=\psi_3 = 0$, consequently from \eqref{eq10} and \eqref{eq11}, $\lambda = \mu = 0$. If $e \theta - y' > 0$ then $\eta = 0$ and the solution is equivalent to the case without credibility problems. Thus, the optimal level of effort and the amount of declared revenue are determined by
\begin{align*}
c'(e) &= \theta(1 - t)\\
pk'(e\theta - y') &= t
\end{align*}


Now assume that $\psi_1 > 0\ \to \beta' = 0$, implying $\mu = 0$, $\lambda = \frac{\psi_1}{(1 - \delta \xi + (1 - \delta)(\alpha + \gamma(1 - \alpha)))(1-t)}$$> 0$ hence the (DE) constraint binds. Then from the FOCs
\begin{align*}
c'(e) &= \frac{(1 + \lambda \delta)\theta(1-t)}{\xi+\lambda(\delta\xi + (1 - \delta)(\alpha + \gamma(1 - \alpha)))}\\
pk'(e\theta - y') &= -(1 - t - \xi)
\end{align*}
and $\lambda$ can be found from the (DE) constraint.
\end{document}