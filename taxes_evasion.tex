\documentclass[a4paper]{article}
\usepackage[14pt]{extsizes} % 
\usepackage[utf8]{inputenc}
\usepackage{setspace,amsmath}
\usepackage{mathtools}
\usepackage{pgfplots}
\usepackage{titlesec}
\usepackage{pdfpages}
\usepackage{makecell}
\usepackage[shortlabels]{enumitem}
\usepackage{tikz}
\usepackage{multirow}
\usetikzlibrary{angles,quotes}
\usepackage{graphicx}
\usepackage{xcolor,colortbl}
\usepackage{amssymb}
\usepackage{float}
\usepackage[section]{placeins}
\usepackage[makeroom]{cancel}
\usepackage{mathrsfs} % 
\newcommand\numberthis{\addtocounter{equation}{1}\tag{\theequation}}
%\addto\captionsrussian{\renewcommand{\figurename}{Fig.}}
\usepackage{amsmath,amsfonts,amssymb,amsthm,mathtools} 
\newcommand*{\hm}[1]{#1\nobreak\discretionary{}
{\hbox{$\mathsurround=0pt #1$}}{}}
\usepackage{graphicx}  % 
\graphicspath{{images/}{images2/}}  % 
\setlength\fboxsep{3pt} %  \fbox{} 
\setlength\fboxrule{1pt} % \fbox{}
\usepackage{wrapfig} % 
\newcommand{\prob}{\mathbb{P}}
\newcommand{\norma}{\mathscr{N}}
\newcommand{\expect}{\mathbb{E}}
\newcommand{\summa}{\sum_{i=1}^n}
\usepackage[left=7mm, top=20mm, right=15mm, bottom=20mm, nohead, footskip=10mm]{geometry} % 
\usepackage{tikz} % 
\def\myrad{2cm}% radius of the circle
\def\myanga{45}% angle for the arc
\def\myangb{195}
\begin{document} % 
\section{Enough undeclared money to pay the worker under the table}
This is the case once $e\theta - y' \ge \beta$.
\subsection{Enforceability problem}
\begin{align*}
&\underset{\beta', e, y'}{\max}\ \pi = y'(1 - t) - \beta'(1-t)(1 - \gamma) + \gamma[e\theta - y' - c(e)] - pk(e\theta - y')\\
s.t.\ &\frac{\delta \pi}{1 - \delta} \ge (1-t)\beta' + \gamma \beta\ (DE) \\
&e\theta - y' \ge \beta
\end{align*}
Using the (IC), (DE) becomes
\begin{align*}
\delta y'(1 - t - \gamma) + \delta \gamma e \theta - \delta p k(e \theta - y') \ge \gamma c(e) + \beta'(1-t)(1 - \gamma)
\end{align*}
Then the problem is
\begin{align*}
\underset{e, y', \beta'}{\max}\ &y'(1 - t) - \beta'(1-t)(1 - \gamma) + \gamma[e\theta - y' - c(e)] - pk(e\theta - y')\\
s.t.\ &\delta y'(1 - t - \gamma) + \delta \gamma e \theta - \delta p k(e \theta - y') \ge \gamma c(e) + \beta'(1-t)(1 - \gamma)\ \to \lambda \\
& e\theta - y' \ge c(e) - (1 - t)\beta'\ \to \mu \\
& \beta', y', e \ge 0\ \to \psi_1, \psi_2, \psi_3\ \\
\end{align*}

Then FOCs are 
\begin{align}
\frac{\partial \pi}{\partial e} &= \gamma \theta - \gamma c'(e)- \theta p k'(e \theta - y') + \lambda \delta \gamma \theta - \lambda \delta \theta pk'(e \theta - y') - \lambda \gamma c'(e) + \mu \theta - \mu c'(e) + \psi_3= 0\label{eq1}\\
\frac{\partial \pi}{\partial y'} &= (1 - t - \gamma)+pk'(e\theta - y') + \lambda \delta (1 - t - \gamma) +\lambda \delta pk'(e\theta - y') - \mu + \psi_2 = 0\label{eq2}\\
\frac{\partial \pi}{\partial \beta'} &= -(1-t)(1-\gamma) - \lambda (1-t)(1-\gamma) + \mu(1 - t) + \psi_1= 0\label{eq3}\\
&\lambda, \mu, \psi_1, \psi_2, \psi_3 \ge 0\nonumber\\
&+\text{ slackness complementary conditions}\nonumber
\end{align}


It is clear that if $e \theta - y' \ge c(e)$ then $\beta' = 0$. Indeed, assume the contrary, $\beta' > 0$ then the principal can slightly decrease $\beta'$. This cannot violate the (DE) constraint. Additionally, if $e\theta - y' \ge c(e)$ then it is for sure $\ge c(e) - (1 - t)\beta'$ for $\beta'>0$, thus, this constraint cannot be violated either. But the profit increases. Hence, $\beta'>0$ is not optimal. Intuitively: if there is enough undeclared money to fully compensate the worker's cost, it is optimal to pay this compensation under the table, since this income is tax free for the worker. Though, it could be not optimal for the firm to make the difference $e\theta - y'$ too large (e.g. more than $c(e)$) because the punishment for tax evasion depends positively upon it, but if it is optimal then the declared bonus is zero.


Let us consider this case. Moreover, suppose that $e \theta - y' > c(e)\ \to \mu = 0$, and that the (DE) constraint does not bind. This case is equivalent to the case without the enforceability problem, thus, the optimal level of effort and the amount of declared revenue are determined by
\begin{align*}
c'(e) &= \frac{(1 - t)\theta}{\gamma}\\
pk'(e\theta - y') &= - (1 - t - \gamma)
\end{align*}
Assuming that the (DE) binds, FOCs become
\begin{align*}
\frac{\partial \pi}{\partial e} &= (1 + \lambda \delta)(\gamma \theta - \theta p k'(e \theta - y')) -(1+ \lambda) \gamma c'(e)= 0\\
\frac{\partial \pi}{\partial y'} &= (1 + \lambda \delta)(1 - t - \gamma+pk'(e\theta - y')) = 0\\
&\text{if }1-t < \gamma \text{ then }c'(e) = \frac{(1+\lambda \delta)\theta(1-t)}{\gamma (1 + \lambda)} \underset{\delta < 1}{<} \frac{\theta(1-t)}{\gamma}\\
&pk'(e \theta - y') = -(1 - t - \gamma)
\end{align*}
(Note that $\lambda$ can be found from the DE constraint which is binding). 


Assume that $\mu > 0$. That means that $e\theta - y' = c(e)$. If the (DE) constraint is not binding then FOCs become
\begin{align*}
\frac{\partial \pi}{\partial e} &= \gamma \theta- \gamma c'(e) - \theta p k'(e \theta - y') + \mu \theta - \mu c'(e) = 0\\
\frac{\partial \pi}{\partial y'} &= (1 - t - \gamma)+pk'(e\theta - y') - \mu  = 0\\
c'(e) &= \frac{\theta(1 - t)}{\mu + \gamma}\\
pk'(e\theta - y') &= \mu - (1 - t - \gamma)
\end{align*}
if the (DE) constraint is binding then 
\begin{align*}
\frac{\partial \pi}{\partial e} &= (1 + \lambda \delta)(\gamma \theta - \theta p k'(e \theta - y')) - (1+\lambda) \gamma c'(e)+ \mu \theta - \mu c'(e) = 0\\
\frac{\partial \pi}{\partial y'} &= (1+\lambda \delta)(1 - t - \gamma+pk'(e\theta - y')) - \mu = 0\\
c'(e) &= \frac{\theta(1-t)(1+\lambda \delta)}{(1 + \lambda)\gamma + \mu}\\
pk'(e\theta - y') &= \frac{\mu}{1+\lambda \delta} - (1 - t- \gamma)
\end{align*}


If $c(e) - (1-t)\beta' \le e \theta - y' < c(e)$ then $\beta'$ can (and should) be positive.
Assuming in addition $y' > 0, e > 0$ then $\psi_1 = 0, \psi_2 = 0, \psi_3= 0$ and hence from \eqref{eq3}:
\begin{align*}
(1 + \lambda)(1 - \gamma) = \mu > 0\ \to e \theta - y' = c(e) - (1 - t)\beta'
\end{align*}
Then if the (DE) constraint does not bind, $\lambda = 0, \mu = 1 - \gamma$ and FOCs imply:
\begin{align*}
\frac{\partial \pi}{\partial e} &= \theta - c'(e) - \theta p k'(e\theta - y') = 0\\
\frac{\partial \pi}{\partial y'} &= (1 - t - \gamma)+pk'(e\theta - y') - (1-\gamma) = 0
\end{align*}
\begin{align*}
c'(e) &= \theta(1 - t)\\
pk'(e\theta - y') &= t\\
\beta' &= \frac{c(e) - (e\theta - y')}{1-t}
\end{align*}
The price of money laundering $\gamma$ affects neither the optimal level of effort nor the amount of declared revenue because the whole undeclared revenue is paid as a black bonus to the worker.


Assume that (DE) does bind. That means that $\lambda > 0$.
\begin{align*}
\frac{\partial \pi}{\partial e} &= (1 + \lambda \delta)(\gamma \theta - \theta p k'(e \theta - y')) - (1+\lambda)c'(e) + (1+\lambda)(1-\gamma)\theta = 0\\
\frac{\partial \pi}{\partial y'} &= (1+\lambda \delta)(1 - t - \gamma+pk'(e\theta - y')) - (1+\lambda)(1-\gamma) = 0\\
c'(e) &= \frac{(1+\lambda \delta)\theta(\gamma - pk'(e \theta - y'))+ (1+\lambda)(1-\gamma)\theta}{1+\lambda} = \frac{\theta(1-t)(1 + \lambda \delta)}{1+\lambda}\underset{\delta < 1}{<} \theta(1-t)\\
pk'(e\theta - y') &= \frac{(1+\lambda)(1-\gamma)}{1+\lambda \delta} - (1 - t - \gamma)\\
\beta' &= \frac{\delta y'(1 - t - \gamma) + \delta \gamma e \theta - \delta p k(e \theta - y') - \gamma c(e)}{(1-t)(1-\gamma)} = \frac{c(e) - (e \theta - y')}{1-t}
\end{align*}
The optimal level of effort still does not depend on the cost of money laundering. While the amount of declared profit $y'$ does depend. That is because the (DE) constraint is binding and $\gamma$ affects this constraint.

The results of this section are summarized by the table below:
\begin{center}
	\begin{tabular}{ |c|c|c|c|c|c| } 
		\hline
		 & \makecell{$\mu = 0$\\$\lambda>0$}&\makecell{$\mu = 0$\\$\lambda=0$}&\makecell{$\mu > 0$\\$\lambda>0$}&\makecell{$\mu > 0$\\$\lambda=0$} \\
		\hline
		$\beta' = 0$ & \makecell{$c'(e) = \frac{(1+\lambda\delta)\theta(1-t)}{\gamma(1 + \lambda)}$\\\\$pk'(e\theta - y') =$\\$ -(1 - t - \gamma)$} & \makecell{$c'(e) = \frac{\theta(1-t)}{\gamma}$\\\\$pk'(e\theta - y') =$\\$ -(1 - t - \gamma)$}&\makecell{$c'(e) = \frac{\theta(1-t)(1+\lambda\delta)}{(1+\lambda)\gamma + \mu}$\\\\$pk'(e\theta - y') =$\\$ \frac{\mu}{1+\lambda\delta}-(1 - t - \gamma)$}&\makecell{$c'(e) = \frac{\theta(1-t)}{\mu +\gamma}$\\\\$pk'(e\theta - y') =$\\$\mu -(1 - t - \gamma)$} \\ 
		\hline
		$\beta' >0$ & \cellcolor[HTML]{B2BEB5}& \cellcolor[HTML]{B2BEB5}&\makecell{$c'(e) = \frac{\theta(1-t)(1+\lambda\delta)}{1+\lambda}$\\\\$pk'(e\theta - y') =$\\$\frac{(1+\lambda)(1-\gamma)}{1+\lambda \delta} -(1 - t - \gamma)$} & \makecell{$c'(e) = \theta(1-t)$\\\\$pk'(e\theta - y') =t$} \\ 
		\hline
	\end{tabular}
\end{center}
\section{Not enough undeclared money}
Consider the case $e\theta - y' < \beta$. Since it is less costly to pay the worker out of undeclared money, it is worth paying as much as possible out of undeclared money and then pay the remaining amount out of declared profit. That means that 
\begin{align*}
e\theta - y' = (1 - \alpha) \beta\\
\alpha = 1 - \frac{e \theta - y'}{\beta}
\end{align*}
The firm's profit becomes:
\begin{align*}
\pi = (y' - \beta')(1 - t) - \beta + e \theta - y' - pk(e \theta - y')
\end{align*}
Using (IC) $\beta = c(e) - \beta'(1-t)$ the profit function is
\begin{align*}
\pi = (y'-\beta')(1-t) - c(e) + \beta' (1-t) +e \theta -y' -pk(e \theta - y') = \\ =-y't - c(e) + e \theta -pk'(e \theta - y')
\end{align*}
\subsection{No credibility problems}
Without enforceability problems, FOCs are
\begin{align*}
\frac{\partial \pi }{\partial e} &= -c'(e) + \theta - \theta p k'(e\theta - y') = 0\\
\frac{\partial \pi}{\partial y'} &= -t + pk'(e \theta - y') = 0\\
c'(e) &= \theta(1-t)\\
pk'(e \theta -y') &= t
\end{align*}
\subsection{Enforceability problems}
The (DE) constraint is
\begin{align*}
\frac{\delta \pi}{1 - \delta} &\ge (1 - t)\beta' + \gamma \beta\\
-\delta y't - \delta c(e) + \delta e \theta -\delta pk'(e \theta - y') &\ge (1 - \delta)(1 - t)\beta' + (1 - \delta)\gamma c(e) - \gamma (1 - \delta )(1 - t)\beta'
\end{align*}
The problem becomes
\begin{align*}
\max\ &-y't - c(e) + e \theta -pk'(e \theta - y')\\
s.t.\ &-\delta y't + \delta e \theta -\delta pk'(e \theta - y') \ge (1 - \gamma)(1 - \delta)(1 - t)\beta' + \gamma c(e)\to \lambda \ (DE)\\
& e \theta - y' < c(e) - (1-t)\beta'\ \to \mu \\
& e\theta - y' \ge 0\ \to \eta\\
& \beta' \ge 0, y' \ge 0, e \ge 0\ \to \psi_1, \psi_2,\psi_3
\end{align*}
\begin{align}
\frac{\partial \pi}{\partial y'} &= -t + pk'(e \theta - y') - \lambda \delta t + \lambda \delta pk'(e \theta - y') + \mu - \eta + \psi_2 = 0 \label{eq4}\\
\frac{\partial \pi}{\partial e} &= -c'(e)+\theta - \theta pk'(e\theta - y')- \lambda \gamma c'(e) + \lambda \delta \theta - \lambda \delta \theta pk'(e \theta - y') - \mu \theta + \nonumber\\
&+\mu c'(e) + \eta \theta+ \psi_3 = 0\label{eq5}\\
\frac{\partial \pi}{\partial \beta'} &= - \lambda(1-\gamma)(1-\delta)(1-t) - \mu(1-t) + \psi_1 = 0\label{eq6}\\
& \lambda, \mu, \eta, \psi_1, \psi_2, \psi_3 \ge 0 \label{eq7}\\
&+\text{ slackness complementary conditions}\nonumber
\end{align}
Assume firstly that $\beta', y', e >0$, that means $\psi_1 = \psi_2=\psi_3 = 0$, consequently from \eqref{eq6} and \eqref{eq7}, $\lambda = \mu = 0$. If $e \theta - y' > 0$ then $\eta = 0$ and the solution is equivalent to the case without credibility problems. 


Now assume that $\psi_1 > 0\ \to \beta' = 0$, thus, implying $\mu = 0$, $\lambda = \frac{\psi_1}{(1-t)(1-\delta)(1-\gamma)} > 0$ hence the (DE) constraint binds. Then from the FOCs
\begin{align*}
c'(e) = \frac{(1 + \lambda \delta)\theta(1-t)}{1+\lambda} \underset{\delta < 1}{<}\theta(1-t)\\
pk'(e\theta - y') = t
\end{align*}
and $\lambda$ can be found from the (DE) constraint.
\end{document}