\documentclass[a4paper]{article}
\usepackage[14pt]{extsizes} % 
\usepackage[utf8]{inputenc}
\usepackage{setspace,amsmath}
\usepackage{mathtools}
\usepackage{pgfplots}
\usepackage{titlesec}
\usepackage{pdfpages}
\usepackage[shortlabels]{enumitem}
\usepackage{tikz}
\usetikzlibrary{angles,quotes}
\usepackage{graphicx}
\usepackage{amssymb}
\usepackage{float}
\usepackage{makecell}
\usepackage[section]{placeins}
\usepackage[makeroom]{cancel}
\usepackage{mathrsfs} % 
\newcommand\numberthis{\addtocounter{equation}{1}\tag{\theequation}}
%\addto\captionsrussian{\renewcommand{\figurename}{Fig.}}
\usepackage{amsmath,amsfonts,amssymb,amsthm,mathtools} 
\newcommand*{\hm}[1]{#1\nobreak\discretionary{}
{\hbox{$\mathsurround=0pt #1$}}{}}
\usepackage{graphicx}  % 
\graphicspath{{images/}{images2/}}  % 
\setlength\fboxsep{3pt} %  \fbox{} 
\setlength\fboxrule{1pt} % \fbox{}
\usepackage{wrapfig} % 
\newcommand{\prob}{\mathbb{P}}
\newcommand{\norma}{\mathscr{N}}
\newcommand{\expect}{\mathbb{E}}
\newcommand{\summa}{\sum_{i=1}^n}
\newcommand{\yrseduc}{\textit{yrseduc}}
\usepackage[left=7mm, top=20mm, right=15mm, bottom=20mm, nohead, footskip=10mm]{geometry} % 
\usepackage{tikz} % 
\def\myrad{2cm}% radius of the circle
\def\myanga{45}% angle for the arc
\def\myangb{195}
\begin{document} % 
	\begin{flushright}
	\begin{tabular}{r}
		Danil Fedchenko, MAE 2020, group A \\
	\end{tabular}
\end{flushright}


\begin{center}
	Game theory. Problem Set 3.
\end{center}
\section*{Problem 1}
The tree of the game is depicted below:
\begin{figure}[H]
	\centering
	\includegraphics[width=0.8\textwidth]{plotdraft}
	\caption{}\label{fig1}
\end{figure}
Obviously, profile of strategies
\begin{align*}
\left\{L, R\right\}
\end{align*}
is a Nash equilibrium: the best response on $R$ is $L$, the best response on $L$ is $R$. Moreover, if we introduce belief $\mu$ as
\begin{align*}
\prob (\text{2nd player at node} \left\{L\right\}|\text{2nd player is called upon to play}) = \mu
\end{align*}
then 
\begin{align*}
-2 \mu < 5\mu + 7 - 7 \mu = 7 - 2\mu\ \forall\ \mu \in (0, 1)
\end{align*}
that means that optimal strategy for player 2 is playing $R$. And aware of it, optimal strategy for player 1 is playing L. As a result the outcome $\left\{LR\right\}$ is both weak sequential equilibrium and Nash equilibrium.
\section*{Problem 2}
Obviously the weak sequential equilibrium (which is also a SPE) is the following profile of strategies:
\begin{align*}
\left\{In\ Acquiesce;\ A\right\}
\end{align*}
and belief for player 2:
\begin{align*}
(0, 1)
\end{align*}
i.e. player 2 knows for sure that he is at node $\left\{In\ Acquiesce\right\}$. This conclusion is derived because in unique proper subgame the strategy $\left\{Fight\right\}$ is strictly dominated by strategy $\left\{Acquisce\right\}$.

Suppose player 1 plays $\left\{Out, Fight\right\}$ for sure. Then the information set of the 2nd player is reached with 0 probability. Hence the concept of weak sequential equilibrium allows us to put arbitrary beliefs but to the extent that given those beliefs player 1 still does not have a profitable deviation. Assume $\mu_{IF} = \mu$ then expected payoffs of player 2 are:
\begin{align*}
F : u_2 &= -1\\
A: u_2 &=-2\mu + 1 - \mu = 1 - 3\mu
\end{align*}
If $\mu < \frac{2}{3}$ then player 2 plays $A$, if $\mu > \frac{2}{3}$ then he plays $F$. If $\mu = \frac{2}{3}$ then player 2 mixes. On the other hand, suppose that player 1 deviates and plays $\left\{In, Fight\right\}$ then his expected payoffs is equal to:
\begin{align*}
\begin{cases}
3, \mu < \frac{2}{3}\\
-2, \mu > \frac{2}{3}\\
3 - 5\alpha, \mu = \frac{2}{3}
\end{cases}
\end{align*}
where $\alpha$ is a probability which player 2 mixes the strategy $F$ with. If $\mu < \frac{2}{3}$ then it is profitable for player 1 to deviate, hence it is not an equilibrium, if $\mu > \frac{2}{3}$ then player 1 does not have a profitable deviation. If $\mu = \frac{2}{3}$ then if $\alpha < \frac{3}{5}$ then again, player 1 has a profitable deviation, if $\alpha > \frac{3}{5}$ then the first player does not have incentives to deviate. If $\alpha = \frac{3}{5}$ then player is indifferent, however mixing for player 1 cannot be consistent with beliefs because once he assigns positive probability for $\left\{In\right\}$, $\mu$ automatically becomes zero, because strategy $\left\{In, Fight\right\}$ is strictly dominated by $\left\{In, Acquisce\right\}$. As a result the profiles of strategies and beliefs for player 2:
\begin{align*}
\begin{cases}
\left\{Out, Fight;\ F \right\}, \prob(In, Fight|In) = \mu > \frac{2}{3},\ \prob(In, Acquisce|In) = 1 - \mu\\
\left\{Out, Fight;\ \alpha F + (1-\alpha)A \right\}, \prob(In, Fight|In) = \frac{2}{3},\ \prob(In, Acquisce|In) = \frac{1}{3}, \alpha > \frac{3}{5}\\
\end{cases}
\end{align*}
constitute sequential equilibrium. These profiles are not SPE.
\section*{Problem 3}
\begin{figure}[H]
	\centering
	\includegraphics[width=0.6\textwidth]{game3}
	\caption{}\label{fig2}
\end{figure}


\textbf{Solution}

Suppose that, conditional on the fact that player 2 is called upon to play, he believes that he is at node $\{AR\}$ with probability $\mu$ and at node $\left\{BR\right\}$ with probability $1 - \mu$. Then playing $W$ yields him $5$ whereas playing $E$ yields 
\begin{align*}
10 - 8 \mu
\end{align*}
hence he should play
\begin{align*}
\begin{cases}
W, \mu > \frac{5}{8}\\
E, \mu < \frac{5}{8}\\
\alpha W + (1 - \alpha)E, \mu = \frac{5}{8}
\end{cases}
\end{align*}
for $\alpha \in (0, 1)$. For player 1, if he plays $R$ he gets:
\begin{align*}
\begin{cases}
0, \mu > \frac{5}{8}\\
5, \mu < \frac{5}{8}\\
5 - 5\alpha, \mu = \frac{5}{8}
\end{cases}
\end{align*}
So, we can conclude that the profile of strategies and belief for player 2:
\begin{align*}
\left\{L, W\right\}, (\mu, 1-\mu), \mu > \frac{5}{8}
\end{align*}
is a weak sequential equilibrium.

If $\mu < \frac{5}{8}$ then player 1 plays for sure $R$ as a result $\mu \neq \frac{1}{2}$ is inconsistent. If $\mu = \frac{5}{8}$ then for $\alpha > \frac{3}{5}$ player 1 plays $L$ and this is also a weak sequential equilibrium, if $\alpha < \frac{3}{5}$ then player 1 should play $R$ for sure, but it is inconsistent with $\mu$. The only remained case is $\alpha = \frac{3}{5}$. In this case, player 1 is indifferent, hence he will mix, however $\mu \neq \frac{1}{2}$ will always be inconsistent with each $\beta$.

If $\mu = \frac{1}{2}$ then player 2 plays $E$ and player 1 plays $R$. This is also a weak sequential equilibrium, since belief is consistent with the strategy and both players maximize their expected payoffs given that the other player plays the strategy.
Thus, weak sequential equilibria of this game are:
\begin{align*}
\begin{cases}
\{L, W\},\ (\mu, 1-\mu),\ \mu > \frac{5}{8}\\
\{L, \alpha W + (1 - \alpha) L\},\ (\frac{5}{8}, \frac{3}{8}),\ \alpha > \frac{3}{5}\\
\{R, E\},\ (\frac{1}{2}, \frac{1}{2})
\end{cases}
\end{align*}
(Belief for player 1 is obviously $(0.5, 0.5)$)

\section*{Problem 4}
\begin{figure}[H]
	\centering
	\includegraphics[width=0.7\textwidth]{game4}
	\caption{}\label{fig3}
\end{figure}


\textbf{Solution}
\begin{enumerate}[(a)]
	\item Yes, for example $\alpha = \frac{3}{4}$ in this case the profile of strategies $\{D, RL\}$ is a Nash equilibrium. Let us prove it. Obviosly, $\{RL\}$ is a best response on $\{D\}$. For player 1 expected payoffs of playing:
	\begin{align*}
	&U\text{ is } \frac{3}{8}\\
	&D\text{ is } \frac{9}{16} > \frac{3}{8}
	\end{align*}
	i.e. $D$ is also a best response on $\{RL\}$, hence it is a Nash equilibrium.
	\item $\alpha = \frac{1}{2}$. One of sequential equilibria is a mixed behavioural strategy for player 1: $\frac{2}{3}U + \frac{1}{3}D$ and for player 2: $\{\frac{2}{3}L + \frac{1}{3}R, R\}$ and believs for player 2: $(\frac{2}{3}, \frac{1}{3})$ (equal for both information sets, where he is called upon to play). That is, player 2 plays $\frac{2}{3}L + \frac{1}{3}R$ once the nature plays $T$ and $R$ otherwise. Suppose "1st information set" stands for the information set where the 2nd player is called upon to play and nature plays T, and "2nd information set" once the nature plays $B$. Since $\frac{1}{6} < \frac{1}{3}$ then at the 2nd information set the best response of player 2 is indeed $R$ and moreover, belief $(\frac{2}{3}, \frac{1}{3})$ is consistent with the strategy:
	\begin{align*}
	\frac{2}{3} = \frac{\frac{2}{3}(1-\alpha)}{(1 - \alpha)}
	\end{align*}. At the 1st information set, player's 2 expected payoffs from playing:
	\begin{align*}
	L: \frac{1}{3}\\
	R: \frac{1}{3}
	\end{align*}
	hence the best response is any mixed behavioural strategy and $\frac{2}{3}L + \frac{1}{3}R$ in particular. Also note, that at 1st information set belief for player 2 is consistent. It remains to check whether the first player is sequentially rational for this strategy. His expected payoffs of playing:
	\begin{align*}
	U: \frac{1}{3}\\
	D: \frac{1}{3}
	\end{align*}
	that means that he can play any mixed strategy and $\frac{2}{3}U + \frac{1}{3}D$ in particular.
	
	Thus, the strategy and believes are indeed sequential equilibrium of the game.
\end{enumerate}
\end{document}