\documentclass[a4paper]{article}
\usepackage[14pt]{extsizes} % 
\usepackage[utf8]{inputenc}
\usepackage{setspace,amsmath}
\usepackage{mathtools}
\usepackage{pgfplots}
\usepackage{titlesec}
\usepackage{pdfpages}
\usepackage[shortlabels]{enumitem}
\usepackage{tikz}
\usetikzlibrary{angles,quotes}
\usepackage{graphicx}
\usepackage{amssymb}
\usepackage{float}
\usepackage[section]{placeins}
\usepackage[makeroom]{cancel}
\usepackage{mathrsfs} % 
\newcommand\numberthis{\addtocounter{equation}{1}\tag{\theequation}}
%\addto\captionsrussian{\renewcommand{\figurename}{Fig.}}
\usepackage{amsmath,amsfonts,amssymb,amsthm,mathtools} 
\newcommand*{\hm}[1]{#1\nobreak\discretionary{}
{\hbox{$\mathsurround=0pt #1$}}{}}
\usepackage{graphicx}  % 
\graphicspath{{images/}{images2/}}  % 
\setlength\fboxsep{3pt} %  \fbox{} 
\setlength\fboxrule{1pt} % \fbox{}
\usepackage{wrapfig} % 
\newcommand{\prob}{\mathbb{P}}
\newcommand{\norma}{\mathscr{N}}
\newcommand{\expect}{\mathbb{E}}
\newcommand{\ubar}{\overline}
\newcommand{\lbar}{\underline}
\newcommand{\summa}{\sum_{i=1}^n}
\usepackage[left=7mm, top=20mm, right=15mm, bottom=20mm, nohead, footskip=10mm]{geometry} % 
\usepackage{tikz} % 
\def\myrad{2cm}% radius of the circle
\def\myanga{45}% angle for the arc
\def\myangb{195}
\begin{document} % 
	\begin{flushright}
	\begin{tabular}{r}
		Danil Fedchenko, MAE 2020, group A \\
	\end{tabular}
\end{flushright}


\begin{center}
	Microeconomics 4. Problem Set 3.
\end{center}
\section*{1 Adverse selection with indivisible good}
	Consider the model we studied in class but where the agent may produce only zero or
	one unit of the good. Principal's valuation of one unit is $S$. Agent's costs of producing
	one unit are $\theta$ that takes two values: $\underline{\theta}$ with probability $\nu$ and $\overline{\theta}$ with probability $1-\nu$,
	$\Delta \theta = \overline{\theta} - \underline{\theta} > 0$. When an agent with costs $\theta$ produces one unit, the principal gets $S-t$ and the agent gets $t - \theta$, $t$ being the transfer from the principal to the agent. No production entails no cost to the agent and no value to the principal. Production is always efficient:
	$S > \overline{\theta}$. The agent's reservation utility is zero.
	In the beginning of the period, the principal offers a menu $\{(\underline{p}, \underline{t}), (\overline{p}, \overline{t})\}$, where $\underline{p}$ and $\overline{p}$ are the probabilities of trade with the low and high cost type, respectively (note
	that the transfer is made independently of whether the trade occurs or not).
\begin{enumerate}
	\item Find the first best.
	\item Find the second best.
	\item Can the principal replicate the second-best solution by a simpler contract? Which
one?
\end{enumerate}


\textbf{Solution}

\begin{enumerate}
	\item Assume that once the trade does not occur, the agent does not produce anything. That means that the agent's objective function is $U^A(t, p, \theta) = t - p \theta$. The principal's objective function is $U^P(\underline{p}, \underline{\theta}, \overline{p}, \overline{\theta}) = \nu(\underline{p}S - \underline{t}) + (1 - \nu)(\overline{p}S - \overline{t})$. Obviously under complete information $\underline{t} = \underline{p}\underline{\theta}$ and $\overline{t} = \overline{p}\overline{\theta}$. And $\underline{p} = \overline{p} = 1$. That is, the first best is the following:
	\begin{align*}
	\{(1, \overline{\theta}), (1, \underline{\theta})\}
	\end{align*}
	\item Under incomplete information, since $\overline{\theta} > \underline{\theta}$ then $\overline{t} - \overline{p}\underline{\theta} = \overline{p} \Delta \theta > \underline{t} - \underline{p}\underline{\theta} = 0$. That is, the type $\underline{\theta}$ will pretend to be the type $\overline{\theta}$. Hence the principal should propose the contracts which would be incentive compatible. Thus, the principal is solving the following optimization problem:
	\begin{align*}
	\underset{(\underline{p}, \underline{t}), (\overline{t}, \overline{p})}{\max}\ \nu(\underline{p}S - \underline{t}) + (1 - \nu)(\overline{p}S - \overline{t})\\
	s.t.\ \begin{cases}
	\overline{t} - \overline{p} \overline{\theta} \ge 0\ (PC1)\\
	\underline{t} - \underline{p} \underline{\theta} \ge 0\ (PC2)\\
	\overline{t} - \overline{p} \overline{\theta} \ge \underline{t} - \underline{p} \overline{\theta}\ (IC1)\\
	\underline{t} - \underline{p} \underline{\theta} \ge \overline{t} - \overline{p} \underline{\theta}\ (IC2)
	\end{cases}
	\end{align*}
	$\underline{t} - \underline{p}\underline{\theta} \ge \overline{t} - \overline{p}\underline{\theta} > \overline{t} - \overline{p}\overline{\theta} \ge 0$, hence, $PC2$ is not binding. While $PC1$ is binding because otherwise the principal can decrease $\underline{t}, \overline{t}$ by the same value, which leads to increase of objective function without violating of any constraints. Moreover, $IC2$ should be binding either, because otherwise the principal can decrease $\underline{t}$. Finally, the principal's problem should look like:
	\begin{align*}
	&\underset{\overline{p}, \underline{p}}{\max}\ \nu(\underline{p}S - \overline{p}\Delta \theta - \underline{p}\underline{\theta}) + (1-\nu) \overline{p}(S - \overline{\theta})\\
	&s.t.\ \overline{p} \le \underline{p}\ (IC1)
	\end{align*}
	Since $\nu (S - \underline{\theta}) > 0$ then $\underline{p} = 1$. If $(1 - \nu)(S - \overline{\theta}) > \nu \Delta \theta$ then $\overline{p} = 1$ and $\overline{t} = \overline{\theta}, \underline{t} = \overline{\theta}$. Effectively, it means that the principal proposes just one contract $(1, \overline{\theta})$. If $(1 - \nu)(S - \overline{\theta}) < \nu \Delta \theta$ then $\overline{p} = 0$, and $\overline{t} = 0, \underline{t} = \underline{\theta}$. If $(1 - \nu)(S - \overline{\theta}) = \nu \Delta \theta$ then $\overline{p} \in [0, 1]$, $\overline{t} = \overline{p} \overline{\theta}, \underline{t} = \overline{p}\Delta \theta + \underline{\theta}$. So, the second best is the following:
	\begin{align*}
	\begin{cases}
	(1, \overline{\theta}), (1, \overline{\theta}),\ \frac{S - \overline{\theta}}{S - \underline{\theta}} > \nu\\
	(1, \underline{\theta}), (0, 0),\ \frac{S - \overline{\theta}}{S - \underline{\theta}} < \nu\\
	(1, \overline{p}\Delta \theta + \underline{\theta}), (\overline{p}, \overline{p}\overline{\theta}),\ \frac{S - \overline{\theta}}{S - \underline{\theta}} = \nu, \overline{p} \in (0, 1)
	\end{cases}
	\end{align*}
	\item If $\frac{S - \ubar{\theta}}{S - \lbar{\theta}} \neq \nu$ then yes, the seller can merely either offer the price $\lbar{\theta}$ or $\ubar{\theta}$, depending on the sign of $\frac{S - \ubar{\theta}}{S - \lbar{\theta}} - \nu$. Otherwise, also yes, but however seller can pick any probability of trading, finally getting the same expected payoff.
\end{enumerate}
\section*{2 Adverse selection with a privately informed buyer}
	Here we consider another standard adverse selection setting: a seller with known costs
	sells a good to a buyer who is privately informed about his valuation.
	The cost for the principal of $q$ units of good (or quality level $q$) is $c(q)$ and $c'(q) >
	0, c''(q) > 0, c(0) = 0$. The utility of the principal is $V = t-c(q)$. The agent's valuation
	for the good is $\theta u(q)$, where $u'(q) > 0, u''(q) < 0, u(0) = 0$ and $\theta = \underline{\theta}$ with probability
	$\nu$ and $\theta = \overline{\theta}$ with probability $1-\nu$. Denote $\Delta \theta = \overline{\theta}- \underline{\theta}> 0$. The utility of the agent is $U = \theta u(q) - t$.
	\begin{enumerate}
	\item Find the first best.
	\item Check the Spence-Mirlees condition.
	\item Find the second best.
	\item When it is optimal to "shut down" the low-valuation type?
	\end{enumerate}

\textbf{Solution}

\begin{enumerate}
	\item If the principle knows the agent's valuation then she should solve the following optimization problems:
	\begin{align*}
	\underset{(\overline{t}, \overline{q})}{\max}\ \overline{t} - c(\overline{q})\\
	s.t.\ \overline{\theta} u(\overline{q}) - \overline{t} \ge 0\\
	\underset{(\underline{t}, \underline{q})}{\max}\ \underline{t} - c(\underline{q})\\
	s.t.\ \underline{\theta} u(\underline{q}) - \underline{t} \ge 0\\
	\end{align*}
	The solutions can be derived from the following FOCs
	\begin{align*}
	\underline{\theta} u'(\underline{q}) - c'(\underline{q}) = 0 \to \underline{q}^* \to \underline{t}^* = \underline{\theta}u(\underline{q}^*)\\
	\overline{\theta} u'(\overline{q}) - c'(\overline{q}) = 0 \to \overline{q}^* \to \overline{t}^* = \overline{\theta}u(\overline{q}^*)
	\end{align*}
	\item The Spence-Mirrlees condition
	\begin{align*}
	\frac{\partial ^2}{\partial q \partial \theta} (\theta u(q) - t) \neq 0\\
	\frac{\partial ^2}{\partial q \partial \theta} \theta u(q) = u'(\theta) > 0
	\end{align*}
	hence the condition is satisfied.
	\item If at the first best $\underline{q}^* > 0$ then $\overline{\theta} u(\underline{q}^*) - \underline{\theta}u(\underline{q}^*) > 0$ i.e. under asymmetric information the $\ubar{\theta}$ type would mimic the $\lbar{\theta}$ type. Hence the second best should be incentive compatible. The principal should solve:
	\begin{align*}
	\underset{(\underline{t}, \underline{q}), (\overline{t}, \overline{q})}{\max}\ \nu(\underline{t} - c(\underline{q})) + (1 - \nu)(\overline{t} - c(\overline{q}))\\
	s.t.\ \begin{cases}
	\overline{\theta} u(\overline{q}) - \overline{t} \ge 0\ (PC1)\\
	\underline{\theta} u(\underline{q}) - \underline{t} \ge 0\ (PC2)\\
	\overline{\theta} u(\overline{q}) - \overline{t} \ge \overline{\theta} u(\underline{q}) - \underline{t}\ (IC1)\\
	\underline{\theta} u(\underline{q}) - \underline{t} \ge \underline{\theta} u(\overline{q}) - \overline{t}\ (IC2)
	\end{cases}
	\end{align*}
	$\ubar{\theta}u(\ubar{q}) - \ubar{t} \ge \ubar{\theta} u(\lbar{q}) - \lbar{t} > \lbar{\theta}u(\lbar{q}) - \lbar{t} > 0$ hence $PC1$ is not binding. While $PC2$ is binding because otherwise the principal can increase $\ubar{t}, \lbar{t}$ by the same amount which leads to increase of the objective function without the violation of any constraints. Moreover, the $IC1$ is binding because otherwise the principal can slightly increase $\ubar{t}$. Hence the optimization problem should like:
	\begin{align*}
	&\underset{(\ubar{q}, \lbar{q})}{\max}\ \nu(\lbar{\theta}u(\lbar{q}) - c(\lbar{q})) + (1 - \nu)(\ubar{\theta}u(\ubar{q}) - \Delta \theta u(\lbar{q}) - c(\ubar{q}))\\
	&s.t.\ u(\lbar{q}) \le u(\ubar{q})
	\end{align*}
	FOCs:
	\begin{align*}
	\begin{cases}
	\lbar{\theta}u'(\lbar{q}^*) - c'(\lbar{q}^*) - \frac{1 - \nu}{\nu} \Delta \theta u'(\lbar{q}^*) = 0\\
	\ubar{\theta}u'(\ubar{q}^*) - c'(\ubar{q}) = 0
	\end{cases}
	\end{align*}
	As we can see for $\ubar{\theta}$ second best $\ubar{q}^*$ coincides with the first best $\ubar{q}^*$. And $\ubar{t}^{*SB} = \ubar{\theta} u(\ubar{q}^*) - \Delta \theta u(\lbar{q}^*) < \lbar{t}^{*FB}$. However for $\lbar{\theta}$, $\lbar{q}^{*SB} < \lbar{q}^{*FB}$, if $\Delta \theta > 0$ (because $\frac{u'(\lbar{q})}{c'(\lbar{q})} = \frac{1}{\lbar{\theta} - \frac{1-\nu}{\nu}\Delta \theta}$ goes up once $\Delta \theta > 0$ which implies that $\lbar{q}$ declines). This also guarantees that $u(\lbar{q}) \le u(\ubar{q})$ because $\lbar{q}^{FB} < \ubar{q}^{FB}$.
	\item It is optimal to shut down the low-valuation type if:
	\begin{align*}
	(1 - \nu) (\ubar{\theta}u(\ubar{q}^{FB}) - c(\ubar{q}^{FB})) > \nu(\lbar{\theta}u(\lbar{q}^{FB}) - c(\lbar{q}^{FB})) + (1 - \nu)(\ubar{\theta}u(\lbar{q}^{FB}) - c(\lbar{q}^{FB}))
	\end{align*}
\end{enumerate}
\section*{3 Lending with adverse selection}
	There is a continuum of risk neutral borrowers with no personal wealth and limited
	liability. A proportion $\nu$ of borrowers (called type 1) have sure projects with return $h$ for
	an investment of $1$. A proportion $1-\nu$ of borrowers (called type 2) have (stochastically
	independent) projects with return $h$ only with probability $\theta$ in $(0; 1)$ and return $0$ with
	probability $1-\theta$, for an investment of $1$. If he does not apply for a loan, the borrower
	has an outside opportunity utility level of $u$.
	There is a single risk neutral bank available for loans which has a financing cost of
	$r$. The bank offers contracts to maximize its expected profit. For simplicity, we assume
	that all projects are socially valuable, i.e.,
	\begin{align*}
	\theta h > r + u
	\end{align*}
	\begin{enumerate}
	\item Explain why there is no loss of generality in considering the menus of contracts
	$\{(R_1, P_1),(R_2, P_2)\}$ where $P_i$
	is the probability of obtaining a loan and $R_i$
	is the
	repayment to the bank when the investment succeeds if the borrower announces
	that he is of type $i$.
	\item Write the maximization program of the bank which chooses the menu $\{(R_1, P_1),(R_2, P_2)\}$
	to maximize its expected profit under the borrower's participation and incentive
	constraints (for simplicity assume that if a borrower applies for a loan he loses his
	outside opportunity $u$).
	\item Show that the optimal contract entails a non-random allocation of loans (i.e., $P_i$
	is
	either $0$ or $1$, $i = 1, 2$). Characterize the optimal contract. Discuss its properties.
	Draw indifference curves (in the space $(R, P)$) and look at the Spence-Mirrlees
	condition. To answer this question, you will need the more general definition of the
	Spence-Mirrlees condition. Let U be the borrower's utility, then the single crossing
	condition says that $\frac{\partial }{\partial \theta}\left(\frac{\frac{\partial U}{\partial P}}{\frac{\partial U}{\partial R}}\right)$ has a constant sign.
	\end{enumerate}

\textbf{Solution}

\begin{enumerate}
	\item There is no loss of generality because of revelation principle. Which implies that set of contracts with number of options equal to cardinality of the type space, is sufficient to replicate any arbitrary complicated contract structure.
	\item The bank is maximizing:
	\begin{align*}
	\underset{(R_1, P_1), (R_2, P_2)}{\max}\ \nu P_1(R_1 - r) + (1 - \nu)P_2(\theta R_2 - r)\\
	s.t.\ \begin{cases}
	P_1 (h - R_1) \ge u\ (PC_1)\\
	P_2\theta(h - R_2) \ge u\ (PC_2)\\
	P_1(h - R_1) \ge P_2(h - R_2)\ (IC_1)\\
	P_2\theta (h - R_2) \ge P_1 \theta (h - R_1)\ (IC_2)
	\end{cases}
	\end{align*}
	\item As always, $(PC_1)$ is not binding because $P_1(h - R_1) \ge P_2(h - R_2) > P_2 \theta (h - R_2) \ge u$. $(PC_2)$ is binding because otherwise bank increases $P_1R_1, P_2R_2$ by the same value, which leads to increase in objective function without violation of any constraints. Moreover, $(IC_1)$ and $(IC_2)$ are also binding because $\theta$ is canceled out. The bank's problem looks as follows:
	\begin{align*}
	\underset{P_1, P_2}{\max}\ \nu P_1(h - \frac{u}{P_1\theta} - r) + (1 - \nu)P_2(\theta h - \frac{u}{P_2} - r)
	\end{align*} 
	The solution is:
	\begin{align*}
	\begin{cases}
	P_1^* = 1, R_1^* = h - \frac{u}{\theta}, h > r\\
	P_1^* = 0, \forall\ R_1, h < r\\
	P_1^* \in (0, 1), R_1^* = h - \frac{u}{\theta P_1}, h = r
	\end{cases}
	\begin{cases}
	P_2^* = 1, R_2^* = h - \frac{u}{\theta}, \theta h - r > 0\\
	P_2^* = 0, \forall\ R_2, \theta h - r < 0\\
	P_2^* \in (0, 1), R_2^* = h - \frac{u}{P_2 \theta}, \theta h - r = 0 
	\end{cases}
	\end{align*}
	As we can see, the optimal contract entails not only non-random allocation, for some values of parameters, $P_1$ or $P_2$ can be $\in (0, 1)$. Since I was asked to show that contracts are non-random, I will further assume that parameters are such that contracts are non-random. Depending on parameters three outcomes are possible: $P_1^* = P_2^* = 0$, if $h < r$, $P_1^* = P_2^* = 1$ if $\theta h < r < h$, or $P_1^* = 1, P_2^* = 0$ if $r < \theta h$. As we can see, if cost of financing are higher than possible returns then obviously bank does not finance anybody, if cost of financing are lower than expected return of risky project than bank finances both groups proposes the same contracts for both groups. If cost are higher than expected gains from risky project then bank stops dealing with risky investors. Since utility functions of agents are
	\begin{align*}
	U_1 &= P_1(h - R_1) - u\\
	U_2 &= P_2\theta (h - R_2) - u 
	\end{align*}
	the indifference curves look as follows.
	\begin{figure}[H]
		\centering
		\includegraphics[width=0.8\textwidth]{plotdraft}
		\caption{}\label{fig1}
	\end{figure}
Single-crossing condition:
\begin{align*}
\frac{\partial}{\partial \theta} \left(\frac{R-h}{P}\right) = 0
\end{align*}
has constant sign, however indifference curves obviously intersect only in infinity.
\end{enumerate}	
\section*{4 The bribing game}
	We consider an administration which is supposed to deliver with some fixed delay
	a service to the citizens (passport, permits,...). With the normal functioning of the
	administration, citizens derive a benefit $u_0$ which depends on their valuation of time.
	With some additional effort the official can deliver the service with a shorter delay.
	Let us call $q$ the decrease of delay that the official can provide at a cost $\frac{(q - Q)^2}{2}$
	for him
	where $Q$ is a constant.
	We assume that there is a proportion $\nu$ (resp. $1-\nu$) of type 1 (resp. type 2) citizens
	who derive a benefit from a decrease $q$ of delay equal to $\lbar{\theta}q$ ($\ubar{\theta}q$). Citizens are willing to
	bribe the official to decrease delays.
	Characterize the optimal bribing contract that the official will offer to the citizen
	
	
	\textbf{Solution}
	Assume that $\lbar{\theta} < \ubar{\theta}$. The official is solving:
	\begin{align*}
	\underset{(b_1, q_1), (b_2, q_2)}{\max}\ \nu\left(b_1 - \frac{(q_1 - Q)^2}{2}\right) + (1 - \nu) \left(b_2 - \frac{(q_2 - Q)^2}{2}\right)\\
	s.t.\ \begin{cases}
	\ubar{\theta}q_1 - b_1 \ge 0\ (PC_1)\\
	\lbar{\theta}q_2 - b_2 \ge 0\ (PC_2)\\
	\ubar{\theta}q_1 - b_1 \ge \ubar{\theta}q_2 - b_2\ (IC_1)\\
	\lbar{\theta} q_2 - b_2 \ge \lbar{\theta}q_1 - b_1\ (IC_2)
	\end{cases}
	\end{align*}
	Which is equivalent to:
	\begin{align*}
	&\underset{q_1, q_2}{\max}\ \nu \left(\ubar{\theta}q_1 - \Delta \theta q_2 - \frac{(q_1 - Q)^2}{2}\right) + (1 - \nu) \left(\lbar{\theta}q_2 -\frac{(q_2 - Q)^2}{2}\right) \\
	&s.t.\ \lbar{\theta}q_1 - b_1 \le 0
	\end{align*}
	FOCs:
	\begin{align*}
	\begin{cases}
	q_1^* = \ubar{\theta} + Q\\
	q_2^* = \lbar{\theta} + Q - \frac{\nu}{1 - \nu} \Delta \theta	
	\end{cases} \begin{cases}
	b_1^* = \ubar{\theta}(\ubar{\theta} + Q) - \Delta \theta (\lbar{\theta} +Q - \frac{\nu}{1 - \nu} \Delta \theta)\\
	b_2^* = \lbar{\theta}(\lbar{\theta} +Q - \frac{\nu}{1 - \nu} \Delta \theta)
	\end{cases}
	\end{align*}
	\begin{align*}
	b_1 = \ubar{\theta}(\ubar{\theta} + Q) - \Delta \theta (\lbar{\theta} +Q - \frac{\nu}{1 - \nu} \Delta \theta) > \ubar{\theta}(\ubar{\theta} + Q) - \Delta \theta (\ubar{\theta} + Q) = \lbar{\theta} q_1
	\end{align*}
	
	\section*{5 Flexibility}
		Since flexibility is always good, an employer who does not know the worker's productivity
		will benefit from adjusting the contract after the worker reveals his productivity. Discus
		
		\textbf{Solution}
		
		In some sense employer indeed benefits from renegotiation of the labour contract, because it can implement the most efficient contract. But in the long run such an opportunity can harm the employer because of the following reason. The employer will renegotiate the contract only with the low productivity worker, because the initial contract with this type of workers was deliberately worse than it could be, in order to prevent the high productivity type from pretending to be low. But if ex-ante the high productivity worker had known that tomorrow contract with low type is renegotiated, he would not have reveled his type, and still would have chosen to pretend being of low type. That is, in short run the employer can benefit, but in the long run without commitment he will lose, because nobody of the high type wants to reveal their type. This results has been demonstrated by Dewatripoint (QJE, 1989):
		\begin{quote}
			"Once the parties cannot precommit any more not to make
			Pareto-improving changes in the contract based on information
			revealed in previous periods, the outcome will be worse than in
			Grossman and Hart's model."
		\end{quote}
	\begin{quote}
		"For example, if at time $t$ it becomes common
		knowledge that the state of nature is $i$, then from $t + 1$ on
		employment has to be set at the Walrasian level for state $i$. Any
		other employment level would not be credible, since it would be
		known to be Pareto dominated and would be changed through
		renegotiation. In this sense, revealing information can be costly (ex
		ante), so that full instantaneous revelation may not be optimal."
	\end{quote}
		
		\section*{6 At your disservice}
		
		I don't want to complain, but my favourite lunchtime haunt has the rudest barman I've
		ever encountered - and I don't think it's an accident. The restaurant is famous for its
		superb, sophisticated Italian cooking and it prices accordingly. A romantic meal for two
		costs about \$150, plus the price of your selection from a wine list of biblical proportions.
		Even if you pick the cheapest main course for lunch and sip water, this frugal meal for
		one will set you back \$15.
		Or, you could sit at the bar or on one of the tables in the bar area. The food is still
		superb: you can fill up on rich, soft pork meatballs nestling on pillows of light polenta for
		about \$8. The veal ragu is rich but you don't have to be, because this perfect spaghetti
		is half the price of a pasta dish in the main restaurant.
		It sounds too good to be true, and I'm afraid there is a catch. To get to the food,
		you have to get past the barman who takes your orders, a man more Bond villain than
		bon vivant. To walk into that bar is to laugh in the face of fear. The barman welcomes
		me in with all the warmth of a Transylvanian butler in a B-movie. He drops menus on
		my table with a sneer. He repeatedly ignores my attempts to order. As he walks past,
		my companion whispers that he picked up his tattoos in a Russian prison. I think she's
		having me on, but I can't be sure.
		It's possible, of course, that this man is in the wrong job by accident, and when
		the restaurant owner reads this column and works out that it's about his restaurant,
		"Igor" will be fired. I'm not convinced. I think Igor is all part of the plan. One food
		critic described a previous bartender as acting like she'd just returned from a tax audit.
		"Venomous" is obviously in the job description.
		Why would a restaurant deliberately sabotage the dining experience?
		
		\textbf{Solution}
		
		
		It is an example of the second degree price discrimination. Unless existence of Igor, everybody went to the bar, get the same, superb food, but paying less. But the firm wants to separate those who can pay more from those who cannot afford going to the restaurant and reserving tables. To do so, the contracts the restaurant proposes should be incentive compatible. That is, the rich guys should not have incentives to deviate and to go to the bar and "underpay". Hence the owner of the restaurant has to decrease the quality of serving in bar, in order to prevent rich guys from going there.
		
		\section*{7 Revelation Principle}
			The revelation principle is a very powerful tool that greatly simplifies the search for optimal
			mechanisms. Think about the intuitive argument and the logic behind the revelation principle
			and, importantly, the assumptions this argument implicitly makes. Suggest a real-life situation
			where the revelation principle would fail. Explain why the logic of the revelation principle
			cannot be applied to your example.
			
			
			\textbf{Solution}
			
			The principle is based on the idea that even if instead of mere reporting his type, the agent reports some function of his type, which would be the best his option given the principal's policy, and then the principal uses some rule $\tilde{g}$ (see Fig.\ref{fig2}) to chose his action. It is equivalent to the agent just reporting the type. 
			\begin{figure}[H]
				\centering
				\includegraphics[width=0.8\textwidth]{DRP}
				\caption{}\label{fig2}
			\end{figure}
			If one see the picture above the most obvious example when this principle can fail is if functions $m^*(\cdot)$ or $\tilde{g}(\cdot)$ are not one-to-one. The real-life situation which can be described by such function is the following. Assume that instead of one principal there are two principals (e.g. two suppliers) and one agent (buyer) who has a capacity constraint $q_1 + q_2 \le \bar{K}$. Where $q_1, q_2$ the amount the agent buys from the first and the second supplier respectively. In this case even if the agent can be of two types: of high valuation and of low valuation. Each supplier should take into account the mechanism which the other principal uses, because it affects the quantity the agent buys from this principal. As a result, the revelation principal in its traditional form fails here, however Peters (Econometrica, 2001) showed that 
			\begin{quote}
				"In very general
				common agency problems in which there is only a single agent, principals do not need
				any more than menus of alternatives to respond to the mechanisms offered by the other
				principals."
			\end{quote}
\end{document}