\documentclass[a4paper]{article}
\usepackage[14pt]{extsizes} % 
\usepackage[utf8]{inputenc}
\usepackage{setspace,amsmath}
\usepackage{mathtools}
\usepackage{pgfplots}
\usepackage{titlesec}
\usepackage{pdfpages}
\usepackage[shortlabels]{enumitem}
\usepackage{tikz}
\usetikzlibrary{angles,quotes}
\usepackage{graphicx}
\usepackage{amssymb}
\usepackage{float}
\usepackage[section]{placeins}
\usepackage[makeroom]{cancel}
\usepackage{mathrsfs} % 
\newcommand\numberthis{\addtocounter{equation}{1}\tag{\theequation}}
%\addto\captionsrussian{\renewcommand{\figurename}{Fig.}}
\usepackage{amsmath,amsfonts,amssymb,amsthm,mathtools} 
\newcommand*{\hm}[1]{#1\nobreak\discretionary{}
{\hbox{$\mathsurround=0pt #1$}}{}}
\usepackage{graphicx}  % 
\graphicspath{{images/}{images2/}}  % 
\setlength\fboxsep{3pt} %  \fbox{} 
\setlength\fboxrule{1pt} % \fbox{}
\usepackage{wrapfig} % 
\newcommand{\prob}{\mathbb{P}}
\newcommand{\norma}{\mathscr{N}}
\newcommand{\expect}{\mathbb{E}}
\newcommand{\ubar}{\overline}
\newcommand{\lbar}{\underline}
\newcommand{\summa}{\sum_{i=1}^n}
\usepackage[left=7mm, top=20mm, right=15mm, bottom=20mm, nohead, footskip=10mm]{geometry} % 
\usepackage{tikz} % 
\def\myrad{2cm}% radius of the circle
\def\myanga{45}% angle for the arc
\def\myangb{195}
\begin{document} % 
	\begin{flushright}
	\begin{tabular}{r}
		Danil Fedchenko, MAE 2020, group A \\
	\end{tabular}
\end{flushright}


\begin{center}
	Microeconomics 4. Problem Set 3.
\end{center}
\section*{Adverse selection with indivisible good}
	Consider the model we studied in class but where the agent may produce only zero or
	one unit of the good. Principal's valuation of one unit is $S$. Agent's costs of producing
	one unit are $\theta$ that takes two values: $\underline{\theta}$ with probability $\nu$ and $\overline{\theta}$ with probability $1-\nu$,
	$\Delta \theta = \overline{\theta} - \underline{\theta} > 0$. When an agent with costs $\theta$ produces one unit, the principal gets $S-t$ and the agent gets $t - \theta$, $t$ being the transfer from the principal to the agent. No production entails no cost to the agent and no value to the principal. Production is always efficient:
	$S > \overline{\theta}$. The agent's reservation utility is zero.
	In the beginning of the period, the principal offers a menu $\{(\underline{p}, \underline{t}), (\overline{p}, \overline{t})\}$, where $\underline{p}$ and $\overline{p}$ are the probabilities of trade with the low and high cost type, respectively (note
	that the transfer is made independently of whether the trade occurs or not).
\begin{enumerate}
	\item Find the first best.
	\item Find the second best.
	\item Can the principal replicate the second-best solution by a simpler contract? Which
one?
\end{enumerate}


\textbf{Solution}

\begin{enumerate}
	\item Assume that once the trade does not occur, the agent does not produce anything. That means that the agent's objective function is $U^A(t, p, \theta) = t - p \theta$. The principal's objective function is $U^P(\underline{p}, \underline{\theta}, \overline{p}, \overline{\theta}) = \nu(\underline{p}S - \underline{t}) + (1 - \nu)(\overline{p}S - \overline{t})$. Obviously under complete information $\underline{t} = \underline{p}\underline{\theta}$ and $\overline{t} = \overline{p}\overline{\theta}$. And $\underline{p} = \overline{p} = 1$. That is, the first best is the following:
	\begin{align*}
	\{(1, \overline{\theta}), (1, \underline{\theta})\}
	\end{align*}
	\item Under incomplete information, since $\overline{\theta} > \underline{\theta}$ then $\overline{t} - \overline{p}\underline{\theta} = \overline{p} \Delta \theta > \underline{t} - \underline{p}\underline{\theta} = 0$. That is, the type $\underline{\theta}$ will pretend to be the type $\overline{\theta}$. Hence the principal should propose the contracts which would be incentive compatible. Thus, the principal is solving the following optimization problem:
	\begin{align*}
	\underset{(\underline{p}, \underline{t}), (\overline{t}, \overline{p})}{\max}\ \nu(\underline{p}S - \underline{t}) + (1 - \nu)(\overline{p}S - \overline{t})\\
	s.t.\ \begin{cases}
	\overline{t} - \overline{p} \overline{\theta} \ge 0\ (PC1)\\
	\underline{t} - \underline{p} \underline{\theta} \ge 0\ (PC2)\\
	\overline{t} - \overline{p} \overline{\theta} \ge \underline{t} - \underline{p} \overline{\theta}\ (IC1)\\
	\underline{t} - \underline{p} \underline{\theta} \ge \overline{t} - \overline{p} \underline{\theta}\ (IC2)
	\end{cases}
	\end{align*}
	$\underline{t} - \underline{p}\underline{\theta} \ge \overline{t} - \overline{p}\underline{\theta} > \overline{t} - \overline{p}\overline{\theta} \ge 0$, hence, $PC2$ is not binding. While $PC1$ is binding because otherwise the principal can decrease $\underline{t}, \overline{t}$ by the same value, which leads to increase of objective function without violating of any constraints. Moreover, $IC2$ should be binding either, because otherwise the principal can decrease $\underline{t}$. Finally, the principal's problem should look like:
	\begin{align*}
	&\underset{\overline{p}, \underline{p}}{\max}\ \nu(\underline{p}S - \overline{p}\Delta \theta - \underline{p}\underline{\theta}) + (1-\nu) \overline{p}(S - \overline{\theta})\\
	&s.t.\ \overline{p} \le \underline{p}\ (IC1)
	\end{align*}
	Since $\nu (S - \underline{\theta}) > 0$ then $\underline{p} = 1$. If $(1 - \nu)(S - \overline{\theta}) > \nu \Delta \theta$ then $\overline{p} = 1$ and $\overline{t} = \overline{\theta}, \underline{t} = \overline{\theta}$. Effectively, it means that the principal proposes just one contract $(1, \overline{\theta})$. If $(1 - \nu)(S - \overline{\theta}) < \nu \Delta \theta$ then $\overline{p} = 0$, and $\overline{t} = 0, \underline{t} = \underline{\theta}$. If $(1 - \nu)(S - \overline{\theta}) = \nu \Delta \theta$ then $\overline{p} \in [0, 1]$, $\overline{t} = \overline{p} \overline{\theta}, \underline{t} = \overline{p}\Delta \theta + \underline{\theta}$. So, the second best is the following:
	\begin{align*}
	\begin{cases}
	(1, \overline{\theta}), (1, \overline{\theta}),\ \frac{S - \overline{\theta}}{S - \underline{\theta}} > \nu\\
	(1, \underline{\theta}), (0, 0),\ \frac{S - \overline{\theta}}{S - \underline{\theta}} < \nu\\
	(1, \overline{p}\Delta \theta + \underline{\theta}), (\overline{p}, \overline{p}\overline{\theta}),\ \frac{S - \overline{\theta}}{S - \underline{\theta}} = \nu, \overline{p} \in (0, 1)
	\end{cases}
	\end{align*}
	\item By the revelation principle, to replicate the second best we need contracts with number of options equal to cardinality of types' space. Here the cardinality is equal to 2, hence we cannot replicate the second best with less number of options.
\end{enumerate}
\section*{2 Adverse selection with a privately informed buyer}
	Here we consider another standard adverse selection setting: a seller with known costs
	sells a good to a buyer who is privately informed about his valuation.
	The cost for the principal of $q$ units of good (or quality level $q$) is $c(q)$ and $c'(q) >
	0, c''(q) > 0, c(0) = 0$. The utility of the principal is $V = t-c(q)$. The agent's valuation
	for the good is $\theta u(q)$, where $u'(q) > 0, u''(q) < 0, u(0) = 0$ and $\theta = \underline{\theta}$ with probability
	$\nu$ and $\theta = \overline{\theta}$ with probability $1-\nu$. Denote $\Delta \theta = \overline{\theta}- \underline{\theta}> 0$. The utility of the agent is $U = \theta u(q) - t$.
	\begin{enumerate}
	\item Find the first best.
	\item Check the Spence-Mirlees condition.
	\item Find the second best.
	\item When it is optimal to "shut down" the low-valuation type?
	\end{enumerate}

\textbf{Solution}

\begin{enumerate}
	\item If the principle knows the agent's valuation then she should solve the following optimization problems:
	\begin{align*}
	\underset{(\overline{t}, \overline{q})}{\max}\ \overline{t} - c(\overline{q})\\
	s.t.\ \overline{\theta} u(\overline{q}) - \overline{t} \ge 0\\
	\underset{(\underline{t}, \underline{q})}{\max}\ \underline{t} - c(\underline{q})\\
	s.t.\ \underline{\theta} u(\underline{q}) - \underline{t} \ge 0\\
	\end{align*}
	The solutions can be derived from the following FOCs
	\begin{align*}
	\underline{\theta} u'(\underline{q}) - c'(\underline{q}) = 0 \to \underline{q}^* \to \underline{t}^* = \underline{\theta}u(\underline{q}^*)\\
	\overline{\theta} u'(\overline{q}) - c'(\overline{q}) = 0 \to \overline{q}^* \to \overline{t}^* = \overline{\theta}u(\overline{q}^*)
	\end{align*}
	\item The Spence-Mirrlees condition
	\begin{align*}
	\frac{\partial ^2}{\partial q \partial \theta} \theta u(q) > 0\\
	\frac{\partial ^2}{\partial q \partial \theta} \theta u(q) = u'(\theta) > 0
	\end{align*}
	hence the condition is satisfied.
	\item If at the first best $\underline{q}^* > 0$ then $\overline{\theta} u(\underline{q}^*) - \underline{\theta}u(\underline{q}^*) > 0$ i.e. under asymmetric information the $\ubar{\theta}$ type would mimic the $\lbar{\theta}$ type. Hence the second best should be incentive compatible. The principal should solve:
	\begin{align*}
	\underset{(\underline{t}, \underline{q}), (\overline{t}, \overline{q})}{\max}\ \nu(\underline{t} - c(\underline{q})) + (1 - \nu)(\overline{t} - c(\overline{q}))\\
	s.t.\ \begin{cases}
	\overline{\theta} u(\overline{q}) - \overline{t} \ge 0\ (PC1)\\
	\underline{\theta} u(\underline{q}) - \underline{t} \ge 0\ (PC2)\\
	\overline{\theta} u(\overline{q}) - \overline{t} \ge \overline{\theta} u(\underline{q}) - \underline{t}\ (IC1)\\
	\underline{\theta} u(\underline{q}) - \underline{t} \ge \underline{\theta} u(\overline{q}) - \overline{t}\ (IC2)
	\end{cases}
	\end{align*}
	$\ubar{\theta}u(\ubar{q}) - \ubar{t} \ge \ubar{\theta} u(\lbar{q}) - \lbar{t} > \lbar{\theta}u(\lbar{q}) - \lbar{t} > 0$ hence $PC1$ is not binding. While $PC2$ is binding because otherwise the principal can increase $\ubar{t}, \lbar{t}$ by the same amount which leads to increase of the objective function without the violation of any constraints. Moreover, the $IC1$ is binding because otherwise the principal can slightly increase $\ubar{t}$. Hence the optimization problem should like:
	\begin{align*}
	&\underset{(\ubar{q}, \lbar{q})}{\max}\ \nu(\lbar{\theta}u(\lbar{q}) - c(\lbar{q})) + (1 - \nu)(\ubar{\theta}u(\ubar{q}) - \Delta \theta u(\lbar{q}) - c(\ubar{q}))\\
	&s.t.\ u(\lbar{q}) \le u(\ubar{q})
	\end{align*}
	FOCs:
	\begin{align*}
	\begin{cases}
	\lbar{\theta}u'(\lbar{q}^*) - c'(\lbar{q}^*) - \frac{1 - \nu}{\nu} \Delta \theta u'(\lbar{q}^*) = 0\\
	\ubar{\theta}u'(\ubar{q}^*) - c'(\ubar{q}) = 0
	\end{cases}
	\end{align*}
	As we can see for $\ubar{\theta}$ second best $\ubar{q}^*$ coincides with the first best $\ubar{q}^*$. And $\ubar{t}^{*SB} = \ubar{\theta} u(\ubar{q}^*) - \Delta \theta u(\lbar{q}^*) < \lbar{t}^{*FB}$. However for $\lbar{\theta}$, $\lbar{q}^{*SB} < \lbar{q}^{*FB}$, if $\Delta \theta > 0$ (because $\frac{u'(\lbar{q})}{c'(\lbar{q})} = \frac{1}{\lbar{\theta} - \frac{1-\nu}{\nu}\Delta \theta}$ goes up once $\Delta \theta > 0$ which implies that $\lbar{q}$ declines). This also guarantees that $u(\lbar{q}) \le u(\ubar{q})$ because $\lbar{q}^{FB} < \ubar{q}^{FB}$.
	\item It is optimal to shut down the low-valuation type if:
	\begin{align*}
	(1 - \nu) (\ubar{\theta}u(\ubar{q}^{FB}) - c(\ubar{q}^{FB})) > \nu(\lbar{\theta}u(\lbar{q}^{FB}) - c(\lbar{q}^{FB})) + (1 - \nu)(\ubar{\theta}u(\lbar{q}^{FB}) - c(\lbar{q}^{FB}))
	\end{align*}
\end{enumerate}
\end{document}