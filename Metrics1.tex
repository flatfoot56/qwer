\documentclass[a4paper]{article}
\usepackage[14pt]{extsizes} % 
\usepackage[utf8]{inputenc}
\usepackage{setspace,amsmath}
\usepackage{mathtools}
\usepackage{pgfplots}
\usepackage{titlesec}
\usepackage{pdfpages}
\usepackage[shortlabels]{enumitem}
\usepackage{tikz}
\usetikzlibrary{angles,quotes}
\usepackage{graphicx}
\usepackage{amssymb}
\usepackage{float}
\usepackage[section]{placeins}
\usepackage[makeroom]{cancel}
\usepackage{mathrsfs} % 
\newcommand\numberthis{\addtocounter{equation}{1}\tag{\theequation}}
%\addto\captionsrussian{\renewcommand{\figurename}{Fig.}}
\usepackage{amsmath,amsfonts,amssymb,amsthm,mathtools} 
\newcommand*{\hm}[1]{#1\nobreak\discretionary{}
{\hbox{$\mathsurround=0pt #1$}}{}}
\usepackage{graphicx}  % 
\graphicspath{{images/}{images2/}}  % 
\setlength\fboxsep{3pt} %  \fbox{} 
\setlength\fboxrule{1pt} % \fbox{}
\usepackage{wrapfig} % 
\newcommand{\prob}{\mathbb{P}}
\newcommand{\norma}{\mathscr{N}}
\newcommand{\expect}{\mathbb{E}}
\usepackage[left=7mm, top=20mm, right=15mm, bottom=20mm, nohead, footskip=10mm]{geometry} % 
\usepackage{tikz} % 
\def\myrad{2cm}% radius of the circle
\def\myanga{45}% angle for the arc
\def\myangb{195}
\begin{document} % 
	\begin{flushright}
	\begin{tabular}{r}
		Danil Fedchenko, MAE 2020, group A \\
	\end{tabular}
\end{flushright}


\begin{center}
	Econometrics 1. Problem Set 1.
\end{center}
\section*{Problem 1}
 Suppose your calculate estimates for $\beta_0$ and $\beta_1$ by finding the solution to the following minimization problem:
 \begin{align*}
 \underset{b_0, b_1}{\min}\ \sum_{i=1}^n \text{exp} \left\{(y_i - b_0 - b_1x_i)^2\right\}
 \end{align*}
 Write down first-order conditions for the estimates.
 
 
 \textbf{Solution}
 
 
 \begin{align*}
 \frac{\partial}{\partial b_0}: -2 \sum_{i=1}^n (y_i - b_0 - b_ix_i)e^{(y_i - b_0 - b_ix_i)^2} = 0\\
 \frac{\partial}{\partial b_1}: -2 \sum_{i=1}^n x_i(y_i - b_0 - b_ix_i)e^{(y_i - b_0 - b_ix_i)^2} = 0
 \end{align*}
 \section*{Problem 2}
 In the simple linear regression model $y = \beta_0 + \beta_1x + u$, suppose that $\expect (u) \neq 0$. Letting $a_0 = \expect(u)$, show that the model can always be rewritten with the same slope, but a new
 intercept and error, where the new error has a zero expected value.
 
 
 
 \textbf{Solution}
 Assume $u' = a_0 - u$. Then $\expect(u') = a_0 - a_0 = 0$ and moreover
 \begin{align*}
 &y = \beta_0 + \beta_1 x + a_0 - u' = \beta_0' + \beta_1 x + u''\\
 &\text{where}\ \beta_0' = \beta_0 + a_0\\
 &\ \ \ u'' = -u'\\
 &\text{and}\ \ \expect(u'') = -\expect(u') = 0
 \end{align*}
 Q.E.D.
 \section*{Problem 3}
 Consider the standard simple linear regression model $y = \beta_0 + \beta_1 x + u$
 When $n = 3$, is it possible that the data point with maximal value of $y$ is located below the
 OLS regression line? If answer is yes, provide an example, if no, provide a proof.
 
 
 \textbf{Solution}
 
 
 Yes, it is possible. Assume $X = (1, 2, 5)$ and corresponded $Y = (2, 3, 4)$ then
 \begin{align*}
 b_1 = \frac{12}{26}\\
 b_0 = \frac{23}{13}
 \end{align*}
 and
 \begin{align*}
 \frac{12}{26} \cdot 5 + \frac{23}{13} = \frac{53}{13} \approx 4.08 > 4
 \end{align*}
 which means that the regression line is lying above the point (5, 4).
 
 
 \section*{Problem 4}
  Consider the following relation:
\begin{align*}
y = \sqrt{x} + u
\end{align*}
 where $x$ is uniformly distributed on segment $[0; 1]$ and error term $u$ has zero mean
 conditional on $x$. A random sample of size $n$ is collected: $\left\{x_i, y_i\right\}_{i=1}^n$. Some econometrician
 performs OLS estimation based on a simple linear model 
 \begin{align*}
	y = \beta_0 + \beta_1x + e
\end{align*}
 Based on her estimates $\hat{\beta_0}, \hat{\beta_1}$, the econometrician constructs the predictor for the conditional mean of the dependent variable, i.e. for $\expect(y|x)$, as $\widehat{\expect(y|x)} = \hat{\beta_0} + \hat{\beta_1} x$.
 \begin{enumerate}[a.]
 \item Calculate the expected value of y.
 \item Find the expected values of $\hat{\beta_0}, \hat{\beta_1}$ conditional on realized values of $x$.
 \end{enumerate}


\textbf{Solution}
\begin{enumerate}[a.]
	\item 
	\begin{align*}
	\expect[\sqrt{x}] = \int_{0}^1 \sqrt{x}dx = \frac{2}{3}\\
	\expect[u] = \expect[\expect[u|X]] = \expect[0] = 0\\
	\expect[y] = \frac{2}{3} + 0 = \frac{2}{3}
	\end{align*}
	\item Denote $\overline{\sqrt{x}} = \frac{1}{n}\sum_{i=1}^n \sqrt{x_i}$, then
	\begin{align*}
	\expect[\hat{\beta_1}|x_1, \dots, x_n] = \expect \left[\frac{\sum_{i=1}^n (x_i - \bar{x})(y_i - \bar{y})}{\sum_{i=1}^n (x_i - \bar{x})^2}\bigg|x_1, \dots, x_n\right] =\\
	= \expect \left[\frac{\sum_{i=1}^n (x_i - \bar{x})(\sqrt{x_i} + u_i - \overline{\sqrt{x}} -  \bar{u})}{\sum_{i=1}^n (x_i - \bar{x})^2}\bigg|x_1, \dots, x_n\right]
	\end{align*}
	Since
	\begin{align*}
	\expect \left[ \sum_{i=1}^n (x_i - \bar{x})(u_i - \bar{u})\bigg|x_1, \dots, x_n\right] = \expect \left[\sum_{i=1}^n (x_i - \bar{x})u_i\bigg|x_1, \dots, x_n\right]
	\end{align*}
	assuming independence of the sample, for each $i$
	\begin{align*}
	\expect[(x_i - \bar{x})u_i|x_1, \dots, x_n] = \expect[(x_i - \bar{x})u_i|x_i] = 0
	\end{align*}
	Thus, 
	\begin{align*}
	\expect[\hat{\beta_1}|x_1, \dots, x_n] = \frac{\sum_{i=1}^n (x_i - \bar{x})(\sqrt{x_i} - \overline{\sqrt{x}})}{\sum_{i=1}^n (x_i - \bar{x})^2}
	\end{align*}
	Similarly,
	\begin{align*}
	\expect[\hat{\beta_0}|x_1, \dots, x_n] = \expect[\bar{y} - \hat{\beta_1}\bar{x}|x_1, \dots, x_n] = \expect[\overline{\sqrt{x}} + \bar{u} - \hat{\beta_1}\bar{x}|x_1, \dots, x_n] = \\
	=\overline{\sqrt{x}} + \expect[\bar{u}|x_1, \dots, x_n] - \bar{x}\expect[\hat{\beta_1}|x_1, \dots, x_n] = \overline{\sqrt{x}} - \bar{x} \cdot \frac{\sum_{i=1}^n (x_i - \bar{x})(\sqrt{x_i} - \overline{\sqrt{x}})}{\sum_{i=1}^n (x_i - \bar{x})^2}
	\end{align*}
	\end{enumerate}
\section*{Problem 5}
 Using R, compute the sample mean and standard deviation of \textit{ahe} (average hourly earnings),
\textit{yrseduc} (years of education), and \textit{female}.



\textbf{Solution}


\begin{align*}
\text{mean(ahe)} = 15.19042\\
\text{sd(ahe)} = 8.027257\\
\text{mean(yrseduc)} = 13.46549\\
\text{sd(yrseduc)} = 2.479692\\
\text{mean(female)} = 0.4385083\\
\text{sd(female)} = 0.49627
\end{align*}
\section*{Problem 6}
 Estimate a regression of \textit{ahe} on \textit{yrseduc}.
 \begin{enumerate}[a.]
 	\item What is the coefficient on \textit{yrseduc}? Explain in words what it means. Is the numerical
value of your estimate large or small in an economic (real-world) sense?
\item Graph the data points and the estimated regression line.
\item Is the slope coefficient statistically significantly different from zero? Show how you reach
this conclusion. Use heteroskedasticity-robust standard errors to answer this question.
\item Report the 95\% confidence interval for $\beta_1$, the slope of the population regression line.
Use heteroskedasticity-robust standard errors to answer this question.
\item  What is the $R^2$ of this regression? What does this mean?
\item  Compute the correlation coefficient between ahe and yrseduc, and compare its square to
the $R^2$. How are the correlation coefficient and the $R^2$
related?
\item What is the root mean squared error of the regression? What does this mean?
\item Based on your graph from (b), does the error term appear to be homoskedastic or
heteroskedastic?

\end{enumerate}


\textbf{Solution}



The results of regression are depicted below
\begin{figure}[h]
	\centering
	\includegraphics[width=0.7\textwidth]{lm}
	\caption{Results of regression}
\end{figure}
\begin{enumerate}[a.]
	\item Coefficient on \textit{yrseduc} is 1.32186. That means that on each additional years of education the average size of the per-hour salary increases by 1.32186 dollars/hour. Or put it another way, we predict that ceteris paribus each additional year of education increases average hours earnings by 1.32186 dollars per hour. 
	
	
	If a standard working week lasts 40-45 hours that means that in average each additional year of education gives about $4 \cdot 45 \cdot 1.32186 \approx \$238$ additional dollars per month which is a rather reasonable number, I suppose.
	\item \begin{figure}[h]
		\centering
		\includegraphics[width=0.7\textwidth]{Rplot}
		\caption{Regression line and data points}
	\end{figure}
\newpage
\item Using coeftest one can get:
 \begin{figure}[h]
	\centering
	\includegraphics[width=0.7\textwidth]{coeftest}
	\caption{t-test of coefficients}
\end{figure}



that is, the slope significantly differs from zero.
\item Using confint one can get:
\begin{figure}[h]
	\centering
	\includegraphics[width=0.6\textwidth]{confint}
	\caption{Confidence intervals for coefficients}
\end{figure}




that is, the 95\% confidence interval for the slope is (1.227613, 1.416104).
\item $R^2$ is equal to 0.1667. This number is rather close to zero that means that regression hardly explains the actual interaction between variables (this conclusion can be drawn just by the glance at the plot on the Fig. 2).
\item Correlation coefficient between \textit{ahe} and \textit{yrseduc} is equal to 0.4083341, and 
\begin{align*}
0.4083341^2 = 0.1667 = R^2
\end{align*}
For the regression with the single regressor the square of the correlation coefficient is exactly equal to $R^2$.
\item 
\begin{align*}
\text{RMSE} = \sqrt{\frac{1}{n}\sum_{i=1}^n u_i^2} = 7.326572
\end{align*}
where $u_i = \hat{y_i} - y_i$.
\item As for me the error term is in fact heteroskedastic.
\end{enumerate}


\section*{Problem 7}
Estimate a regression of \textit{ahe} on \textit{female}.
\begin{enumerate}[a.]
\item What is the coefficient on \textit{female}? Explain in words what this means. Is the numerical
value of your estimate large or small in an economic (real-world) sense?
\item Test the hypothesis that average hourly earnings are the same for male and female
workers, against the alternative that they differ, at the 5\% significance level. Make sure to use
heteroskedasticity-robust standard errors to answer this question.
\item Compute the sample average of \textit{ahe} for women, and the sample average of \textit{ahe} for men;
from this compute an estimate of the “gender gap” in earnings; and construct the t-statistic
testing the hypothesis that the gender gap is zero. Make sure to use heteroskedasticity-robust
standard errors to answer this question. Compare your results to those in parts (a) and (b).

\end{enumerate}



\textbf{Solution}


Results of regression are depicted below



\begin{figure}[H]
	\centering
	\includegraphics[width=0.6\textwidth]{lm_female}
	\caption{Results of regression}
\end{figure}


\begin{enumerate}[a.]
	\item Coefficient on \textit{female} is -3.2026. That means that average per-hour salary falls by 3.2026 dollars per hour for women compare with the one for men. The number is also rather reasonable.
	\item If average hourly earnings for women and men are equal then the slope coefficient should be equal to zero. Hence, let us test the hypothesis whether the slope coefficient is equal to zero. Using coeftest one can get:
	
	
	\begin{figure}[H]
		\centering
		\includegraphics[width=0.6\textwidth]{coeftest_female}
		\caption{Test of coefficients}
	\end{figure}

	that is, on the significance level of 5\% null is rejected that is, average hourly earnings are actually not the same for men and for women.
	\item Means for women's $\bar{w}$ and men's $\bar{m}$ average hourly earnings are:
	\begin{align*}
	\bar{w} = 13.3922\\
	\bar{m} = 16.59478\\
	\end{align*}
	and
	\begin{align*}
	s_w = 6.916653\\
	s_m = 8.539754\\
	\end{align*}
	hence assuming normality of data
	\begin{align*}
	t = \frac{\bar{w} - \bar{m}}{\sqrt{\frac{s_m^2}{n_m} + \frac{s_w^2}{n_w}}}
	\end{align*}
	then using t.test() one can get the following results:
		\begin{figure}[H]
		\centering
		\includegraphics[width=0.6\textwidth]{t_test}
		\caption{Results of the t-test for testing the equality of means in two groups}
	\end{figure}
that is we can reject the null hypothesis which means that there is in fact a "gender gap". This conclusion is fully consistent with those were gotten in (a) and (b).
\end{enumerate}
\end{document}