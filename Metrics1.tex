\documentclass[a4paper]{article}
\usepackage[14pt]{extsizes} % 
\usepackage[utf8]{inputenc}
\usepackage{setspace,amsmath}
\usepackage{mathtools}
\usepackage{pgfplots}
\usepackage{titlesec}
\usepackage{pdfpages}
\usepackage[shortlabels]{enumitem}
\usepackage{tikz}
\usetikzlibrary{angles,quotes}
\usepackage{graphicx}
\usepackage{amssymb}
\usepackage[section]{placeins}
\usepackage[makeroom]{cancel}
\usepackage{mathrsfs} % 
\newcommand\numberthis{\addtocounter{equation}{1}\tag{\theequation}}
%\addto\captionsrussian{\renewcommand{\figurename}{Fig.}}
\usepackage{amsmath,amsfonts,amssymb,amsthm,mathtools} 
\newcommand*{\hm}[1]{#1\nobreak\discretionary{}
{\hbox{$\mathsurround=0pt #1$}}{}}
\usepackage{graphicx}  % 
\graphicspath{{images/}{images2/}}  % 
\setlength\fboxsep{3pt} %  \fbox{} 
\setlength\fboxrule{1pt} % \fbox{}
\usepackage{wrapfig} % 
\newcommand{\prob}{\mathbb{P}}
\newcommand{\norma}{\mathscr{N}}
\newcommand{\expect}{\mathbb{E}}
\usepackage[left=7mm, top=20mm, right=15mm, bottom=20mm, nohead, footskip=10mm]{geometry} % 
\usepackage{tikz} % 
\def\myrad{2cm}% radius of the circle
\def\myanga{45}% angle for the arc
\def\myangb{195}
\begin{document} % 
	\begin{flushright}
	\begin{tabular}{r}
		Danil Fedchenko, MAE 2020, group A \\
	\end{tabular}
\end{flushright}


\begin{center}
	Econometrics 1. Problem Set 1.
\end{center}
\section*{Problem 1}
 Suppose your calculate estimates for $\beta_0$ and $\beta_1$ by finding the solution to the following minimization problem:
 \begin{align*}
 \underset{b_0, b_1}{\min}\ \sum_{i=1}^n \text{exp} \left\{(y_i - b_0 - b_1x_i)^2\right\}
 \end{align*}
 Write down first-order conditions for the estimates.
 
 
 \textbf{Solution}
 
 
 \begin{align*}
 \frac{\partial}{\partial b_0}: -2 \sum_{i=1}^n (y_i - b_0 - b_ix_i)e^{(y_i - b_0 - b_ix_i)^2} = 0\\
 \frac{\partial}{\partial b_1}: -2 \sum_{i=1}^n x_i(y_i - b_0 - b_ix_i)e^{(y_i - b_0 - b_ix_i)^2} = 0
 \end{align*}
 \section*{Problem 2}
 In the simple linear regression model $y = \beta_0 + \beta_1x + u$, suppose that $\expect (u) \neq 0$. Letting $a_0 = \expect(u)$, show that the model can always be rewritten with the same slope, but a new
 intercept and error, where the new error has a zero expected value.
 
 
 
 \textbf{Solution}
 Assume $u' = a_0 - u$. Then $\expect(u') = a_0 - a_0 = 0$ and moreover
 \begin{align*}
 &y = \beta_0 + \beta_1 x + a_0 - u' = \beta_0' + \beta_1 x + u''\\
 &\text{where}\ \beta_0' = \beta_0 + a_0\\
 &\ \ \ u'' = -u'\\
 &\text{and}\ \ \expect(u'') = -\expect(u') = 0
 \end{align*}
 Q.E.D.
 \section*{Problem 3}
 Consider the standard simple linear regression model $y = \beta_0 + \beta_1 x + u$
 When $n = 3$, is it possible that the data point with maximal value of $y$ is located below the
 OLS regression line? If answer is yes, provide an example, if no, provide a proof.
 
 
 \textbf{Solution}
 
 
 Yes, it is possible. Assume $X = (1, 2, 5)$ and corresponded $Y = (2, 3, 4)$ then
 \begin{align*}
 b_1 = \frac{12}{26}\\
 b_0 = \frac{23}{13}
 \end{align*}
 and
 \begin{align*}
 \frac{12}{26} \cdot 5 + \frac{23}{13} = \frac{53}{13} \approx 4.08 > 4
 \end{align*}
 which means that the regression line is lying above the point (5, 4).
 
 
 \section*{Problem 4}
  Consider the following relation:
\begin{align*}
y = \sqrt{x} + u
\end{align*}
 where $x$ is uniformly distributed on segment $[0; 1]$ and error term $u$ has zero mean
 conditional on $x$. A random sample of size $n$ is collected: $\left\{x_i, y_i\right\}_{i=1}^n$. Some econometrician
 performs OLS estimation based on a simple linear model 
 \begin{align*}
	y = \beta_0 + \beta_1x + e
\end{align*}
 Based on her estimates $\hat{\beta_0}, \hat{\beta_1}$, the econometrician constructs the predictor for the conditional mean of the dependent variable, i.e. for $\expect(y|x)$, as $\widehat{\expect(y|x)} = \hat{\beta_0} + \hat{\beta_1} x$.
 \begin{enumerate}[a.]
 \item Calculate the expected value of y.
 \item Find the expected values of $\hat{\beta_0}, \hat{\beta_1}$ conditional on realized values of $x$.
 \end{enumerate}


\textbf{Solution}
\begin{enumerate}[a.]
	\item 
	\begin{align*}
	\expect[\sqrt{x}] = \int_{0}^1 \sqrt{x}dx = \frac{2}{3}\\
	\expect[u] = \expect[\expect[u|X]] = \expect[0] = 0\\
	\expect[y] = \frac{2}{3} + 0 = \frac{2}{3}
	\end{align*}
	\item Denote $\overline{\sqrt{x}} = \frac{1}{n}\sum_{i=1}^n \sqrt{x_i}$, then
	\begin{align*}
	\expect[\hat{\beta_1}|x_1, \dots, x_n] = \expect \left[\frac{\sum_{i=1}^n (x_i - \bar{x})(y_i - \bar{y})}{\sum_{i=1}^n (x_i - \bar{x})^2}\bigg|x_1, \dots, x_n\right] =\\
	= \expect \left[\frac{\sum_{i=1}^n (x_i - \bar{x})(\sqrt{x_i} + u_i - \overline{\sqrt{x}} -  \bar{u})}{\sum_{i=1}^n (x_i - \bar{x})^2}\bigg|x_1, \dots, x_n\right]
	\end{align*}
	Since
	\begin{align*}
	\expect \left[ \sum_{i=1}^n (x_i - \bar{x})(u_i - \bar{u})\bigg|x_1, \dots, x_n\right] = \expect \left[\sum_{i=1}^n (x_i - \bar{x})u_i\bigg|x_1, \dots, x_n\right]
	\end{align*}
	assuming independence of the sample, for each $i$
	\begin{align*}
	\expect[(x_i - \bar{x})u_i|x_1, \dots, x_n] = \expect[(x_i - \bar{x})u_i|x_i] = 0
	\end{align*}
	Thus, 
	\begin{align*}
	\expect[\hat{\beta_1}|x_1, \dots, x_n] = \frac{\sum_{i=1}^n (x_i - \bar{x})(\sqrt{x_i} - \overline{\sqrt{x}})}{\sum_{i=1}^n (x_i - \bar{x})^2}
	\end{align*}
	Similarly,
	\begin{align*}
	\expect[\hat{\beta_0}|x_1, \dots, x_n] = \expect[\bar{y} - \hat{\beta_1}\bar{x}|x_1, \dots, x_n] = \expect[\overline{\sqrt{x}} + \bar{u} - \hat{\beta_1}\bar{x}|x_1, \dots, x_n] = \\
	=\overline{\sqrt{x}} + \expect[\bar{u}|x_1, \dots, x_n] - \bar{x}\expect[\hat{\beta_1}|x_1, \dots, x_n] = \overline{\sqrt{x}} - \bar{x} \cdot \frac{\sum_{i=1}^n (x_i - \bar{x})(\sqrt{x_i} - \overline{\sqrt{x}})}{\sum_{i=1}^n (x_i - \bar{x})^2}
	\end{align*}
	\end{enumerate}
\section*{Problem 5}
 Using R, compute the sample mean and standard deviation of \textit{ahe} (average hourly earnings),
\textit{yrseduc} (years of education), and \textit{female}.



\textbf{Solution}


\begin{align*}
\text{mean(ahe)} = 15.19042\\
\text{sd(ahe)} = 8.027257\\
\text{mean(yrseduc)} = 13.46549\\
\text{sd(yrseduc)} = 2.479692\\
\text{mean(female)} = 0.4385083\\
\text{sd(female)} = 0.49627
\end{align*}
\section*{Problem 6}
 Estimate a regression of \textit{ahe} on \textit{yrseduc}.
 \begin{enumerate}
 	\item What is the coefficient on \textit{yrseduc}? Explain in words what it means. Is the numerical
value of your estimate large or small in an economic (real-world) sense?
\item Graph the data points and the estimated regression line.
\item Is the slope coefficient statistically significantly different from zero? Show how you reach
this conclusion. Use heteroskedasticity-robust standard errors to answer this question.
\item Report the 95\% confidence interval for $\beta_1$, the slope of the population regression line.
Use heteroskedasticity-robust standard errors to answer this question.
\item  What is the $R^2$ of this regression? What does this mean?
\item  Compute the correlation coefficient between ahe and yrseduc, and compare its square to
the $R^2$. How are the correlation coefficient and the $R^2$
related?
\item What is the root mean squared error of the regression? What does this mean?
\item Based on your graph from (b), does the error term appear to be homoskedastic or
heteroskedastic?

\end{enumerate}


\textbf{Solution}
\begin{enumerate}
	\item Coefficient on \textit{yrseduc} is 1.32186. That means that on each additional years of education the average size of the per-hour salary increases by 1.32186 dollars/hour.
	\item \begin{figure}[h]
		\centering
		\includegraphics[width=0.7\textwidth]{Rplot}
		\caption{Regression line and data points}
	\end{figure}
\end{enumerate}
\end{document}