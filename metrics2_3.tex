\documentclass[a4paper]{article}
\usepackage[14pt]{extsizes} % 
\usepackage[utf8]{inputenc}
\usepackage{setspace,amsmath}
\usepackage{mathtools}
\usepackage{pgfplots}
\usepackage{titlesec}
\usepackage{pdfpages}
\usepackage[shortlabels]{enumitem}
\usepackage{tikz}
\usetikzlibrary{angles,quotes}
\usepackage{graphicx}
\usepackage{amssymb}
\usepackage{float}
\usepackage[section]{placeins}
\usepackage[makeroom]{cancel}
\usepackage{mathrsfs} % 
\newcommand\numberthis{\addtocounter{equation}{1}\tag{\theequation}}
%\addto\captionsrussian{\renewcommand{\figurename}{Fig.}}
\usepackage{amsmath,amsfonts,amssymb,amsthm,mathtools} 
\newcommand*{\hm}[1]{#1\nobreak\discretionary{}
{\hbox{$\mathsurround=0pt #1$}}{}}
\usepackage{graphicx}  % 
\graphicspath{{images/}{images2/}}  % 
\setlength\fboxsep{3pt} %  \fbox{} 
\setlength\fboxrule{1pt} % \fbox{}
\usepackage{wrapfig} % 
\newcommand{\prob}{\mathbb{P}}
\newcommand{\norma}{\mathscr{N}}
\newcommand{\expect}{\mathbb{E}}
\newcommand{\summa}{\sum_{i=1}^n}
\newcommand{\yrseduc}{\textit{yrseduc}}
\usepackage[left=7mm, top=20mm, right=15mm, bottom=20mm, nohead, footskip=10mm]{geometry} % 
\usepackage{tikz} % 
\def\myrad{2cm}% radius of the circle
\def\myanga{45}% angle for the arc
\def\myangb{195}
\begin{document} % 
	\begin{flushright}
	\begin{tabular}{r}
		Danil Fedchenko, MAE 2020, group A \\
	\end{tabular}
\end{flushright}


\begin{center}
	Econometrics 2. Problem Set 3.
\end{center}
\section*{Problem 1}
In Example 12.8, we found evidence of heteroskedasticity in $u_t$ in equation (12.47). Thus, we compute the heteroskedasticity-robust standard errors along with the usual standard errors. What does using the heteroskedasticity-robust $t$ statistic do to the significance of $return_{t-1}$?

\textbf{Solution}

\begin{align*}
t_{old} = \frac{0.059}{0.038} = 1.55\\
t_{new} = \frac{0.059}{0.069} = 0.85
\end{align*}
Significance of $return_{t-1}$ becomes smaller because for non-robust case p-value is $0.06$ while in robust case it is $0.19$ i.e. for example at the significant level of 10\% non-robust t-test reject hypothesis that $return_{t-1}$ does not affect $return_t$ while robust test does not.


\section*{Problem 2}
\begin{align*}
\hat{\nu} = \sum_{t=1}^n \hat{a}_t^2 + 2 \sum_{h=1}^g \frac{g+1-h}{g+1} \left(\sum_{t = h+1}^n \hat{a}_{t}\hat{a}_{t-h}\right)
\end{align*}
To prove that $\hat{\nu} \ge 0$ firstly let us fixed some $0<g<n$. Denote 
\begin{align*}
\mu_h = \sum_{t = h+1}^n\hat{a}_t \hat{a}_{t-h}
\end{align*}
This implies
\begin{align*}
\hat{\nu} = \mu_0 + 2\sum_{h=1}^g \frac{g+1-h}{g+1}\mu_h
\end{align*}
Then let us assume that $a_t \equiv 0,\ \forall\ t < 1, t > n$. Denote
\begin{align*}
x_i &= \begin{pmatrix}
a_{i+1}\\
a_{i+2}\\
\dots\\
\dots\\
a_{i+n}
\end{pmatrix}\\
\sum_{i = -n + 1}^{n-1} x_i x_i^T &= \begin{pmatrix}
\mu_0 & \mu_1 & \mu_2 & \dots & \mu_{n-1}\\
\mu_1 & \mu_0 & \mu_1 & \dots & \mu_{n-2}\\
\dots & \dots & \dots & \dots & \dots \\
\dots & \dots & \dots & \dots & \dots \\
\mu_{n-1} & \mu_{n-2} & \mu_{n-3} & \dots & \mu_0
\end{pmatrix} = M
\end{align*} 
\begin{align*}
\forall\ v \in \mathbb{R}^{n}\ v^TMv = \sum_{i=-n+1}^{n-1} v^Tx_ix_i^Tv = \sum_{i=-n+1}^{n-1} (x_i^Tv)^T(x_i^Tv) \ge 0
\end{align*}
That is matrix, $M$ is positive semi-definite. Denote vector $s \in \mathbb{R}^n$ such that first $g+1$ entries are 1 and other are 0. Then
\begin{align*}
\frac{1}{g+1}s^TMs = \mu_0 + 2\frac{g}{g+1}\mu_1 + 2\frac{g-1}{g+1}\mu_2 + \dots + 2 \frac{1}{g+1}\mu_g = \hat{\nu} \ge 0
\end{align*}
Q.E.D.
\end{document}