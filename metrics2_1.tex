\documentclass[a4paper]{article}
\usepackage[14pt]{extsizes} % 
\usepackage[utf8]{inputenc}
\usepackage{setspace,amsmath}
\usepackage{mathtools}
\usepackage{pgfplots}
\usepackage{titlesec}
\usepackage{pdfpages}
\usepackage[shortlabels]{enumitem}
\usepackage{tikz}
\usetikzlibrary{angles,quotes}
\usepackage{graphicx}
\usepackage{amssymb}
\usepackage{float}
\usepackage{makecell}
\usepackage[section]{placeins}
\usepackage[makeroom]{cancel}
\usepackage{mathrsfs} % 
\newcommand\numberthis{\addtocounter{equation}{1}\tag{\theequation}}
%\addto\captionsrussian{\renewcommand{\figurename}{Fig.}}
\usepackage{amsmath,amsfonts,amssymb,amsthm,mathtools} 
\newcommand*{\hm}[1]{#1\nobreak\discretionary{}
{\hbox{$\mathsurround=0pt #1$}}{}}
\usepackage{graphicx}  % 
\graphicspath{{images/}{images2/}}  % 
\setlength\fboxsep{3pt} %  \fbox{} 
\setlength\fboxrule{1pt} % \fbox{}
\usepackage{wrapfig} % 
\newcommand{\prob}{\mathbb{P}}
\newcommand{\norma}{\mathscr{N}}
\newcommand{\expect}{\mathbb{E}}
\newcommand{\summa}{\sum_{i=1}^n}
\newcommand{\yrseduc}{\textit{yrseduc}}
\usepackage[left=7mm, top=20mm, right=15mm, bottom=20mm, nohead, footskip=10mm]{geometry} % 
\usepackage{tikz} % 
\def\myrad{2cm}% radius of the circle
\def\myanga{45}% angle for the arc
\def\myangb{195}
\begin{document} % 
	\begin{flushright}
	\begin{tabular}{r}
		Danil Fedchenko, MAE 2020, group A \\
	\end{tabular}
\end{flushright}


\begin{center}
	Econometrics 2. Problem Set 1.
\end{center}
\section*{Problem 1}
 In Example 10.4, we saw that our estimates of the individual lag coefficients in a distributed lag model were very imprecise. One way to alleviate the multicollinearity problem is to assume that the $\delta_j$ follow a relatively simple pattern. For concreteness, consider a model with four lags:
 \begin{align*}
y_t = \alpha_0 + \delta_0z_t + \delta_1 z_{t-1} + \delta_2 z_{t-2} + \delta_3 z_{t-3} + \delta_4 z_{t-4}+ u_t.
\end{align*} 
Now, let us assume that the $\delta_j$ follow a quadratic in the lag, $j$: 
\begin{align*}
\delta_j = \gamma_0 + \gamma_1 j + \gamma_2 j^2,
\end{align*}
for parameters $\gamma_0, \gamma_1$, and $\gamma_2$. This is an example of a polynomial distributed lag (PDL) model. 
\begin{enumerate}[(i)]
	\item  Plug the formula for each $\delta_j$ into the distributed lag model and write the model in terms of the parameters $\gamma_h$, for $h = 0,1,2$. 
	\item Explain the regression you would run to estimate the $\gamma_h$. 
	\item The polynomial distributed lag model is a restricted version of the general model. How many restrictions are imposed? How would you test these? (Hint: Think F test.)
\end{enumerate}


\textbf{Solution}

\begin{enumerate}[(i)]
	\item 
	\begin{align*}
	y_t = \alpha_0 + \gamma_0 z_t + (\gamma_0 + \gamma_1 + \gamma_2)z_{t-1}+(\gamma_0 + 2\gamma_1 + 4\gamma_2) z_{t-2} + (\gamma_0 + 3\gamma_1 + 9\gamma_2)z_{t-3} + \\
	+(\gamma_0 + 4\gamma_1 + 16\gamma_2)z_{t-4} + u_t\\
	y_t = \alpha_0 + (z_t + z_{t-1} + z_{t-2} + z_{t-3}+z_{t-4})\gamma_0 + (z_{t-1} + 2z_{t-2}+3z_{t-3}+4z_{t-4})\gamma_1 +\\
	+(z_{t-1} + 4z_{t-2} + 9z_{t-3} + 16z_{t-4})\gamma_2 + u_t
	\end{align*}
	\item I would create new variables $\tilde{z}_{1t} = z_t + z_{t-1} + z_{t-2} + z_{t-3}+z_{t-4}$, $\tilde{z}_{2t} = z_{t-1} + 2z_{t-2}+3z_{t-3}+4z_{t-4}$ and $\tilde{z}_{3t} = z_{t-1} + 4z_{t-2} + 9z_{t-3} + 16z_{t-4}$ and run the regression $y_t$ on $1, \tilde{z}_{1t}, \tilde{z}_{2t}, \tilde{z}_{3t}$, using OLS then to estimate $\gamma_h$
	\item In general model we have 6 parameters whereas in restricted model only 4, that means that 2 restrictions were imposed. Using formula for F-statistic through the $R^2$ of restricted and unrestricted models, namely:
	\begin{align*}
	F = \frac{(R^2_{ur} - R^2_r)(n-6)}{2(1 - R^2_{ur})}
	\end{align*}
	we can test the hypothesis whether restrictions are true or not. Under null hypothesis F-statistics has asymptotically $\frac{1}{2} \chi^2(2)$ distribution. And under assumption of normality of errors we can use also $F(2, n-6)$ distribution.
\end{enumerate}
\section*{Problem 2}
 In Example 10.4, we wrote the model that explicitly contains the long-run propensity, $\theta_0$, as 
 \begin{align*} 
 gfr_t = \alpha_0 + \theta_0 pe_t + \delta_1 (pe_{t-1} - pe_t) + \delta_2 (pe_{t-2} - pe_t) + u_t,
 \end{align*}  
 where we omit the other explanatory variables for simplicity. As always with multiple regression analysis, $\theta_0$ should have a ceteris paribus interpretation. Namely, if $pe_t$  increases by one (dollar) holding ($pe_{t-1} - pe_t$) and ($pe_{t-2} - pe_t$) fixed, $gfr_t$ should change by $\theta_0$. 
 \begin{enumerate}[(i)]
\item  If ($pe_{t - 1} - pe_t$) and ($pe_{t-2} - pe_t$) are held fixed but $pe_t$ is increasing, what must be true about changes in $pe_{t-1}$ and $pe_{t-2}$?
\item How does your answer in part (i) help you to interpret $\theta_0$ in the above equation as the LRP? 
\end{enumerate}

\textbf{Solution}

\begin{enumerate}[(i)]
	\item If $pe_t$ increases while $pe_{t-1} - pe_t$ remains unchanged that means that $pe_{t-1}$ should also increase by the same as $pe_t$, as well as $pe_{t-2}$.
	\item As the LRP $\theta_0$ shows the increase of $gfr$ in the long-run (from $t+2$ to $n$) once $pe_t$ permanently changes by 1 starting from period $t$. The result in (i) demonstrates that $\theta_0$ shows the increase in $gfr$ once starting from $t-2$ the $pe$ changes permanently. Obviously the results are equivalent up to renumbering of observations' indices. 
\end{enumerate}
\end{document}