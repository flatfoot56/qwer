\documentclass[a4paper]{article}
\usepackage[14pt]{extsizes} % 
\usepackage[utf8]{inputenc}
\usepackage{setspace,amsmath}
\usepackage{mathtools}
\usepackage{pgfplots}
\usepackage{titlesec}
\usepackage{pdfpages}
\usepackage[shortlabels]{enumitem}
\usepackage{tikz}
\usetikzlibrary{angles,quotes}
\usepackage{graphicx}
\usepackage{amssymb}
\usepackage{float}
\usepackage[section]{placeins}
\usepackage[makeroom]{cancel}
\usepackage{mathrsfs} % 
\newcommand\numberthis{\addtocounter{equation}{1}\tag{\theequation}}
%\addto\captionsrussian{\renewcommand{\figurename}{Fig.}}
\usepackage{amsmath,amsfonts,amssymb,amsthm,mathtools} 
\newcommand*{\hm}[1]{#1\nobreak\discretionary{}
{\hbox{$\mathsurround=0pt #1$}}{}}
\usepackage{graphicx}  % 
\graphicspath{{images/}{images2/}}  % 
\setlength\fboxsep{3pt} %  \fbox{} 
\setlength\fboxrule{1pt} % \fbox{}
\usepackage{wrapfig} % 
\newcommand{\prob}{\mathbb{P}}
\newcommand{\norma}{\mathscr{N}}
\newcommand{\expect}{\mathbb{E}}
\newcommand{\summa}{\sum_{i=1}^n}
\usepackage[left=7mm, top=20mm, right=15mm, bottom=20mm, nohead, footskip=10mm]{geometry} % 
\usepackage{tikz} % 
\def\myrad{2cm}% radius of the circle
\def\myanga{45}% angle for the arc
\def\myangb{195}
\begin{document} % 
	\begin{flushright}
	\begin{tabular}{r}
		Danil Fedchenko, MAE 2020, group A \\
	\end{tabular}
\end{flushright}


\begin{center}
	Game theory. Problem Set 1.
\end{center}
\section*{Problem 1}
John and Peter are bidding for an indivisible object. The rules of the
auction are as follows. The individual who places the highest bid wins the
object. If John wins he pays the amount of money Peter bade. If Peter
wins he pays the amount of money he bade. Both players place their bids
simultaneously. For each individual $i \in \left\{J, P\right\}$ let $v_i \ge 0$ denote their
valuation for the object and let $b_i \ge 0$ denote the bid they placed. Their
preferences are captured by the following payoff functions for John and
Peter respectively:
\begin{align*}
u_J(b_J, b_P) = \begin{cases}
v_J - b_P, &b_J > b_P\\
\frac{1}{2}(v_J - b_P), &b_J = b_P\\
0, &b_J < b_P
\end{cases}\\
u_P(b_J, b_P) = \begin{cases}
v_P - b_P, &b_P > b_J\\
\frac{1}{2}(v_P - b_P), &b_P = b_J\\
0, &b_P < b_J
\end{cases}
\end{align*}
The information above (including valuations) is common knowledge. Does
either player have a weakly dominant strategy to bid his valuation? Can
you find values for the parameters $(v_J , v_P ) \in \mathbb{R}^2_+$ such that the pair of
strategies $(b_J , b_P ) = (v_J , v_P)$ constitutes a Nash Equilibrium? For each
answer you give provide the relevant demonstration.


\textbf{Solution}


Suppose that Peter's strategy is to bid his valuation $v_P$. Then
\begin{align*}
u_P(b_J, v_P) = 0 < u_P\left(b_J, \frac{b_J + v_P}{2}\right) = \frac{v_P - b_J}{2}, \text{for }b_J < v_P\\
\end{align*}
hence such a strategy cannot be (weakly) dominant for Peter.
Suppose that John's strategy is to bid his valuation $v_J$. Then
\begin{align*}
\begin{cases}
u_J(v_J, b_P) = 0 \ge u_J(b_J, v_P)\ \forall\ b_J , \ \ \ &b_P \ge v_{J}\\
u_J(v_J, b_P) = v_J - b_P \ge u_J(b_J, v_P)\ \forall\ b_J, \ \ \ &b_P < v_J
\end{cases}
\end{align*}
Thus, for John, the strategy to bid his valuation is indeed weakly dominant strategy.
Assume $v_J \ge v_P$ then the pair of strategies $(b_J, b_P) = (v_J, v_P)$ indeed constitutes a Nash equilibrium. Let us prove it. Denote $r_J(b_P), r_P(b_J)$ best responses on the opponent's bid of John and Peter respectively. Obviously:
\begin{align*}
r_J(b_P) &= \begin{cases}
[0, b_P), b_P > v_J\\
[0, b_P], b_p = v_J\\
(b_P, +\infty), b_P < v_J
\end{cases}\\
r_P(b_J) &= \begin{cases}
[0, b_J), b_J > v_P\\
[0, b_J], b_j = v_P\\
b_J + \varepsilon, b_J < v_P
\end{cases}
\end{align*}
Then 
\begin{align*}
\begin{cases}
r_J(v_P) = (v_P, +\infty) \ni v_J \text{ and } r_P(v_J) = [0, v_J) \ni v_P, \ \ \ &v_J > v_P\\
r_J(v_P) = [0, v_P] \ni v_J \text{ and } r_P(v_J) = [0, v_J] \ni v_P, \ \ \ &v_J = v_P
\end{cases}
\end{align*} 
Thus, for each player his strategy belongs to his best response on rival's playing this strategy, so, it is a Nash equilibrium.
If $v_J < v_P$ then $r_J(v_P) = (0, v_P) \ni v_J$ however $r_P(v_J) = v_J + \varepsilon < v_P$ (we assume $\varepsilon$ is sufficiently small). Hence for $v_J < v_P$, $(b_J, b_P) = (v_J, v_P)$ is not a Nash equilibrium.
\section*{Problem 2}
Prove the following statement
In a finite simultaneous move game representable in normal form,
\begin{align*}
\Gamma = (I, \times_{i \in I}S_i, (u_i)_{i \in I})
\end{align*}
if some pure strategy $s_i \in S_i$
is strictly dominated by some $\sigma_i \in \Delta(S_i)$, then
for any Nash Equilibrium of the game $\sigma^* \in \times_{i \in I}\Delta(A_i)$, we have $\sigma^*_i(s_i) = 0$
(that is, the equilibrium strategy of player $i$ assigns zero probability to pure
strategy $s_i$).
Hint: Proceed by contradiction. As a first step, show that $\sigma^*_i
(a_i) \neq 1$.


\textbf{Solution}

Assume $\sigma^* \in \times_{i \in I}\Delta(S_i)$ is a Nash equilibrium. And assume that for a player $i$, $\sigma_i^*$ assigns a positive probability to the pure strategy $s_j$ which is strictly dominated by the strategy $\sigma_i$. Then 
\begin{align*}
&\exists\ \alpha_1, \dots, \alpha_{|S_i|}: \alpha_k \in [0, 1]\ \forall\ k, \sum_{k=1}^{|S_i|}\alpha_k = 1\text{ and }\\
 u_i(\sigma^*_i, \sigma^*_{-i}) = &\alpha_1 u_1(s_1, \sigma^*_{-i}) + \dots + \alpha_j u_i(s_j, \sigma^*_{-i}) + \dots + \alpha_{|S_i|}u_i(s_{|S_i|}, \sigma^*_{-i}),\ \  \alpha_j > 0
\end{align*}
Since $\sigma_i$ strictly dominates $s_j$ then
\begin{align*}
&\exists\ \beta_1, \dots, \beta_{|S_i|}:\beta_k \in [0, 1]\ \forall\ k,\ \sum_{k=1}^{|S_i|}\beta_k = 1\text{ and }
\forall\ \sigma_{-i} \in \times_{l \in I\setminus \left\{i\right\}}\Delta(S_l)\\
u(s_j, \sigma_{-i}) < &u(\sigma_i, \sigma_{-i}) = \beta_1 u_i(s_1, \sigma_{-i}) + \dots + \beta_j u_i(s_j, \sigma_{-i}) + \dots + \beta_{|S_i|}u_i(s_{|S_i|}, \sigma_{-i})
\end{align*}
But then
\begin{align*}
&u_i(\sigma_i^*, \sigma^*_{-i}) < \alpha_1u_1(s_1, \sigma^*_{-i}) + \dots + \alpha_j \left(\sum_{k=1}^{|S_i|}\beta_k u_i(s_k, \sigma^*_{-i})\right) + \dots + \alpha_{|S_i|}u_i(s_{|S_i|}, \sigma^*_{-i}) = \\
&= (\alpha_1 + \alpha_j\beta_1)u_i(s_1, \sigma^*_{-i}) + \dots + (\alpha_j + \alpha_j\beta_j)u_i(s_1, \sigma^*_{-i}) + \dots + (\alpha_{|S_i|} + \alpha_{|S_i|}\beta_{|S_i|})u_i(s_{|S_i|}, \sigma^*_{-i}) = \\
&=u(\hat{\sigma_i}, \sigma^*_{-i})
\end{align*}
Since $\exists\ t \in 1, \dots, |S_i|: \beta_{t} > 0\ \to \alpha_t + \alpha_j \beta_t \neq \alpha_t$ hence there exists another strategy $\hat{\sigma_i} \neq \sigma^*_i$ which gives strictly greater payoff than $\sigma^*_i$ hence $\sigma^*$ is not a Nash equilibrium, contradiction. This contradiction shows that indeed $\alpha_j = 0$ i.e. the euilibrium strategy must assign $0$ probability to the strictly dominated strategy.
\section*{Problem 3}
There are $n < +\infty$ firms in an industry. Each can try to convince Congress
to give the industry a subsidy. Let $h_i$ denote the hours of effort put by
some industry $i \in \left\{1,\dots , n\right\}$. Let $c(h_i) = w_ih_i^2$, with $w_i > 0$, be the cost of
this effort to firm $i$. For each profile of efforts $(h_1, \dots , h_n) \in \mathbb{R}^n_+$, the value
of the subsidy that gets approved equals

\begin{align*}
\alpha \sum_{i \in \left\{1, \dots, n\right\}} h_i + \beta \left(\prod_{i \in \left\{1, \dots, n\right\}} h_i\right)
\end{align*}
where $\alpha, \beta \ge 0$. The approved subsidy is shared equally among the firms in
the industry. Suppose that all firms choose the effort level simultaneously.
Show that each firm has a strictly dominant strategy if and only if $\beta = 0$.
What is firm $i$’s strictly dominant strategy when this is so?


\textbf{Solution}


Each firm's payoff function is:
\begin{align*}
u_i(h_1, \dots, h_i, \dots, h_n) = \alpha \cdot \frac{h_1 + \dots h_i + \dots + h_n}{n} + \beta h_1\times\dots\times h_i \times \dots\times h_n - w_ih_i^2
\end{align*}

hence to find a strictly dominant strategy one need to find $h_i^*$:
\begin{align*}
u_i(h_i^*, h_{-i}) > u_i(h_i, h_{-i})\ \forall\ h_{-i}, \forall\ h_i
\end{align*}
Suppose $\beta = 0$ then 
\begin{align*}
u_i(h_i, h_{-i}) = \alpha\frac{h_i}{n} + \alpha\frac{h_1 + \dots+h_{i-1}+h_{i+1}+\dots+h_n}{n} - w_ih_i^2
\end{align*}
it is a quandratic function in $h_i$ and regardless of what $h_j\ \forall\ j \in\left\{1, \dots, n\right\}\setminus \left\{i\right\}$ is it attains maximum at $h_i = \frac{\alpha}{2nw_i}$. Thus, strategy $h_i = \frac{\alpha}{2nw_i}$ is a strictly dominant strategy for each firm in case $\beta = 0$. Now assume conversly that each firm has a strictly dominant strategy $h_i^*$. Then 
\begin{align*}
u_i(h_i^*, h_{-i}) > u_i(h_i, h_{-i})\ \  \forall\ h_i, h_{-i}
\end{align*}
Suppose by contradiction that $\beta > 0$. Let us denote $\prod_{k \in \left\{1, \dots, n\right\}\setminus\left\{i\right\}}h_k$ as $H_{-i}$ then if for some $h_{-i}$ - profile of strategies of other players
\begin{align*}
u_i(h_i^*, h_{-i}) > u_i(h_i, h_{-i})\ \forall\ h_i \neq h_{i}^*
\end{align*}
then $h_{i}^*$ is the extremum of the payoff function, given $H_{-i}$, i.e.
\begin{align*}
h_i^* = \frac{\alpha}{2nw_i} + \frac{\beta}{2nw_i}H_{-i}
\end{align*}
but now suppose $\forall j = 1, \dots, i-1, i+1, \dots, n$, $\hat{h_j} = \varepsilon^{\frac{1}{n-1}}h_j$ for some positive $\varepsilon$. Then
\begin{align*}
\exists\ \hat{h_i} = \frac{\alpha}{2nw_i} + \frac{\varepsilon \beta}{2nw_i}H_{-i} \neq h^*_i: u_i(\hat{h_i}, \hat{h_{-i}}) \ge u_i(h_i^*, \hat{h_{-i}})
\end{align*}
Because the value of the function at its maximum is always greater or equal than in any other point. That means that $h_i^*$ is not a strictly dominant strategy. This contradiction shows that $\beta = 0$. Q.E.D.
\section*{Problem 4}
Prove the following statement
In a finite simultaneous move game representable in normal form,
\begin{align*}
\Gamma =
(I, \times_{i \in I}S_i,(u_i)_{i\in I})
\end{align*}
if a single profile of pure strategies survives the process of iterated deletion
of strictly dominated strategies, then this profile must be the unique Nash
equilibrium of the game


\textbf{Solution}



\end{document}