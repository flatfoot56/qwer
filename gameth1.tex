\documentclass[a4paper]{article}
\usepackage[14pt]{extsizes} % 
\usepackage[utf8]{inputenc}
\usepackage{setspace,amsmath}
\usepackage{mathtools}
\usepackage{pgfplots}
\usepackage{titlesec}
\usepackage{pdfpages}
\usepackage[shortlabels]{enumitem}
\usepackage{tikz}
\usetikzlibrary{angles,quotes}
\usepackage{graphicx}
\usepackage{amssymb}
\usepackage{float}
\usepackage[section]{placeins}
\usepackage[makeroom]{cancel}
\usepackage{mathrsfs} % 
\newcommand\numberthis{\addtocounter{equation}{1}\tag{\theequation}}
%\addto\captionsrussian{\renewcommand{\figurename}{Fig.}}
\usepackage{amsmath,amsfonts,amssymb,amsthm,mathtools} 
\newcommand*{\hm}[1]{#1\nobreak\discretionary{}
{\hbox{$\mathsurround=0pt #1$}}{}}
\usepackage{graphicx}  % 
\graphicspath{{images/}{images2/}}  % 
\setlength\fboxsep{3pt} %  \fbox{} 
\setlength\fboxrule{1pt} % \fbox{}
\usepackage{wrapfig} % 
\newcommand{\prob}{\mathbb{P}}
\newcommand{\norma}{\mathscr{N}}
\newcommand{\expect}{\mathbb{E}}
\newcommand{\summa}{\sum_{i=1}^n}
\usepackage[left=7mm, top=20mm, right=15mm, bottom=20mm, nohead, footskip=10mm]{geometry} % 
\usepackage{tikz} % 
\def\myrad{2cm}% radius of the circle
\def\myanga{45}% angle for the arc
\def\myangb{195}
\begin{document} % 
	\begin{flushright}
	\begin{tabular}{r}
		Danil Fedchenko, MAE 2020, group A \\
	\end{tabular}
\end{flushright}


\begin{center}
	Game theory. Problem Set 1.
\end{center}
\section*{Problem 1}
John and Peter are bidding for an indivisible object. The rules of the
auction are as follows. The individual who places the highest bid wins the
object. If John wins he pays the amount of money Peter bade. If Peter
wins he pays the amount of money he bade. Both players place their bids
simultaneously. For each individual $i \in \left\{J, P\right\}$ let $v_i \ge 0$ denote their
valuation for the object and let $b_i \ge 0$ denote the bid they placed. Their
preferences are captured by the following payoff functions for John and
Peter respectively:
\begin{align*}
u_J(b_J, b_P) = \begin{cases}
v_J - b_P, &b_J > b_P\\
\frac{1}{2}(v_J - b_P), &b_J = b_P\\
0, &b_J < b_P
\end{cases}\\
u_P(b_J, b_P) = \begin{cases}
v_P - b_P, &b_P > b_J\\
\frac{1}{2}(v_P - b_P), &b_P = b_J\\
0, &b_P < b_J
\end{cases}
\end{align*}
The information above (including valuations) is common knowledge. Does
either player have a weakly dominant strategy to bid his valuation? Can
you find values for the parameters $(v_J , v_P ) \in \mathbb{R}^2_+$ such that the pair of
strategies $(b_J , b_P ) = (v_J , v_P)$ constitutes a Nash Equilibrium? For each
answer you give provide the relevant demonstration.


\textbf{Solution}


Suppose that Peter's strategy is to bid his valuation $v_P$. Then
\begin{align*}
u_P(b_J, v_P) = 0 < u_P\left(b_J, \frac{b_J + v_P}{2}\right) = \frac{v_P - b_J}{2}, \text{for }b_J < v_P\\
\end{align*}
hence such a strategy cannot be (weakly) dominant for Peter.
Suppose that John's strategy is to bid his valuation $v_J$. Then
\begin{align*}
\begin{cases}
u_J(v_J, b_P) = 0 \ge u_J(b_J, v_P)\ \forall\ b_J , \ \ \ &b_P \ge v_{J}\\
u_J(v_J, b_P) = v_J - b_P \ge u_J(b_J, v_P)\ \forall\ b_J, \ \ \ &b_P < v_J
\end{cases}
\end{align*}
Thus, for John, the strategy to bid his valuation is indeed weakly dominant strategy.
Assume $v_J \ge v_P$ then the pair of strategies $(b_J, b_P) = (v_J, v_P)$ indeed constitutes a Nash equilibrium. Let us prove it. Denote $r_J(b_P), r_P(b_J)$ best responses on the opponent's bid of John and Peter respectively. Obviously:
\begin{align*}
r_J(b_P) &= \begin{cases}
[0, b_P), b_P > v_J\\
[0, b_P], b_p = v_J\\
(b_P, +\infty), b_P < v_J
\end{cases}\\
r_P(b_J) &= \begin{cases}
[0, b_J), b_J > v_P\\
[0, b_J], b_j = v_P\\
b_J + \varepsilon, b_J < v_P
\end{cases}
\end{align*}
Then 
\begin{align*}
\begin{cases}
r_J(v_P) = (v_P, +\infty) \ni v_J \text{ and } r_P(v_J) = [0, v_J) \ni v_P, \ \ \ &v_J > v_P\\
r_J(v_P) = [0, v_P] \ni v_J \text{ and } r_P(v_J) = [0, v_J] \ni v_P, \ \ \ &v_J = v_P
\end{cases}
\end{align*} 
Thus, for each player his strategy belongs to his best response on rival's playing this strategy, so, it is a Nash equilibrium.
If $v_J < v_P$ then $r_J(v_P) = (0, v_P) \ni v_J$ however $r_P(v_J) = v_J + \varepsilon < v_P$ (we assume $\varepsilon$ is sufficiently small). Hence for $v_J < v_P$, $(b_J, b_P) = (v_J, v_P)$ is not a Nash equilibrium.
\section*{Problem 2}
Prove the following statement
In a finite simultaneous move game representable in normal form,
\begin{align*}
\Gamma = (I, \times_{i \in I}S_i, (u_i)_{i \in I})
\end{align*}
if some pure strategy $s_i \in S_i$
is strictly dominated by some $\sigma_i \in \Delta(S_i)$, then
for any Nash Equilibrium of the game $\sigma^* \in \times_{i \in I}\Delta(A_i)$, we have $\sigma^*_i(s_i) = 0$
(that is, the equilibrium strategy of player $i$ assigns zero probability to pure
strategy $s_i$).
Hint: Proceed by contradiction. As a first step, show that $\sigma^*_i
(a_i) \neq 1$.


\textbf{Solution}

Assume $\sigma^* \in \times_{i \in I}\Delta(S_i)$ is a Nash equilibrium. And assume that for a player $i$, $\sigma_i^*$ assigns a positive probability to the pure strategy $s_j$ which is strictly dominated by the strategy $\sigma_i$. Then 
\begin{align*}
&\exists\ \alpha_1, \dots, \alpha_{|S_i|}: \alpha_k \in [0, 1]\ \forall\ k, \sum_{k=1}^{|S_i|}\alpha_k = 1\text{ and }\\
 u_i(\sigma^*_i, \sigma^*_{-i}) = &\alpha_1 u_1(s_1, \sigma^*_{-i}) + \dots + \alpha_j u_i(s_j, \sigma^*_{-i}) + \dots + \alpha_{|S_i|}u_i(s_{|S_i|}, \sigma^*_{-i}),\ \  \alpha_j > 0
\end{align*}
Since $\sigma_i$ strictly dominates $s_j$ then
\begin{align*}
&\exists\ \beta_1, \dots, \beta_{|S_i|}:\beta_k \in [0, 1]\ \forall\ k,\ \sum_{k=1}^{|S_i|}\beta_k = 1\text{ and }
\forall\ \sigma_{-i} \in \times_{l \in I\setminus \left\{i\right\}}\Delta(S_l)\\
u(s_j, \sigma_{-i}) < &u(\sigma_i, \sigma_{-i}) = \beta_1 u_i(s_1, \sigma_{-i}) + \dots + \beta_j u_i(s_j, \sigma_{-i}) + \dots + \beta_{|S_i|}u_i(s_{|S_i|}, \sigma_{-i})
\end{align*}
But then
\begin{align*}
&u_i(\sigma_i^*, \sigma^*_{-i}) < \alpha_1u_1(s_1, \sigma^*_{-i}) + \dots + \alpha_j \left(\sum_{k=1}^{|S_i|}\beta_k u_i(s_k, \sigma^*_{-i})\right) + \dots + \alpha_{|S_i|}u_i(s_{|S_i|}, \sigma^*_{-i}) = \\
&= (\alpha_1 + \alpha_j\beta_1)u_i(s_1, \sigma^*_{-i}) + \dots + (\alpha_j + \alpha_j\beta_j)u_i(s_1, \sigma^*_{-i}) + \dots + (\alpha_{|S_i|} + \alpha_{|S_i|}\beta_{|S_i|})u_i(s_{|S_i|}, \sigma^*_{-i}) = \\
&=u(\hat{\sigma_i}, \sigma^*_{-i})
\end{align*}
Since $\exists\ t \in 1, \dots, |S_i|: \beta_{t} > 0\ \to \alpha_t + \alpha_j \beta_t \neq \alpha_t$ hence there exists another strategy $\hat{\sigma_i} \neq \sigma^*_i$ which gives strictly greater payoff than $\sigma^*_i$ hence $\sigma^*$ is not a Nash equilibrium, contradiction. This contradiction shows that indeed $\alpha_j = 0$ i.e. the euilibrium strategy must assign $0$ probability to the strictly dominated strategy.
\end{document}