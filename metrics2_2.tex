\documentclass[a4paper]{article}
\usepackage[14pt]{extsizes} % 
\usepackage[utf8]{inputenc}
\usepackage{setspace,amsmath}
\usepackage{mathtools}
\usepackage{pgfplots}
\usepackage{titlesec}
\usepackage{pdfpages}
\usepackage[shortlabels]{enumitem}
\usepackage{tikz}
\usetikzlibrary{angles,quotes}
\usepackage{graphicx}
\usepackage{amssymb}
\usepackage{float}
\usepackage[section]{placeins}
\usepackage[makeroom]{cancel}
\usepackage{mathrsfs} % 
\newcommand\numberthis{\addtocounter{equation}{1}\tag{\theequation}}
%\addto\captionsrussian{\renewcommand{\figurename}{Fig.}}
\usepackage{amsmath,amsfonts,amssymb,amsthm,mathtools} 
\newcommand*{\hm}[1]{#1\nobreak\discretionary{}
{\hbox{$\mathsurround=0pt #1$}}{}}
\usepackage{graphicx}  % 
\graphicspath{{images/}{images2/}}  % 
\setlength\fboxsep{3pt} %  \fbox{} 
\setlength\fboxrule{1pt} % \fbox{}
\usepackage{wrapfig} % 
\newcommand{\prob}{\mathbb{P}}
\newcommand{\norma}{\mathscr{N}}
\newcommand{\expect}{\mathbb{E}}
\newcommand{\summa}{\sum_{i=1}^n}
\newcommand{\yrseduc}{\textit{yrseduc}}
\usepackage[left=7mm, top=20mm, right=15mm, bottom=20mm, nohead, footskip=10mm]{geometry} % 
\usepackage{tikz} % 
\def\myrad{2cm}% radius of the circle
\def\myanga{45}% angle for the arc
\def\myangb{195}
\begin{document} % 
	\begin{flushright}
	\begin{tabular}{r}
		Danil Fedchenko, MAE 2020, group A \\
	\end{tabular}
\end{flushright}


\begin{center}
	Econometrics 2. Problem Set 2.
\end{center}
\section*{Problem 1}
Let $\{y_t: t = 1, 2,\dots\}$ follow a random walk, as in (11.20), with $y_0 = 0$. Show that $Corr(y_t, y_{t+h}) =\sqrt{\frac{t}{t + h}}$   for $t \ge 1, h > 0$.

\textbf{Solution}

\begin{align*}
y_t = y_{t-1} + e_t = y_{t-2} + e_{t-1} + e_t = \dots = y_0 + e_1 + \dots + e_t\\
y_{t+h} = y_{t+h-1} + e_{t+h} = \dots = y_0 + e_1 + \dots + e_{t} + \dots + e_{t+h}
\end{align*}
\begin{align*}
&Cov(y_t, y_{t+h}) = (y_0 + e_1 + \dots + e_t, y_0 + e_1 + \dots + e_{t} + \dots + e_{t+h}) \underset{e_t \text{ i.i.d. and indep. of }y_0}{=} t\sigma^2_e\\
&Var(y_t) = Var(y_0 + e_1 + \dots + e_t)\underset{e_t \text{ i.i.d. and indep. of }y_0}{=} t\sigma^2_e\\
&Var(y_{t+h}) = Var(y_0 + e_1 + \dots + e_{t} + \dots + e_{t+h}) \underset{e_t \text{ i.i.d. and indep. of }y_0}{=} (t+h) \sigma^2_e\\
\end{align*}
\begin{align*}
Corr(y_t, y_{t+h}) = \frac{t\sigma^2_e}{\sqrt{t(t+h)}\sigma^2_e} = \sqrt{\frac{t}{t+h}}
\end{align*}
\section*{Problem 2}
A partial adjustment model is
\begin{align*}
y_t^* = \gamma_0 + \gamma_1 x_t + e_t\\
y_t - y_{t-1} = \lambda(y_t^* - y_{t-1}) + a_t, 
\end{align*}
where $y_t^*$ is the desired or optimal level of $y$, and $y_t$ is the actual (observed) level. For example, $y_t^*$ is the desired growth in firm inventories, and $x_t$ is growth in firm sales. The parameter $\gamma_1$ measures the effect of $x_t$ on $y_t^*$. The second equation describes how the actual $y$ adjusts depending on the relationship between the desired $y$ in time $t$ and the actual $y$ in time $t-1$. The parameter $\lambda$ measures the speed of adjustment and satisfies $0 < \lambda <1$.

\begin{enumerate}[(i)]
	\item  Plug the first equation for $y_t^*$ into the second equation and show that we can write
	\begin{align*}
	y_t = \beta_0 = \beta_1 y_{t-1} + \beta_2 x_{t}+u_t.
	\end{align*}
	In particular, find the $\beta_j$ in terms of the $\gamma_j$ and $\lambda$ and find $u_t$ in terms of $e_t$ and $a_t$. Therefore, the partial adjustment model leads to a model with a lagged dependent variable and a contemporaneous $x$. 
	\item  If $\expect(e_t|x_t, y_{t-1}, x_{t-1}, \dots) = \expect(a_t|x_t, y_{t-1}, x_{t-1},\dots) = 0$ and all series are weakly dependent, how would you estimate the $\beta_j$? 
	\item If $\hat{\beta}_1 = 0.7$ and $\hat{\beta}_2=0.2$, what are the estimates of $\gamma_1$ and $\lambda$?
\end{enumerate}


\textbf{Solution}

\begin{enumerate}[(i)]
	\item \begin{align*}
	y_t - y_{t-1} = \lambda(\gamma_0 + \gamma_1 x_t + e_t - y_{t-1}) + a_t\\
	y_t = \lambda \gamma_0 + (1 - \lambda)y_{t-1} + \lambda \gamma_1 x_t + \lambda e_t + a_t\\
	\beta_0 = \lambda \gamma_0,\ \beta_1 = 1 - \lambda,\ \beta_2 = \lambda \gamma_1,\ u_t = \lambda e_t + a_t
	\end{align*}
	\item Since
	\begin{align*}
	\expect[u_t|x_t, y_{t-1}, x_{t-1}, \dots] = 0
	\end{align*}
	and series are weakly dependent then we can consistently estimate $\beta_j$ by OLS regressing $y_t$ on $y_{t-1}, x_{t}$.
	\item \begin{align*}
	\hat{\lambda} = 1 - \hat{\beta}_1 = 0.3\\
	\hat{\gamma}_1 = \frac{\hat{\beta}_2}{\hat{\lambda}} \approx 0.67
	\end{align*}
\end{enumerate}


\section*{Problem 3}
Suppose that the equation 
\begin{align*}
y_t = \alpha + \delta _t + \beta_1 x_{t1}+\dots+\beta_k x_{tk}+u_t
\end{align*} satisfies the sequential exogeneity assumption in equation (11.40). 
\begin{enumerate}[(i)]
\item Suppose you difference the equation to obtain 
\begin{align}\label{eq1}
\Delta y_t = \delta +\beta_1 \Delta x_{t1} + \dots+ \beta_k \Delta x_{tk} + \Delta u_t.
\end{align}
How come applying OLS on the differenced equation does not generally result in consistent estimators of the $\beta_j$?
\item What assumption on the explanatory variables in the original equation would ensure that OLS on the differences consistently estimates the $\beta_j$?
\item Let $z_{t1},\dots, z_{tk}$ be a set of explanatory variables dated contemporaneously with $y_t$. If we specify the static regression model $y_t = \beta_0 + \beta_1z_{t1} +\dots+ \beta_k z_{tk} + u_t$, describe what we need to assume for $x_t = z_t$ to be sequentially exogenous. Do you think the assumptions are likely to hold in economic applications?
\end{enumerate}

\textbf{Solution}

\begin{enumerate}
	\item First of all, since we do not have any assumptions about the model apart from sequential exogeneity
	\begin{align*}
	\expect[u_t|\vec{x}_t, \vec{x}_{t-1}, \dots] = \expect[u_t] = 0
	\end{align*}
	we can have a perfect multicollinearity among $\vec{x}_t$. Assume for example $k = 3$, and the model is:
	\begin{align*}
	y_t =\alpha + \delta t + \beta_1 x_t + \beta_2 z_t + \beta_3 w_t + u_t
	\end{align*} 
	where $w_t = x_t + z_t\ \forall\ t$. In this case obviously $\Delta w_t = \Delta x_t + \Delta z_t$ and the OLS estimate for \eqref{eq1} is inconsistent.
	\item To ensure that OLS for \eqref{eq1} is consistent. Assumptions TS1'-TS3' should hold. The process \eqref{eq1} should be weakly dependent (and stationary). Of course, if $h_t = \beta_1 x_{t1} + \dots + \beta_k x_{tk} + u_t$ is weakly dependent and stationary then the process $\Delta h_t = \beta_1 \Delta x_{t1} + \dots + \beta_k \Delta x_{tk} + \Delta u_t$ (and hence the process $\Delta y_t$) will be weakly dependent and stationary (though this condition is not necessary, e.g. random walk process is not a weakly dependent while the differenced process is). Moreover, $x_{t1}, \dots, x_{tk}$ should not be linearly dependent (then $\Delta x_{t1}, \dots, \Delta x_{tk}$ will not linearly dependent too). In addition, contemporaneous exogeneity assumption should hold, i.e.
	\begin{align*}
	\expect[\Delta u_t| \Delta x_{t1}, \dots, \Delta x_{tk}] = 0
	\end{align*} 
	\begin{align*}
	\expect[\Delta u_t|\Delta x_{t1}, \dots, \Delta x_{tk}] = \expect[u_{t}|\Delta x_{t1}, \dots, \Delta x_{tk}] - \expect[u_{t-1}|\Delta x_{t1}, \dots, \Delta x_{tk}] = \\
	=\expect[\expect[u_t|\vec{x}_{t}, \vec{x}_{t-1}, \dots]|\Delta x_{t1}, \dots, \Delta x_{tk}] - \expect[u_{t-1}|\Delta x_{t1}, \dots, \Delta x_{tk}] = \\
	=0 - \expect[u_{t-1}|\Delta x_{t1}, \dots, \Delta x_{tk}]
	\end{align*}
	hence to ensure that TS3' holds the following condition should hold:
	\begin{align*}
	\expect[u_{t-1}|x_{t,1} - x_{t-1, 1}, \dots, x_{t, k} - x_{t-1, k}] = 0
	\end{align*}
	\item If we assume that 
	\begin{align*}
	\expect[y_t|\vec{x}_t, \vec{x}_{t-1}, \dots] = \vec{x}_t^T\vec{\beta}
	\end{align*}
	then
	\begin{align*}
	\expect[u_t|\vec{x}_t, \vec{x}_{t-1}, \dots] = 0
	\end{align*}
	that is, if we assume that only contemporaneously dated (with $y_t$) variables "explain" $y_t$ (i.e. the average of $y_t$ can be predicted based only on contemporaneously dated variables) then the model is sequentially exogenous. This is not a plausible assumption for economic applications.
\end{enumerate}
	\section*{Problem 4}
	\begin{enumerate}[(i)]
		\item The result of regression $\Delta inf$ on $\Delta unem$ is shown in the table below
		
		\begin{center}
		\begin{tabular}{lc} \hline
			& (1) \\
			VARIABLES & cinf \\ \hline
			&  \\
			cunem & -0.842** \\
			& (0.314) \\
			Constant & -0.0782 \\
			& (0.348) \\
			&  \\
			Observations & 48 \\
			R-squared & 0.135 \\ \hline
			\multicolumn{2}{c}{ Standard errors in parentheses} \\
			\multicolumn{2}{c}{ *** p$<$0.01, ** p$<$0.05, * p$<$0.1} \\
		\end{tabular}
	\end{center}
	We have obtained statistically significant estimate of $\beta_1$. It has a negative sign, that is there is a short-run trade-off between unemployment and inflation: a 1 percent point raise in unemployment leads to on average about 0.8 fall in inflation.
	\item $R^2$ of model in (i) is higher than that of model 11.19, hence it fits the data better.
	\end{enumerate}
\end{document}