\documentclass[a4paper]{article}
\usepackage[14pt]{extsizes} % 
\usepackage[utf8]{inputenc}
\usepackage{setspace,amsmath}
\usepackage{mathtools}
\usepackage{pgfplots}
\usepackage{titlesec}
\usepackage{pdfpages}
\usepackage[shortlabels]{enumitem}
\usepackage{tikz}
\usetikzlibrary{angles,quotes}
\usepackage{graphicx}
\usepackage{amssymb}
\usepackage{float}
\usepackage[section]{placeins}
\usepackage[makeroom]{cancel}
\usepackage{mathrsfs} % 
\newcommand\numberthis{\addtocounter{equation}{1}\tag{\theequation}}
%\addto\captionsrussian{\renewcommand{\figurename}{Fig.}}
\usepackage{amsmath,amsfonts,amssymb,amsthm,mathtools} 
\newcommand*{\hm}[1]{#1\nobreak\discretionary{}
{\hbox{$\mathsurround=0pt #1$}}{}}
\usepackage{graphicx}  % 
\graphicspath{{images/}{images2/}}  % 
\setlength\fboxsep{3pt} %  \fbox{} 
\setlength\fboxrule{1pt} % \fbox{}
\usepackage{wrapfig} % 
\newcommand{\prob}{\mathbb{P}}
\newcommand{\norma}{\mathscr{N}}
\newcommand{\expect}{\mathbb{E}}
\newcommand{\summa}{\sum_{i=1}^n}
\newcommand{\yrseduc}{\textit{yrseduc}}
\usepackage[left=7mm, top=20mm, right=15mm, bottom=20mm, nohead, footskip=10mm]{geometry} % 
\usepackage{tikz} % 
\def\myrad{2cm}% radius of the circle
\def\myanga{45}% angle for the arc
\def\myangb{195}
\begin{document} % 
	\begin{flushright}
	\begin{tabular}{r}
		Danil Fedchenko, MAE 2020, group A \\
	\end{tabular}
\end{flushright}


\begin{center}
	Econometrics 2. Problem Set 2.
\end{center}
\section*{Problem 1}
Let $\{y_t: t = 1, 2,\dots\}$ follow a random walk, as in (11.20), with $y_0 = 0$. Show that $Corr(y_t, y_{t+h}) =\sqrt{\frac{t}{t + h}}$   for $t \ge 1, h > 0$.

\textbf{Solution}

\begin{align*}
y_t = y_{t-1} + e_t = y_{t-2} + e_{t-1} + e_t = \dots = y_0 + e_1 + \dots + e_t\\
y_{t+h} = y_{t+h-1} + e_{t+h} = \dots = y_0 + e_1 + \dots + e_{t} + \dots + e_{t+h}
\end{align*}
\begin{align*}
&Cov(y_t, y_{t+h}) = (y_0 + e_1 + \dots + e_t, y_0 + e_1 + \dots + e_{t} + \dots + e_{t+h}) \underset{e_t \text{ i.i.d. and indep. of }y_0}{=} t\sigma^2_e\\
&Var(y_t) = Var(y_0 + e_1 + \dots + e_t)\underset{e_t \text{ i.i.d. and indep. of }y_0}{=} t\sigma^2_e\\
&Var(y_{t+h}) = Var(y_0 + e_1 + \dots + e_{t} + \dots + e_{t+h}) \underset{e_t \text{ i.i.d. and indep. of }y_0}{=} (t+h) \sigma^2_e\\
\end{align*}
\begin{align*}
Corr(y_t, y_{t+h}) = \frac{t\sigma^2_e}{\sqrt{t(t+h)}\sigma^2_e} = \sqrt{\frac{t}{t+h}}
\end{align*}
\section*{Problem 2}
A partial adjustment model is
\begin{align*}
y_t^* = \gamma_0 + \gamma_1 x_t + e_t\\
y_t - y_{t-1} = \lambda(y_t^* - y_{t-1}) + a_t, 
\end{align*}
where $y_t^*$ is the desired or optimal level of $y$, and $y_t$ is the actual (observed) level. For example, $y_t^*$ is the desired growth in firm inventories, and $x_t$ is growth in firm sales. The parameter $\gamma_1$ measures the effect of $x_t$ on $y_t^*$. The second equation describes how the actual $y$ adjusts depending on the relationship between the desired $y$ in time $t$ and the actual $y$ in time $t-1$. The parameter $\lambda$ measures the speed of adjustment and satisfies $0 < \lambda <1$.

\begin{enumerate}[(i)]
	\item  Plug the first equation for $y_t^*$ into the second equation and show that we can write
	\begin{align*}
	y_t = \beta_0 = \beta_1 y_{t-1} + \beta_2 x_{t}+u_t.
	\end{align*}
	In particular, find the $\beta_j$ in terms of the $\gamma_j$ and $\lambda$ and find $u_t$ in terms of $e_t$ and $a_t$. Therefore, the partial adjustment model leads to a model with a lagged dependent variable and a contemporaneous $x$. 
	\item  If $\expect(e_t|x_t, y_{t-1}, x_{t-1}, \dots) = \expect(a_t|x_t, y_{t-1}, x_{t-1},\dots) = 0$ and all series are weakly dependent, how would you estimate the $\beta_j$? 
	\item If $\hat{\beta}_1 = 0.7$ and $\hat{\beta}_2=0.2$, what are the estimates of $\gamma_1$ and $\lambda$?
\end{enumerate}


\textbf{Solution}

\begin{enumerate}[(i)]
	\item \begin{align*}
	y_t - y_{t-1} = \lambda(\gamma_0 + \gamma_1 x_t + e_t - y_{t-1}) + a_t\\
	y_t = \lambda \gamma_0 + (1 - \lambda)y_{t-1} + \lambda \gamma_1 x_t + \lambda e_t + a_t\\
	\beta_0 = \lambda \gamma_0,\ \beta_1 = 1 - \lambda,\ \beta_2 = \lambda \gamma_1,\ u_t = \lambda e_t + a_t
	\end{align*}
	\item Since
	\begin{align*}
	\expect[u_t|x_t, y_{t-1}, x_{t-1}, \dots] = 0
	\end{align*}
	and series are weakly dependent then we can consistently estimate $\beta_j$ by OLS regressing $y_t$ on $y_{t-1}, x_{t}$.
	\item \begin{align*}
	\hat{\lambda} = 1 - \hat{\beta}_1 = 0.3\\
	\hat{\gamma}_1 = \frac{\hat{\beta}_2}{\hat{\lambda}} \approx 0.67
	\end{align*}
\end{enumerate}
\end{document}