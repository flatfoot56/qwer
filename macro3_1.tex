\documentclass[a4paper]{article}
\usepackage[14pt]{extsizes} % 
\usepackage[utf8]{inputenc}
\usepackage{setspace,amsmath}
\usepackage{mathtools}
\usepackage{pgfplots}
\usepackage{titlesec}
\usepackage{pdfpages}
\usepackage[shortlabels]{enumitem}
\usepackage{tikz}
\usetikzlibrary{angles,quotes}
\usepackage{graphicx}
\usepackage{amssymb}
\usepackage{float}
\usepackage[section]{placeins}
\usepackage[makeroom]{cancel}
\usepackage{mathrsfs} % 
\newcommand\numberthis{\addtocounter{equation}{1}\tag{\theequation}}
%\addto\captionsrussian{\renewcommand{\figurename}{Fig.}}
\usepackage{amsmath,amsfonts,amssymb,amsthm,mathtools} 
\newcommand*{\hm}[1]{#1\nobreak\discretionary{}
{\hbox{$\mathsurround=0pt #1$}}{}}
\usepackage{graphicx}  % 
\graphicspath{{images/}{images2/}}  % 
\setlength\fboxsep{3pt} %  \fbox{} 
\setlength\fboxrule{1pt} % \fbox{}
\usepackage{wrapfig} % 
\newcommand{\prob}{\mathbb{P}}
\newcommand{\norma}{\mathscr{N}}
\newcommand{\expect}{\mathbb{E}}
\newcommand{\summa}{\sum_{i=1}^n}
\usepackage[left=7mm, top=20mm, right=15mm, bottom=20mm, nohead, footskip=10mm]{geometry} % 
\usepackage{tikz} % 
\def\myrad{2cm}% radius of the circle
\def\myanga{45}% angle for the arc
\def\myangb{195}
\begin{document} % 
	\begin{flushright}
	\begin{tabular}{r}
		Danil Fedchenko, MAE 2020, group A \\
	\end{tabular}
\end{flushright}


\begin{center}
	Macroeconomics 3. Problem Set 1.
\end{center}
\section*{Problem 1}
Consider the following model
\begin{align}\label{eq1}
\underset{\left\{k(t+1), c(t)\right\}_{t=0}^{\infty}}{\max}\ \sum_{t=0}^{\infty} \beta^t u(c(t))\nonumber\\
s.t.\ k(t+1) = k(t) - c(t) \\
c(t) \ge 0, k(t) \ge 0, k(0) > 0\ \text{is given}\nonumber
\end{align}
\begin{enumerate}
	\item Rewrite it recursively. Replace $\beta$ with $\frac{1}{1+\rho}$
	and derive the asset pricing equations for the whole
	cake and for the marginal piece of cake. Explain what are the “dividends” and the “capital gains”
	in each case.
	\item Does the value of the marginal piece of cake increase or decrease over time? Use the asset pricing
	equation and explain the intuition.
	\item Does the value of the whole cake increase or decrease over time? Explain the intuition.
	\item At which rate does the “current value” of the marginal piece of cake change over time? What
	about the “present value”? Use the optimality conditions and explain the intuition.
	\item Assume CRRA utility function,
	\begin{align*}
	u(c(t)) = \frac{c(t)^{1 - \sigma}}{1 - \sigma}, \sigma > 0
	\end{align*}
	Derive the Euler equation. Explain the intuition of how the consumption path depends on the
	intertemporal elasticity of substitution.

	\item Now assume that the cake goes bad as the time goes on, that is the constraint \eqref{eq1} changes to
	\begin{align*}
	k (t + 1) = (1 - \delta) k (t) - c (t)
	\end{align*}
	where $0 < \delta < 1$. Explain how does it change your answers to the previous questions.
	\item Assume $\sigma = 1$ and $\delta = 0$. Find the analytical solutions for the value and policy functions.
\end{enumerate}


\textbf{Solution}

\begin{enumerate}
	\item Let $V(k)$ be the value of the agent who acts optimally and possesses $k$ units of capital now. Then Bellman equation is:
	\begin{align*}
	V(k) = \underset{0 \le c \le k}{\max}\left\{u(c) + \beta V(k - c)\right\} \\
	V(k) = \underset{0 \le c \le k}{\max}\left\{u(c) + \frac{1}{1+\rho} V(k - c)\right\} 
	\end{align*}
	Assume $c^*$ is the optimizer, then
	\begin{align*}
	V(k) = u(c^*) + \frac{1}{1+\rho}V(k-c^*)\\
	\rho = \frac{V(k - c^*) - V(k) + u(c^*)}{V(k) - u(c^*)}
	\end{align*}
	or alternatively:
	\begin{align*}
	\rho = \frac{u(c^*(t)) + V(k^*(t+1)) - V(k^*(t))}{V(k^*(t)) - u(c^*(t))}
	\end{align*}
	(* means optimal trajectory).
	Here $\frac{V(k(t+1)) - V(k(t))}{V(k(t))}$ are "capital gains" and $u(c^*)$ are "dividends". 
	
	
	By the envelope theorem
	\begin{align*}
	V'(k) = \frac{1}{1+\rho}V'(k-c^*)\\
	\rho = \frac{V'(k-c^*) - V'(k)}{V'(k)}
	\end{align*}
	or alternatively:
	\begin{align*}
	\rho = \frac{V'(k(t+1)) - V'(k(t))}{V'(k(t))}
	\end{align*}
	as one can observe there are no marginal dividends (because there are no production) only $\frac{V'(k(t+1)) - V'(k(t))}{V'(k(t))}$ which describes percentage change of shadow price of capital i.e. marginal "capital gains". 
	\item Of course in general the answer to this question depends on the particular type of function $u()$, however if we assume "good" increasing and concave utility function then price of marginal piece of cake increases over time. From the Euler's equation (which is obtained in 5.) $\frac{u'(c(t))}{u'(c(t+1))} = \frac{\lambda(t)}{\lambda(t+1)} = \beta$. Since $\lambda(t) = V'(k(t))$ then $\frac{V'(k(t))}{V'(k(t+1))} = \beta < 1$. That means that $V'(k(t))$ is increasing over time. Since $\rho$ is a constant (its meaning is the interest rate) then to hold it constant with increasing denumerator, the numerator $V'(k(t+1)) - V'(k(t))$ should also increase but $V'(k(t+1)) - V'(k(t))$ is exactly the price of marginal piece of cake. Intuitivelly it can be understood as follows: the amount of capital is shrinking hence over time the agent values each marginal unit of capital more and more. 
	\item Using the same equations as in the 2. $\frac{V'(k(t))}{V'(k(t+1))} = \beta$, that means that the present value of the whole cake stay constant: \begin{align*}
	V'(k(t)) = \frac{1}{\beta}V'(k(t-1)) &= \frac{1}{\beta^2}V'(k(t-2)) = \dots = \frac{1}{\beta^t}V'(k(0))\\
	\end{align*}
	But the present value of $V'(k(t))$ is $\beta^t V'(k(t))$ hence 
	\begin{align*}
	\forall\ t\ \beta^t V'(k(t)) = V'(k(0))
	\end{align*}
	In the same time the current value is decreasing, which can be inferred from $\frac{V'(k(t))}{V'(k(t+1))} = \beta < 1$.
	\item Present value is constant, current is increasing with rate $\frac{1}{\beta}$.
	\item Assume 
	\begin{align*}
	u(c(t)) = \frac{c(t)^{1 - \sigma}}{1 - \sigma}, \sigma > 0
\end{align*}
then
\begin{align*}
&L = \sum_{t=0}^{\infty} \beta^t(u(c(t)) + \lambda(t)(k(t) - c(t) - k(t+1)))\\
&\begin{cases}\frac{\partial}{\partial c(t)}: u'(c(t)) - \lambda(t) = 0\\
\frac{\partial}{\partial k(t+1)}: -\beta^t \lambda(t) + \beta^{t+1}\lambda(t+1) = 0\\
\end{cases}\\
&\lambda(t) = \beta \lambda(t+1)\\
&u'(c(t+1)) = \lambda(t+1)\\
&\frac{u'(c(t))}{u'(c(t+1))} = \beta\\
\end{align*}
Thus, the Euler's equation is:
\begin{align*}
\left(\frac{c(t+1)}{c(t)}\right)^{\sigma} = \beta\\
\frac{c(t+1)}{c(t)} = \beta^{\frac{1}{\sigma}}
\end{align*}
intertemporal elasticity of substitution is $\frac{1}{\sigma}$. If it is equal to 0 then any changes of interest rate have no effect on consumption growth. If the intertemporal elasticity becomes larger than 0, then the ratio $\frac{c(t+1)}{c(t)}$ becomes low, that is the price of consumption 
\item 
\item Assume $\sigma = 1, \delta = 0$ then
from Euler's equation:
\begin{align*}
c(t+1) = \beta c(t)
\end{align*}
it remains to find $c(0)$,
\begin{align*}
&k(t) = k(t+1) + c(t) = k(t+2) + c(t+1) + c(t) = k(t+2) + \beta c(t) + c(t) = \\
&=k(t+3) + c(t+2) + \beta c(t) + c(t) = k(t+3) + \beta^2 c(t) + \beta c(t) + c(t) = \dots =\\
&=k(t + T) + c(t) \sum_{\tau = 0}^{T-1} \beta^{\tau} \\
&\text{since the amount of cake cannot be negative hence for optimal consumption path } k\\
&\text{ should tend to 0 at infinity, }\\
&k(t + T) + c(t) \sum_{\tau = 0}^{T-1} \beta^{\tau} \underset{T \to \infty}{\to} \frac{c(t)}{1-\beta} \to c(0) = (1-\beta)k(0)
\end{align*}
Thus, optimal consumption path is:
\begin{align*}
c(t+1) = \beta c(t), c(0) = (1-\beta)k(0)
\end{align*}
capital:
\begin{align*}
k(t+1) = k(t) - c(t)
\end{align*}
and the value function is:
\begin{align*}
\sum_{t=0}^{\infty} \beta^t \ln c(t) = \sum_{t=0}^{\infty} \beta^{t} \ln (\beta^t c(0)) = \sum_{t=0}^{\infty} \beta^t t \ln \beta + \sum_{t=0}^{\infty} \beta^t \ln c(0) = \frac{\beta \ln \beta}{(1-\beta)^2} + \frac{\ln c(0)}{1 - \beta } = \\
=\frac{\beta \ln \beta}{(1-\beta)^2} + \frac{\ln((1-\beta)k(0))}{1-\beta}
\end{align*}
\end{enumerate}

\section*{Problem 2}
Problem:
\begin{align*}
\underset{\left\{k(t+1), c(t)\right\}_{t=1}^{\infty}}{\max}\ \sum_{t=1}^{\infty} 0.96^t \ln c(t)\\
\text{s.t.\ } k(t+1) = Ak(t)^{1/3} + (1-0.02)k(t) - c(t)\\
c(t) \ge 0, k(t) \ge 0, k(1) = 0.1
\end{align*}
Euler's equation is:
\begin{align*}
\frac{c(t+1)}{\beta c(t)} = \frac{A}{3k(t+1)^{2/3}} + 1 - \delta
\end{align*}
In the steady state $c(t+1) = c(t)$ hence:
\begin{align*}
\frac{1}{\beta} = \frac{A}{3(k^*)^{2/3}} + 1 - \delta\\
\text{if } k^* = 1 \to A = 3(\frac{1}{\beta} + \delta - 1) = 0.185
\end{align*}
Our initial guess for $k(2)$ should lie within the interval $ [k(1), f(k(1)) + (1 - \delta)k(1)]$ because even if we assume $c(1) = 0$ we will not be able to increase capital $k(2)$ more than $f(k(1)) + (1 - \delta)k(1)$ and moreover since $k^* = 1$ while $k(1) = 0.1$ by the theorem optimal capital growth $s(k(t))$ is non-decreasing in $k(t)$ hence $k(2)$ cannot be less than $k(1)$. The global result which follows is that $k(t)$ cannot be greater than $k^*$ and less than $k(1)$.



Applying shouting algoritm the following value of $k(2)$ has been obtained:
\begin{align*}
k(2) = 0.1368665097187844
\end{align*}
and corresponding capital and consumption paths are depicted on the Fig. \ref{fig1} , \ref{fig2} respectively.
\begin{figure}[h]
	\centering
	\includegraphics[width=0.8\textwidth]{k(t)}
	\caption{}\label{fig1}
\end{figure}
\begin{figure}[h]
	\centering
	\includegraphics[width=0.8\textwidth]{c(t)}
	\caption{}\label{fig2}
\end{figure}




It should be borne in mind that aforementioned $k(2)$ is the $\textit{approximate}$ value of true $k(2)$ that means that with this particular (in general not optimal) $k(2)$ for large $t$ the algorithm will diverge from the steady state. Hence those plots describe true capital and consumption paths only approximately. 
\end{document}