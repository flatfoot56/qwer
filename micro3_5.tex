\documentclass[a4paper]{article}
\usepackage[14pt]{extsizes} % 
\usepackage[utf8]{inputenc}
\usepackage{setspace,amsmath}
\usepackage{mathtools}
\usepackage{pgfplots}
\usepackage{titlesec}
\usepackage{pdfpages}
\usepackage[shortlabels]{enumitem}
\usepackage{tikz}
\usetikzlibrary{angles,quotes}
\usepackage{graphicx}
\usepackage{amssymb}
\usepackage{float}
\usepackage[section]{placeins}
\usepackage[makeroom]{cancel}
\usepackage{mathrsfs} % 
\newcommand\numberthis{\addtocounter{equation}{1}\tag{\theequation}}
%\addto\captionsrussian{\renewcommand{\figurename}{Fig.}}
\usepackage{amsmath,amsfonts,amssymb,amsthm,mathtools} 
\newcommand*{\hm}[1]{#1\nobreak\discretionary{}
{\hbox{$\mathsurround=0pt #1$}}{}}
\usepackage{graphicx}  % 
\graphicspath{{images/}{images2/}}  % 
\setlength\fboxsep{3pt} %  \fbox{} 
\setlength\fboxrule{1pt} % \fbox{}
\usepackage{wrapfig} % 
\newcommand{\prob}{\mathbb{P}}
\newcommand{\norma}{\mathscr{N}}
\newcommand{\expect}{\mathbb{E}}
\newcommand{\summa}{\sum_{i=1}^n}
\newcommand{\yrseduc}{\textit{yrseduc}}
\usepackage[left=7mm, top=20mm, right=15mm, bottom=20mm, nohead, footskip=10mm]{geometry} % 
\usepackage{tikz} % 
\def\myrad{2cm}% radius of the circle
\def\myanga{45}% angle for the arc
\def\myangb{195}
\begin{document} % 
	\begin{flushright}
	\begin{tabular}{r}
		Danil Fedchenko, MAE 2020, group A \\
	\end{tabular}
\end{flushright}


\begin{center}
	Microeonomics 3. Problem Set 5.
\end{center}
\section*{Problem 1}
\begin{enumerate}[(a)]
	\item  Consider a modified version of Question 4 in Problem Set 4 in which trade takes
	place sequentially. Suppose there are two Arrow securities (denominated in terms of the
	physical good 1). Find a Radner equilibrium of this modified model. (You should determine
	the equilibrium values of all variables, i.e., consumption plans, asset portfolios, spot prices
	and asset prices.)
	\item Suppose now that in place of the Arrow securities, there are two assets defined by the
	return vectors $r_1 = (2, 1)$ and $r_2 = (1, 2)$, respectively. As usual, the $s$’th coordinate of $r_k$
	represents the amount of good 1 that the asset $k$ pays in state $s$ (for $k = 1, 2$ and $s = 1, 2$).
	Find a Radner equilibrium of the corresponding sequential exchange model.

\end{enumerate}


\textbf{Solution}


\begin{enumerate}[(a)]
	\item From the Problem 4 of Problem Set 4 it is known that the Arrow-Debreu equilibrium consists of prices:
	\begin{align*}
	p_{11} = p_{22} = 1,\ p_{12} = p_{21} = 3
	\end{align*}
	and allocation:
	\begin{align*}
	x_1 = \begin{pmatrix}
	3\\
	1\\
	3\\
	9
	\end{pmatrix},\ x_2 = \begin{pmatrix}
	9\\
	3\\
	1\\
	3
	\end{pmatrix}
	\end{align*}
From the theorem of equivalence between Arrow-Debreu equilibrium and Radner equilibrium it follows that prices of two Arrow securities are equal to:
\begin{align*}
q^* = \begin{pmatrix}
p_{11}\\
p_{12}
\end{pmatrix} = \begin{pmatrix}
1\\
3
\end{pmatrix}
\end{align*}
Equilibrium consumption plans do not change. It remains to find equilibrium portfolio for each consumer. 
\begin{align*}
z^*_{si} = \frac{1}{p_{1s}}p_s(x_{si} - \omega_{si})\\
z^*_{1} =  \begin{pmatrix}
-18\\
6
\end{pmatrix},\ z^*_{2} = \begin{pmatrix}
18\\
-6
\end{pmatrix}
\end{align*}
\item 
\begin{align*}
R = \begin{pmatrix}
2 & 1\\
1 & 2
\end{pmatrix}
\end{align*}
The return matrix has a full rank hence the consumption plan do not change and the asset prices will be:
\begin{align*}
\tilde{q}^* = R^Tq = \begin{pmatrix}
5\\
7
\end{pmatrix}
\end{align*}
Again, by the theorem of equivalence between Radner equilibrium with assets and Arrow-Debreu equilibrium, prices do not change. It remains to find assets portfolio for each consumer. To do so, for the first consumer we need to solve the following linear system:
\begin{align*}
\begin{pmatrix}
2 & 1\\
1 & 2
\end{pmatrix} \begin{pmatrix}
\tilde{z_{11}}^*\\
\tilde{z_{12}}^*
\end{pmatrix} = \begin{pmatrix}
-18\\
6
\end{pmatrix}
\end{align*}
and the solution is 
\begin{align*}
\tilde{z}_1^* = \begin{pmatrix}
-14\\
10
\end{pmatrix}
\end{align*}
For the second consumer, we need to clear the security market hence:
\begin{align*}
\tilde{z_2}^* = \begin{pmatrix}
14\\
-10
\end{pmatrix}
\end{align*}
(Also note that $\tilde{q}^*\tilde{z_i}^* = 0\ \forall\ i$ i.e. no-arbitrage condition holds)
\end{enumerate}
\section*{Problem 2}
\begin{enumerate}[(a)]
\item Consider a sequential version of Question 5 in Problem Set 4 with two Arrow
securities. Find a Radner equilibrium of this modified model.
\item  If we were to add one more asset to this economy defined by a return vector $r_3 \in \mathbb{R}^2$,
how would you compute the equilibrium price of this new asset?

\end{enumerate}



\textbf{Solution}

\begin{enumerate}
	\item Again, from the Problem 5 of Problem Set 4 it is known that the Arrow-Debreu equilibrium consists of prices:
	\begin{align*}
	p_{11} = p_{21} = 1,\  p_{22} = p_{12} = 2 
	\end{align*}
	and the equilibrium allocation:
	\begin{align*}
	x_1 = \begin{pmatrix}
	4\\
	8\\
	2\\
	4
	\end{pmatrix},\ x_2 = \begin{pmatrix}
	8\\
	4\\
	4\\
	2
	\end{pmatrix}
	\end{align*}
	From the theorem of equivalence between Arrow-Debreu equilibrium and Radner equilibrium it follows that prices of two Arrow securities are equal to:
	\begin{align*}
	q^* = \begin{pmatrix}
	p_{11}\\
	p_{12}
	\end{pmatrix} = \begin{pmatrix}
	1\\
	2
	\end{pmatrix}
	\end{align*}
	Equilibrium consumption plans do not change. It remains to find equilibrium portfolio for each consumer. 
	\begin{align*}
	z^*_{si} = \frac{1}{p_{1s}}p_s(x_{si} - \omega_{si})\\
	z^*_{1} =  \begin{pmatrix}
	0\\
	0
	\end{pmatrix},\ z^*_{2} = \begin{pmatrix}
	0\\
	0
	\end{pmatrix}
	\end{align*}
	\item Since market with 2 Arrow securities was complete that means that we can express $r_3$ as a unique linear combination of columns of the identity matrix $2 \times 2$ i.e. $r_3 = r_{31} e_1 + r_{32} e_2$. Then:
	\begin{align*}
	q_3 = r_{31} q_1 + r_{32} q_2= r_{31} + 2r_{32}
	\end{align*}
\end{enumerate}
\section*{Problem 3}
Consider a sequential exchange model with two physical goods, two states, two consumers, and two assets. The return vectors of the assets are as follows:
\begin{align*}
r_1 = (1,\ 1)\ r_2 = (2,\ 1)
\end{align*}
As usual, the $s$’th entry of $r_k$ represents the return of asset $k$ in state $s$ in terms of the
physical good 1.
Consumers’ Bernoulli utility functions are state independent:
\begin{align*}
u_1(x_{s1}) = (x_{1s1})^{1/2}(x_{2s1})^{1/2},\ 
u_2(x_{s2}) = (x_{1s2})^{1/3}(x_{2s2})^{2/3}
\end{align*}
Here, $x_{si} = (x_{1si}, x_{2si})$ stands for the consumption bundle of consumer $i$ in state $s = 1, 2$.
Both consumers assign probability $1/2$ to each state.
Let $\omega_i = (\omega_{lsi})$, where $\omega_{lsi}$ is consumer $i$’s endowment of good $l$ in state $s$. Assume
further that:
\begin{align*}
\omega_1 = (\omega_{111}, \omega_{211}, \omega_{121}, \omega_{221}) = (5, 1, 0, 0),\ \ \omega_2 = (\omega_{112}, \omega_{212}, \omega_{122}, \omega_{222}) = (0, 0, 5, 1).
\end{align*}
Thus, consumer 1 has all the endowment in state 1 while consumer 2 has all the endowment
in state 2. Moreover, the total endowment vector equals $(5,\ 1)$ in both states.
Compute a Radner equilibrium of this model.


\textbf{Solution}


Let us compute firstly the Arrow-Debreu equilibrium. Consumers are solving the following problems:
\begin{align*}\
\underset{x_{111}, x_{211}, x_{121}, x_{221} \ge 0}{\max}\ \frac{1}{2} (x_{111})^{1/2} (x_{211})^{1/2} + \frac{1}{2} (x_{121})^{1/2}(x_{221})^{1/2}\\
s.t.\ p_{11}x_{111} + p_{21}x_{211} + p_{12}x_{121} + p_{22}x_{221} \le 5p_{11} + p_{21}\\
\underset{x_{112}, x_{212}, x_{122}, x_{222} \ge 0}{\max}\ \frac{1}{2} (x_{112})^{1/3} (x_{212})^{2/3} + \frac{1}{2} (x_{122})^{1/3}(x_{222})^{2/3}\\
s.t.\ p_{11}x_{111} + p_{21}x_{211} + p_{12}x_{121} + p_{22}x_{221} \le 5p_{12} + p_{22}\\
\end{align*}
And moreover market clearing conditions imply:
\begin{align*}
x_{111} + x_{112} = 5 \nonumber\\
x_{211} + x_{212} = 1\nonumber\\
x_{121} + x_{122} = 5\\
x_{221} + x_{222} = 1\nonumber
\end{align*}
Since derivatives of objective functions tend to infinity while some of $x$ tend to 0 hence the solution should lie at the interior. The solution is:
\begin{align*}
\begin{cases}
x_{111} = \frac{5p_{11} + p_{21}}{4p_{11}}\\
x_{211} = \frac{5p_{11} + p_{21}}{4p_{21}}\\
x_{121} = \frac{5p_{11} + p_{21}}{4p_{12}}\\
x_{221} = \frac{5p_{11} + p_{21}}{4p_{22}}\\
\end{cases}\\
\begin{cases}
x_{112} = \frac{5p_{12} + p_{22}}{6p_{11}}\\
x_{212} = \frac{5p_{12} + p_{22}}{3p_{21}}\\
x_{122} = \frac{5p_{12} + p_{22}}{6p_{12}}\\
x_{222} = \frac{5p_{12} + p_{22}}{3p_{22}}\\
\end{cases} 
\end{align*}
Market clearing conditions imply that:
\begin{align*}
\begin{cases}
-45p_{11} + 3p_{21} + 10p_{12} + 2p_{22} = 0\\
15p_{11} - 9p_{21} + 20p_{12} + 4p_{22} = 0\\
15p_{11} + 3p_{21} - 50p_{12} + 2p_{22} = 0\\
15p_{11} + 3p_{21} +20p_{12} - 8p_{22} = 0\\
\end{cases}
\end{align*}
Normalizing $p_{11} = 1$ we get the solution:
\begin{align*}
p_{11} = p_{12} = 1,\ p_{22} = p_{21} = 7
\end{align*}
and the equilibrium allocation is:
\begin{align*}
x_1^* = \begin{pmatrix}
3\\
\\
\frac{3}{7}\\
\\
3\\
\\
\frac{3}{7}
\end{pmatrix},\ x_2^* = \begin{pmatrix}
2\\
\\
\frac{4}{7}\\
\\
2\\
\\
\frac{4}{7}
\end{pmatrix}
\end{align*}
Then, in Radner equilibrium with 2 Arrow securities:
\begin{align*}
q^* = \begin{pmatrix}
1\\
1
\end{pmatrix}
\end{align*}
equilibrium portfolio for each consumer can be found as follows: 
\begin{align*}
z^*_{si} = \frac{1}{p_{1s}}p_s(x_{si} - \omega_{si})\\
z^*_{1} =  \begin{pmatrix}
-6\\
6
\end{pmatrix},\ z^*_{2} = \begin{pmatrix}
6\\
-6
\end{pmatrix}
\end{align*}
For
\begin{align*}
&R = \begin{pmatrix}
1 & 2\\
1 & 1
\end{pmatrix}\\
&\text{assets' price vector will be }\tilde{q}^* = R^Tq = \begin{pmatrix}
2\\
3
\end{pmatrix}
\end{align*}
It remains to find assets portfolio for each consumer. To do so, for the first consumer we need to solve the following linear system:
\begin{align*}
\begin{pmatrix}
1 & 2\\
1 & 1
\end{pmatrix} \begin{pmatrix}
\tilde{z_{11}}^*\\
\tilde{z_{12}}^*
\end{pmatrix} = \begin{pmatrix}
-6\\
6
\end{pmatrix}
\end{align*}
and the solution is 
\begin{align*}
\tilde{z}_1^* = \begin{pmatrix}
18\\
-12
\end{pmatrix}
\end{align*}
For the second consumer, we need to clear the security market hence:
\begin{align*}
\tilde{z_2}^* = \begin{pmatrix}
-18\\
12
\end{pmatrix}
\end{align*}
\section*{Problem 4}
Consider a sequential exchange model with one physical good, three states, three Arrow
securities and two consumers. Let $x_{si}$ stand for consumption of consumer $i = 1, 2$ in state
$s = 1, 2, 3$. Consumers’ Bernoulli utility functions are state independent, and defined as
follows:
\begin{align*}
u_1(x_{s1}) = \sqrt{x_{s1}},\  u_2(x_{s2}) = \ln x_{s2}.
\end{align*}
Both consumers assign probability $\pi_s$ to the state $s$, where 
\begin{align*}
\pi_1 =\frac{1}{6},\ \pi_2 =\frac{2}{6},\ \pi_3 =\frac{3}{6}
\end{align*}
Consumers’ endowment vectors are as follows:
$\omega_1 = (0,\ 3,\ 0), \omega_2 = (3,\ 0,\ 1)$.
(As usual, the $s$-th entry of $\omega_i$ stands for the endowment of consumer $i$ in state $s$.) Questions
(i)-(iv) below refer to a Radner equilibrium of this model.
\begin{enumerate}[(i)]
\item Let $q_s$ denote the price of the $s$-th Arrow security, which has a positive return in state $s$.
Find a relation between $q_1$ and $q_2$. Explain your answer. (Hint. Utility functions are state
independent and strictly concave, we have a common prior, and total endowments are the
same in states 1 and 2. This should remind you an earlier example.)
\item Normalize $q_3$ to $1$. Use your answer in part (i) to determine consumers’ consumption
plans and asset portfolios, as a function of $q_1$.
\item Compute the value of $q_1$.
\item Provide a formula for the price of any additional asset that we may introduce into this
economy
\end{enumerate}


\textbf{Solution}

\begin{enumerate}[(i)]
	\item At the interior equilibrium we should have:
	\begin{align*}
	\frac{\pi_1 u_1'(x_{1})}{\pi_2 u_1'(x_2)} = \frac{p_1}{p_2} = \frac{q_1}{q_2} = \frac{\pi_1u_2'(3 - x_1)}{\pi_2u_2'(3 - x_2)}
	\end{align*}
	If $x_1 > x_2$ then $3-x_1 < 3-x_2$ and by strict concavity of utility functions:
	\begin{align*}
	\frac{u_1'(x_1)}{u_1'(x_2)} < 1 < \frac{u_2'(3-x_1)}{u_2'(3-x_2)}
	\end{align*}
	it follows that $x_1 = x_2$ and as a result $q_1 = q_2$.
	\item $q_3 = 1$. Similarly for other states:
	\begin{align*}
	\frac{\pi_1 u_1'(x_{11})}{\pi_3 u_1'(x_{31})} = \frac{p_1}{p_3} = &q_1 = \frac{\pi_1u_2'(x_{12})}{\pi_3u_2'(x_{32})}\\
	\sqrt{\frac{x_{31}}{x_{11}}} = &q_1 = \frac{x_{32}}{x_{12}}
	\end{align*}
	incorporating market-clearing:
	\begin{align*}
	\sqrt{\frac{x_{31}}{x_{11}}} = q_1 = \frac{1- x_{31}}{3 - x_{11}}
	\end{align*}
	\begin{align*}
	x_{31} = 1 - (3-&x_{11})q_1\\
	\frac{1 - (3-x_{11})q_1}{x_{11}} &= q_1^2\\
	x_{11} = \frac{3q_1 - 1}{q_1(1 - q_1)} &= x_{21}\\
	x_{31} = \frac{q_1(3q_1 - 1)}{1 - q_1}\\
	x_1^* = \begin{pmatrix}
	\frac{3q_1 - 1}{q_1(1 - q_1)}\\
		\\
		\frac{3q_1 - 1}{q_1(1 - q_1)}\\
			\\
		\frac{q_1(3q_1 - 1)}{1 - q_1}	
	\end{pmatrix},\ &x_2^* = \begin{pmatrix}
	\frac{1 - 3q_1^2}{q_1(1 - q_1)}\\
	\\
	\frac{1 - 3q_1^2}{q_1(1 - q_1)}\\
	\\
	\frac{1 - 3q_1^2}{1 - q_1}\\
	\end{pmatrix}
	\end{align*}
	And the assets portfolios will be:
	\begin{align*}
	z_1^* = \begin{pmatrix}
	\frac{3q_1 - 1}{q_1(1 - q_1)}\\
	\\
	-\frac{1 - 3q_1^2}{q_1(1 - q_1)}\\
	\\
	\frac{q_1(3q_1 - 1)}{1 - q_1}\\
	\end{pmatrix},\ z_2^* = \begin{pmatrix}
	-\frac{3q_1 - 1}{q_1(1 - q_1)}\\
	\\
	-\frac{1 - 3q_1^2}{q_1(1 - q_1)}\\
	\\
	-\frac{q_1(3q_1 - 1)}{1 - q_1}	
	\end{pmatrix}
	\end{align*}
	\item To compute $q_1$ we need to use the budget constraint, i.e.
	\begin{align*}
	2\frac{3q_1 - 1}{1-q_1} + \frac{q_1(3q_1 - 1)}{1 - q_1} = 3q_1\\
	q_1 = \frac{\sqrt{13} - 1}{6}
	\end{align*}
	\item Since the market with 3 Arrow securities was complete alike in the Problem 2, price of new security can be computed as a linear combination of prices of basic securities, with weights being equal to returns of the new security. That is, if new asset pays $r_4 = (r_{14}\ r_{24}\ r_{32})$ then its price will be:
	\begin{align*}
	q_4 = r_{14}q_1 + r_{24}q_2+ r_{32}q_3
	\end{align*}
\end{enumerate}
\section*{Problem 5}
\textbf{Step 1}

The no arbitrage condition implies that the linear space $\Phi$ does not intersect the convex set $\mathbb{R}^s_+\setminus\left\{0\right\}$. Because if some $\tilde{z} \in \mathbb{R}^s_+\setminus\left\{0\right\}$ and at the same time $\tilde{z} \in \Phi$ that means that $\exists\ z\in \mathbb{R}^K$ such that $qz = 0$ and $Rz = \tilde{z} > 0$ which contradicts the no-arbitrage condition. Since $\mathbb{R}^s_+\setminus\left\{0\right\}$ is a convex set as well as a linear subspace $\Phi$, we can use a swparation hyperplane theorem. That is, $\exists\ \mu \in \mathbb{R}^s: \mu \neq 0$ and $a \in \mathbb{R}$ such that
\begin{align*}
\mu z \ge a \ge \mu \phi\, \forall\ z \in \mathbb{R}^s_+\setminus\left\{0\right\},\ \forall\ \phi \in \Phi
\end{align*}
Since $\left\{0\right\} \in \Phi$ then
\begin{align}\label{eq1}
\mu z \ge a \ge 0 \ge \mu \phi\, \forall\ z \in \mathbb{R}^s_+\setminus\left\{0\right\},\ \forall\ \phi \in \Phi
\end{align}
Moreover, since $\Phi$ is a linear space that means that $\forall\ \phi \in \Phi\ \exists\ -\phi \in \Phi$ hence:
\begin{align*}
\forall\ \phi\in\Phi\, \mu \phi \le 0\ \text{ and } -\mu \phi \le 0 \to \mu\phi = 0\
\end{align*}
It remains to prove that $\mu >0$. To do so, let us assume the opposite, i.e. $\exists\ \text{ a coordinate } s: \mu_s < 0$. Obviously $\forall\ \varphi\ \in \mathbb{R}^s_{+}\setminus\left\{0\right\}$ and $\forall\ \lambda > 0$, $\varphi + \mathbb{I}_{s}\lambda \in \mathbb{R}^s_{+}\setminus\left\{0\right\}$. But for all $\varphi$ we can pick such a large $\lambda$ that
\begin{align*}
\mu (\varphi + \mathbb{I}_s\lambda) = \mu \varphi + \mu_s\lambda < 0
\end{align*}
but by \eqref{eq1} it should be nonnegative for any $\lambda$. In conclusion, $\exists\ \mu >0:\ \mu \phi = 0\, \forall\ \phi \in \Phi$, as we sought.

\textbf{Step 2}


Given that $\mu$, the vector $\mu R$ belongs to the linear space $\left\{\alpha q : \alpha \in \mathbb{R}\right\}$. Indeed, otherwise, we can separate the vector $\mu R$ from the latter space, since both $\mu R$ as a point and $\left\{\alpha q : \alpha \in \mathbb{R}\right\}$ are convex sets. That is
\begin{align*}
\exists\ \text{ nonzero }\ v: \mu R v  \ge a \ge rv\ \forall\ r\ \in\left\{\alpha q : \alpha \in \mathbb{R}\right\}
\end{align*}
Since the latter space is a linear
space, just as in the previous step, $\forall\ r\ \in \left\{\alpha q : \alpha \in \mathbb{R}\right\},\ -r\ \in \left\{\alpha q : \alpha \in \mathbb{R}\right\} $ and as a result $rv = 0\ \forall\ r\in\left\{\alpha q : \alpha \in \mathbb{R}\right\}$. This leads to a separating normal vector $v \in \mathbb{R}^K$ such that
$qv = 0$ and $(\mu R)v = \mu (Rv) \ge 0$. Since $\mu >0$ and each column of $R$ also $>0$ hence $\mu R >0$ therefore if $z \neq 0$ and $\mu R z \ge 0$ then $\mu R v > 0$. Letting $\phi \equiv Rv \in \Phi$ yields a contradiction to the fact that $\mu \phi = 0\ \forall\ \phi$. The contradiction shows that $\mu R$ belongs to the linear space $\left\{\alpha q : \alpha \in \mathbb{R}\right\}$.

\textbf{Step 3}

As it has been concluded in step 2 $\mu R >0$. Moreover $q >> 0$ because otherwise no-arbitrage condition would be violated (assume there exists some coordinate $s$ such that $q_s \le 0$ then there exists $z = \mathbb{I}_s \lambda$ for some $\lambda > 0$ such that $qz \le 0$ however since there exists $s$ such that $r_{sk} >0$ hence $Rz > 0$).
Since $\mu R = \alpha q$ it follows that $\alpha >0$ but that means that
\begin{align*}
\exists\ \hat{\mu} = \frac{1}{\alpha}\mu >0: \hat{\mu}R = q
\end{align*}
and this completes the proof.
\end{document}