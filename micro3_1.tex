\documentclass[a4paper]{article}
\usepackage[14pt]{extsizes} % 
\usepackage[utf8]{inputenc}
\usepackage{setspace,amsmath}
\usepackage{mathtools}
\usepackage{pgfplots}
\usepackage{titlesec}
\usepackage{pdfpages}
\usepackage[shortlabels]{enumitem}
\usepackage{tikz}
\usetikzlibrary{angles,quotes}
\usepackage{graphicx}
\usepackage{amssymb}
\usepackage[section]{placeins}
\usepackage[makeroom]{cancel}
\usepackage{mathrsfs} % 
\newcommand\numberthis{\addtocounter{equation}{1}\tag{\theequation}}
%\addto\captionsrussian{\renewcommand{\figurename}{Fig.}}
\usepackage{amsmath,amsfonts,amssymb,amsthm,mathtools} 
\newcommand*{\hm}[1]{#1\nobreak\discretionary{}
{\hbox{$\mathsurround=0pt #1$}}{}}
\usepackage{graphicx}  % 
\graphicspath{{images/}{images2/}}  % 
\setlength\fboxsep{3pt} %  \fbox{} 
\setlength\fboxrule{1pt} % \fbox{}
\usepackage{wrapfig} % 
\newcommand{\prob}{\mathbb{P}}
\newcommand{\norma}{\mathscr{N}}
\newcommand{\expect}{\mathbb{E}}
\usepackage[left=7mm, top=20mm, right=15mm, bottom=20mm, nohead, footskip=10mm]{geometry} % 
\usepackage{tikz} % 
\begin{document} % 
	\begin{flushright}
	\begin{tabular}{r}
		Danil Fedchenko, MAE 2020, group A \\
	\end{tabular}
\end{flushright}


\begin{center}
	Microeconomics 3. Problem Set 1.
\end{center}
\section*{Problem 1}
Graphically, argue that if the indifference curves of a consumer is thick (due to absence of local non-satiation), then the conclusion of the first fundamental theorem of welfare economics may fail. That is, there may exist an equilibrium that is not Pareto efficient.
(Hint. If the indifference curve of a consumer is thick, we can take some goods from her,
and give them to the other consumer, without hurting the former.)


\textbf{Solution}


Assume that for some $p$ the equilibrium allocation is $x$ (see Fig. 1). Moreover for the first consumer LNS property of preferences is violated, that is his indifference domain is depicted as a shaded area, while the second consumer has LNS and strictly convex preferences, which are represented by indifference curves. The depicted equilibrium point $x$ will not be a Pareto efficient because the first consumer is indifferent between points $x$ and $x'$, however the second strictly prefers $x'$ to $x$.
\newpage
\section*{Problem 2}
Normalize the price of good 1 by letting $p_1 = 1$. As a function of the parameter $\alpha \in (0, 1/2)$, compute the equilibrium value of $p_2$ and the equilibrium allocation for the
following utility functions and endowments:
\begin{align*}
u_1(x_{11}, x_{21}) = (x_{11})^{\alpha}(x_{21})^{1-\alpha}, \omega_1 = (2, 1)\\
u_2(x_{12}, x_{22}) = (x_{12})^{2\alpha}(x_{22})^{1-2\alpha}, \omega_2 = (2, 1)\\
\end{align*}


\textbf{Solution}


Assume $p_1 = 1$. To find the Walrasian demand functions we need to solve the following optimization problems:
\begin{align*}
\underset{x_{11} \ge 0, x_{21}\ge 0}{\max}\ (x_{11})^{\alpha}(x_{21})^{1-\alpha}\ \ s.t.\ \ x_{11} + p_2x_{21} \le 2 + p_2\\
\underset{x_{12} \ge 0, x_{22} \ge 0}{\max}\ (x_{12})^{2\alpha}(x_{22})^{1-2\alpha}\ \ s.t.\ \ x_{12} + p_2x_{22} \le 2 + p_2\\
\end{align*}
Since $\alpha \in (0, 1/2)$ it is a well-known \textit{Cobb-Douglas-utility-linear-budget-constraint-interrior-solution} optimization problem and its solutions are:
\begin{align*}
x_1 = \begin{pmatrix}
\alpha (2 + p_2)\\
\\
(1-\alpha)\frac{2+p_2}{p_2}
\end{pmatrix}\\
x_2 = \begin{pmatrix}
2\alpha (2 + p_2)\\
\\
(1-2\alpha)\frac{2+p_2}{p_2}
\end{pmatrix}\\
\end{align*}
Using market clearing condition $x_{11} + x_{12} = 4$ one can get
\begin{align*}
2\alpha + \alpha p_2 + 4\alpha + 2\alpha p_2 = 4\\
p_2 = \frac{4}{3\alpha} - 2
\end{align*}
Thus, the optimal allocation should be: 
\begin{align*}
x_1^* = \begin{pmatrix}
\frac{4}{3}\\
\\
\frac{4 - 4\alpha}{4 - 6\alpha}
\end{pmatrix}\\
x_2^* = \begin{pmatrix}
\frac{8}{3}\\
\\
\frac{4 - 8\alpha}{4 - 6\alpha}
\end{pmatrix}\\
\end{align*}
As it can be seen $\frac{4}{3} + \frac{8}{3} = 4 = \omega_{11} + \omega_{12}, \frac{4-4\alpha}{4 - 6\alpha} + \frac{4 - 8 \alpha}{4 - 6\alpha} = 2 = \omega_{21} + \omega_{22}$. That is, this allocation is non-wasteful.
\section*{Problem 3}
Normalize the price of good 1 by letting $p_1 = 1$. As a function of the parameter $a > 0$,
compute the equilibrium value of $p_2$ and the equilibrium allocation for the following utility
functions and endowments:
\begin{align*}
u_1(x_{11}, x_{21}) = x_{11} + x_{21},\ \omega_1 = (1, 1)\\
u_2(x_{12}, x_{22}) = \sqrt{x_{12} x_{22}},\ \omega_2 = (0, a)
\end{align*}


\textbf{Solution}


Assume $p_1 = 1$. To find the Walrasian demand functions we need to solve the following optimization problems:
\begin{align*}
\underset{x_{11} \ge 0, x_{21}\ge 0}{\max}\ x_{11} + x_{21}\ \ s.t.\ \ x_{11} + p_2x_{21} \le 1 + p_2\\
\underset{x_{12} \ge 0, x_{22} \ge 0}{\max}\ (x_{12})^{\frac{1}{2}}(x_{22})^{\frac{1}{2}}\ \ s.t.\ \ x_{12} + p_2x_{22} \le ap_2\\
\end{align*}
Obviously solutions are:
\begin{align*}
&x_1 = \begin{cases}
\begin{pmatrix}
0\\
\frac{1+p_2}{p_2}
\end{pmatrix}, p_2 < 1\\
\begin{pmatrix}
1+p_2\\
0
\end{pmatrix}, p_2 > 1\\
\left\{(x_{11}, x_{21}):x_{11} + x_{21} = 2 \right\}
\end{cases}\\
&x_2 = \begin{pmatrix}
\frac{p_2a}{2}\\
\\
\frac{a}{2}
\end{pmatrix}
\end{align*}
If $p_2 > 1, a > 0$ then $0 + \frac{a}{2} < 1 + a$ that is market clearing condition is violated, hence $p_2 \le 1$. In case $p_2 < 1$:
\begin{align*}
\frac{p_2a}{2} = 1 \to p_2 = \frac{2}{a}, a > 2
\end{align*}
and the allocation should be:
\begin{align*}
x_1^* = \begin{pmatrix}
0\\
1 + \frac{a}{2}
\end{pmatrix}\\
x_2^* = \begin{pmatrix}
1\\
\frac{a}{2}
\end{pmatrix}
\end{align*}
If $p_2 = 1$ then
\begin{align*}
x_1^* = \begin{pmatrix}
1 - \frac{a}{2}\\
1 + \frac{a}{2}
\end{pmatrix}\\
x_2^* = \begin{pmatrix}
\frac{a}{2}\\
\frac{a}{2}
\end{pmatrix}
\end{align*}
Of course, $1 - \frac{a}{2} \ge 0 \iff a \le 2$. Thus, the optimal allocation is:
\begin{align*}
(x_1^*, x_2^*) = \begin{cases}
\begin{cases}
\begin{pmatrix}
0\\
1 + \frac{a}{2}
\end{pmatrix}\\
\begin{pmatrix}
1\\
\frac{a}{2}
\end{pmatrix}
\end{cases}, p_2 < 1, a > 2\\
\begin{cases}
\begin{pmatrix}
1 - \frac{a}{2}\\
1 + \frac{a}{2}
\end{pmatrix}\\
\begin{pmatrix}
\frac{a}{2}\\
\frac{a}{2}
\end{pmatrix}
\end{cases}, p_2 = 1, 0 < a \le 2
\end{cases}
\end{align*}
\section*{Problem 4}
In a Robinson Crusoe economy, find the equilibrium allocation, relative prices, and the
firm profit for $u(x_1, x_2) = \ln x_1 + \ln x_2, f(z) = z^{1/4}$ and $L = 5$.


\textbf{Solution}


As the owner of the firm the consumer is solving the following optimization problem, taking $w, p$ as given:
\begin{align*}
\underset{z \ge 0}{\max}\ pz^{1/4} - wz\\
\text{FOC}:\ \frac{p}{4z^{3/4}} - w = 0\\
z^* = \left(\frac{p}{4w}\right)^{4/3}
\end{align*}
that means that the profit function is
\begin{align*}
\pi(p, w) = p \left(\frac{p}{4w}\right)^{1/3} - w\left(\frac{p}{4w}\right)^{4/3} = \left(\frac{1}{4^{1/3}} - \frac{1}{4^{4/3}}\right) \frac{p^{4/3}}{w^{1/3}} = \frac{3}{4^{4/3}}\frac{p^{4/3}}{w^{1/3}}
\end{align*}\\
Moreover, as a consumer the agent is solving the following optimization problem:
\begin{align*}
\underset{0 \le x_1 \le 5,\ x_2 \ge 0}{\max}\ \ln x_1 + \ln x_2\ s.t.\ \ wx_1 + px_2 \le 5w + \pi(p, w)
\end{align*}
Applying increasing transformation the problem can be reduced to a well known Cobb-Douglas optimization problem which has a solution:
\begin{align*}
x_1 = \frac{5w + \frac{3}{4^{4/3}}\frac{p^{4/3}}{w^{1/3}}}{2w}\\
x_2 = \frac{5w + \frac{3}{4^{4/3}}\frac{p^{4/3}}{w^{1/3}}}{2p}
\end{align*}
The market clearing condition should be satisfied, that is
\begin{align*}
x_2 = f(z^*)\\
x_1 = 5 - z^*
\end{align*}
Thus, 
\begin{align*}
\frac{5w + \frac{3}{4^{4/3}}\frac{p^{4/3}}{w^{1/3}}}{2p} &= \left(\frac{p}{4w}\right)^{1/3}\\
5w + \frac{3}{4^{4/3}}\frac{p^{4/3}}{w^{1/3}} &= 2 \frac{p^{4/3}}{4^{1/3}w^{1/3}}\\
\frac{w}{p} &= \frac{1}{4}
\end{align*}
\end{document}