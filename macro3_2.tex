\documentclass[a4paper]{article}
\usepackage[14pt]{extsizes} % 
\usepackage[utf8]{inputenc}
\usepackage{setspace,amsmath}
\usepackage{mathtools}
\usepackage{pgfplots}
\usepackage{titlesec}
\usepackage{pdfpages}
\usepackage[shortlabels]{enumitem}
\usepackage{tikz}
\usetikzlibrary{angles,quotes}
\usepackage{graphicx}
\usepackage{amssymb}
\usepackage{float}
\usepackage[section]{placeins}
\usepackage[makeroom]{cancel}
\usepackage{mathrsfs} % 
\newcommand\numberthis{\addtocounter{equation}{1}\tag{\theequation}}
%\addto\captionsrussian{\renewcommand{\figurename}{Fig.}}
\usepackage{amsmath,amsfonts,amssymb,amsthm,mathtools} 
\newcommand*{\hm}[1]{#1\nobreak\discretionary{}
{\hbox{$\mathsurround=0pt #1$}}{}}
\usepackage{graphicx}  % 
\graphicspath{{images/}{images2/}}  % 
\setlength\fboxsep{3pt} %  \fbox{} 
\setlength\fboxrule{1pt} % \fbox{}
\usepackage{wrapfig} % 
\newcommand{\prob}{\mathbb{P}}
\newcommand{\norma}{\mathscr{N}}
\newcommand{\expect}{\mathbb{E}}
\newcommand{\summa}{\sum_{i=1}^n}
\usepackage[left=7mm, top=20mm, right=15mm, bottom=20mm, nohead, footskip=10mm]{geometry} % 
\usepackage{tikz} % 
\def\myrad{2cm}% radius of the circle
\def\myanga{45}% angle for the arc
\def\myangb{195}
\begin{document} % 
	\begin{flushright}
	\begin{tabular}{r}
		Danil Fedchenko, MAE 2020, group A \\
	\end{tabular}
\end{flushright}


\begin{center}
	Macroeconomics 3. Problem Set 2.
\end{center}
\section*{Problem 1}
Consider a neoclassical growth model in continuous time with technological progress. Production function
is given by $F (K, AL)$, $A_t = Ze^{gt}, 0 < g < \rho$, where $\rho$ is the discount factor. Utility function is CRRA, $u (c) = c^{\gamma}, 0 < \gamma < 1$.
\begin{enumerate}[1.]
\item Write down the planner’s problem in terms of original variables $K_t, L_t, C_t$.
\item Denote $k_t \equiv \frac{K_t}{e^{gt}L_t}$ and $c_t =\frac{C_t}{e^{gt}L_t}$. Rewrite the planner’s problem in terms of these variables.
\item Write down all optimality conditions.
\item Plot the phase diagram in coordinates $(k, c)$. Show the optimal trajectory.
\item Suppose that at moment $t = 0$ the economy is in the steady state. Use the phase diagram to
describe what happens after an unexpected permanent change in labor productivity $Z$.
\item Suppose that at moment $t = 0$ the economy is in the steady state. Use the phase diagram to
describe what happens after an unexpected temporary change in labor productivity $Z$: at $t = 0$ $Z$
jumps to its new value, and it is expected that at $t = T > 0$ $Z$ drops back to its original value.
\end{enumerate}



\textbf{Solution}


\begin{enumerate}
	\item The planner's problem is:
	\begin{align*}
	&\underset{C(t)}{\max}\ \int_{0}^{\infty} L(t) \left(\frac{C(t)}{L(t)}\right)^{\gamma}e^{-\rho t}dt\\
	s.t.\ &\dot{K}(t) = F(K(t), Ze^{gt}L(t)) - \delta K(t) - C(t)\\
	&C(t) \ge 0, K(t) \ge 0, L(t) \ge 0\\
	&K(0) \ge 0, L(0) \ge 0 \text{ are given }
	\end{align*}
	\item $k_t = \frac{K_t}{e^{gt}L_t}$ assuming $L_t = L(0)\ \forall\ t$, we obtain
	\begin{align*}
	\dot{K}(t) = \frac{d}{dt}(k_te^{gt}L_t) = \dot{k}_te^{gt}L_t + gk_te^{gt}L_t
	\end{align*}
	hence resources constraint can be rewritten as follows:
	\begin{align*}
	\dot{k_t} + gk_t = \frac{F(K_t, e^{gt}ZL_t)}{e^{gt}L_t} - \delta\frac{K(t)}{e^{gt}L_t} - \frac{C_t}{e^{gt}L_t}
	\end{align*}
	 assuming CRS production function, finally the planner's problem should be rewritten in the following manner:
	\begin{align*}
	&\underset{c_t}{\max}\ \int_{0}^{\infty} c_t^{\gamma}e^{-(\rho - g \gamma)t}dt\\
	s.t.\ &\dot{k}(t) = F(k_t, Z) - (\delta+g) k_t - c_t\\
	&c_t \ge 0, k_t \ge 0\\
	&k_0 \ge 0 \text{ is given}
	\end{align*}
	\item 
	\begin{align*}
	H = c_t^{\gamma}e^{-(\rho - g\gamma)t} + \lambda_t (F(k_t, Z) - (\delta +g) k_t - c_t)\\
	\begin{cases}
	\frac{\partial H}{\partial c_t} = \gamma c_t^{\gamma - 1}e^{-(\rho - g\gamma)t} - \lambda_t = 0\\
	\dot{\lambda}_t = -\frac{\partial H}{\partial k_t} = -\lambda_t(F'(k_t, Z) - \delta - g)\\
	\lim_{t \to \infty} \lambda_tk_t = 0
	\end{cases}
	\end{align*}
	\item \begin{align*}
	&\frac{d}{dt} \left(\frac{\gamma e^{-(\rho - g\gamma)t}}{c_t^{1 - \gamma}}\right) = \dot{\lambda}_t\\
	-\frac{\gamma(1 - \gamma)\dot{c}_t}{c_t^{2 - \gamma}}e^{-(\rho - g\gamma)t} - &\frac{\gamma(\rho - g\gamma)}{c_t^{1 - \gamma}}e^{-(\rho - g\gamma)t} = \dot{\lambda}_t = -\lambda_t(F'(k_t, Z) - \delta - g)\\
	\frac{\gamma(1 - \gamma)\dot{c}_t}{c_t^{2 - \gamma}}e^{-(\rho - \gamma g)t} + &\frac{\gamma(\rho - g\gamma)}{c_t^{1 - \gamma}}e^{-(\rho - g\gamma)t} = \frac{\gamma}{c_t^{1 - \gamma}}e^{-(\rho - g\gamma)t}(F'(k_t, Z) - \delta - g)\\
	&\frac{(1 - \gamma)\dot{c}_t}{c_t} + \rho - g\gamma = F'(k_t, Z) - \delta - g
	\end{align*}
	If $\dot{c}_t = 0$ then either $F'(k_t, Z) = \rho +(1-\gamma)g + \delta$ or $c_t = 0$. Denote the solution to this equation as $\tilde{k}$. If $\dot{k}_t = 0$ then $c_t = F(k_t, Z) - (\delta + g)k_t$ assuming $F(\cdot)$ satisfies Inada conditions, the isoclines will be as it depicted on the Fig \ref{fig1}. 
	\begin{figure}[h]
		\centering
		\includegraphics[width=0.8\textwidth]{plotdraft}
		\caption{}\label{fig1}
	\end{figure}
\item If $Z$ increases than $F(k_t, Z)$ increases either as a result isocline $\dot{k}(t) = 0$ moves upward because for a given $k_t$, $c_t$ increases. Moreover, if marginal productivity of capital increases with increase of the other factor (which is a standard assumption about production functions) then in order for $F'(k_t, Z)$ remains unchanged and equal to $\delta + g + \rho - g\gamma$, $k_t$ should also increase (since $\frac{\partial F}{\partial k}$ decreases in $k$). Thus, the isocline,corresponded to $\dot{c} = 0$ moves to the left (see Fig. ). Of course, exact values of these shifts depend on parameters and a type of the production function, as a result in the new steady state consumption could be more or less than the old one. Let us consider the case, which is depicted on the Fig. \ref{fig1}. In this case at $t = 0$ consumption will be increased such that the future dynamics will be along the optimal saddle trajectory of the new steady state (see Fig. \ref{fig1}).
\item In this case the social planner should from $t = 0$ to $T = t$ moves along the trajectory of new, modified system but he should choose those trajectory of new system which exactly at $t = T$ brings the system to the saddle trajectory of old steady state. See Fig. \ref{fig3} for clarifications.

	\begin{figure}[H]
	\centering
	\includegraphics[width=0.6\textwidth]{plotdraft}
	\caption{}\label{fig3}
\end{figure}

\end{enumerate}
\section*{Problem 2}
Problem:
\begin{enumerate}[1.]
	\item
\begin{align*}
\underset{\left\{k(t+1), c(t)\right\}_{t=1}^{\infty}}{\max}\ \sum_{t=1}^{\infty} 0.96^t \ln c(t)\\
\text{s.t.\ } k(t+1) = Ak(t)^{1/3} + (1-0.02)k(t) - c(t)\\
c(t) \ge 0, k(t) \ge 0, k(1) = 0.1
\end{align*}
Euler's equation is:
\begin{align*}
\frac{c(t+1)}{\beta c(t)} = \frac{A}{3k(t+1)^{2/3}} + 1 - \delta
\end{align*}
In the steady state $c(t+1) = c(t)$ hence:
\begin{align*}
\frac{1}{\beta} = \frac{A}{3(k^*)^{2/3}} + 1 - \delta\\
\text{if } k^* = 1 \to A = 3(\frac{1}{\beta} + \delta - 1) = 0.185
\end{align*}
\item 
\begin{align*}
V(k) = \underset{c, k'}{\max}\ [u(c) + \beta V(k')]\\
s.t.\ k' = f(k) + (1 - \delta)k - c
\end{align*}
we don’t need values of $k$ below $0.1$ and
above $1$ because by theorem $k(t)$ monotonically increases from $k(1)$ towards the steady state $k^* = 1$.
\item Comparisons with the shooting algorithm for 100, 500 and 1000-points grids are depicted below on the Fig. \ref{fig4}, \ref{fig2} and \ref{fig5} respectively.
	\begin{figure}[h]
	\centering
	\includegraphics[width=0.8\textwidth]{shooting100}
	\caption{Comparison with shooting algorithm for 100-points grid}\label{fig4}
\end{figure}
	\begin{figure}[h]
	\centering
	\includegraphics[width=0.8\textwidth]{VFI}
	\caption{Comparison with shooting algorithm for 500-points grid}\label{fig2}
\end{figure}
	\begin{figure}[h]
	\centering
	\includegraphics[width=0.8\textwidth]{shooting1000}
	\caption{Comparison with shooting algorithm for 1000-points grid}\label{fig5}
\end{figure}



For 1000 points-grid accuracy is indeed significantly higher comparing to 100-points. However even in case of 500-points the accuracy is high enough so, in fact there is no need in 1000-points grid which requires high computational power of the computer (it has taken about 10-15 minutes for my computer to execute the algorithm for 1000 points).
\end{enumerate}
\end{document}