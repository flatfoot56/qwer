\documentclass[a4paper]{article}
\usepackage[14pt]{extsizes} % 
\usepackage[utf8]{inputenc}
\usepackage{setspace,amsmath}
\usepackage{mathtools}
\usepackage{pgfplots}
\usepackage{titlesec}
\usepackage{pdfpages}
\usepackage[shortlabels]{enumitem}
\usepackage{tikz}
\usetikzlibrary{angles,quotes}
\usepackage{graphicx}
\usepackage{amssymb}
\usepackage{float}
\usepackage[section]{placeins}
\usepackage[makeroom]{cancel}
\usepackage{mathrsfs} % 
\newcommand\numberthis{\addtocounter{equation}{1}\tag{\theequation}}
%\addto\captionsrussian{\renewcommand{\figurename}{Fig.}}
\usepackage{amsmath,amsfonts,amssymb,amsthm,mathtools} 
\newcommand*{\hm}[1]{#1\nobreak\discretionary{}
{\hbox{$\mathsurround=0pt #1$}}{}}
\usepackage{graphicx}  % 
\graphicspath{{images/}{images2/}}  % 
\setlength\fboxsep{3pt} %  \fbox{} 
\setlength\fboxrule{1pt} % \fbox{}
\usepackage{wrapfig} % 
\newcommand{\prob}{\mathbb{P}}
\newcommand{\norma}{\mathscr{N}}
\newcommand{\expect}{\mathbb{E}}
\newcommand{\ubar}{\overline}
\newcommand{\lbar}{\underline}
\newcommand{\summa}{\sum_{i=1}^n}
\usepackage[left=7mm, top=20mm, right=15mm, bottom=20mm, nohead, footskip=10mm]{geometry} % 
\usepackage{tikz} % 
\def\myrad{2cm}% radius of the circle
\def\myanga{45}% angle for the arc
\def\myangb{195}
\begin{document} % 
	\begin{flushright}
	\begin{tabular}{r}
		Danil Fedchenko, MAE 2020, group A \\
	\end{tabular}
\end{flushright}


\begin{center}
	Microeconomics 4. Problem Set 4.
\end{center}
\section*{1 Credit Markets}
	A risk-neutral entrepreneur has a project that requires capital $K$ to initiate. The project
	will either succeed, in which case it generates profit $R$, or it will fail, in which case
	it generates profit $0$. The probability of success is equal to the entrepreneur's effort
	level $e \in [0, 1]$, and his effort cost is $\frac{e^2}{2}$. He seeks outside funding from an investor. The
	investor receives a repayment $r$ if the project is successful and $0$ otherwise. Therefore, the
	entrepreneur's and the investor's expected profit is $\Pi _E = e(R-r) - \frac{e^2}{2}$ and $\Pi _I = er-K$,respectively. Assume that the entrepreneur makes a take-it-or-leave-it offer to an investor,
	whose outside option is $0$. (This is equivalent to assuming that the investor offers the
	contract and he operates in a perfectly competitive market.) Furthermore, assume that $R^2 > 4K$.
	\begin{enumerate}
	\item Assume that effort level is contractible. Derive the first-best outcome $e^{FB}$ and $r^{FB}$.
	\item  Now suppose that effort is not contractible. Find the optimal solution $e^*$ and $r^*$.
	\item  Assume that the entrepreneur has initial wealth $w \in [0, K]$. Therefore, he will invest
	his own wealth $w$ in the project, and borrow $K-w$ from an investor. Compute the entrepreneur's expected utility $V(w)$. Show that it is increasing and concave in $w$. Provide some intuition and interpretation.
	\end{enumerate}

\textbf{Solution}

Assume that $R > r$.
\begin{enumerate}
	\item If the entrepreneur makes take-it-or-leave offer than he is solving the following optimization problem:
	\begin{align*}
	\underset{e, r}{\max}\ e(R - r) - \frac{e^2}{2}\\
	s.t.\ er - K \ge 0
	\end{align*}
	The solution is obviously
	\begin{align*}
	e^{FB} = \min\{R, 1\},\ r^{FB} = \frac{K}{\min\{R, 1\}}
	\end{align*}
	\item If the effort level is not contractible then the entrepreneur will take $r$ as given and to choose an effort level $e$ he should solve:
	\begin{align*}
	\underset{e}{\max}\ e(R - r) - \frac{e^2}{2}
	\end{align*}
	the solution is $e^* = \min\{R - r, 1\}$. To choose $r$ such that the investor would be willing to accept take-it-or-leave offer, the entrepreneur should solve:
	\begin{align*}
	\underset{r}{\max}\ e^*(R-r) - \frac{e^{*2}}{2}\\
	s.t.\ e^*r - K \ge 0
	\end{align*}
	If $e^* = 1$ then obviously $r^* = \max\{K, R-1\}$. If $e^* = R - r$ then
	\begin{align*}
	\underset{r}{\max}\ \frac{(R - r)^2}{2}\\
	s.t.\ r(R - r) - K \ge 0\\
	r^2 - rR + K \le 0\\
	\frac{R - \sqrt{R^2 - 4K}}{2} \le r \le \frac{R + \sqrt{R^2 - 4K}}{2}
	\end{align*}
	Note that $R^2 - 4K > 0$ guarantees existence of $\sqrt{R^2 - 4K}$. Since we assumed that $R - r > 0$ then $r^* = \frac{R - \sqrt{R^2 - 4K}}{2}$. The solution is
	\begin{align*}
	e = \begin{cases}
	1, K \le R - 1\\
	\frac{R + \sqrt{R^2 - 4K}}{2}, K > R - 1
	\end{cases} r = \begin{cases}
	R - 1, K \le R - 1\\
	\frac{R - \sqrt{R^2 - 4K}}{2}, K > R - 1
	\end{cases}
	\end{align*}
	\item Now the entrepreneur should solve:
	\begin{align*}
	\underset{r}{\max}\ e^*(R - r) - \frac{e^{*2}}{2}\\
	s.t.\ e^*r - (K - w) \ge 0
	\end{align*}
	If we assume that $e = \frac{R + \sqrt{R^2 - 4(K - w)}}{2} \le 1$ then expected profit is
	\begin{align*}
	V(w) = \frac{\left(R + \sqrt{R^2 - 4(K - w)}\right)^2}{8}
	\end{align*}
	\begin{align*}
	V'(w) = \frac{R + \sqrt{R^2 - 4(K - w)}}{\sqrt{R^2 - 4(K - w)}} \ge 0\\
	V''(w) = \frac{1 - \frac{2R}{\sqrt{R^2 - 4(K - w)}}}{R^2 - 4(K - w)} \le 0
	\end{align*}
	Interpretation: if the entrepreneur has more wealth then his expected wealth will be more (surprisingly). And the less initial wealth of entrepreneur is, the more he is benefited from the loan.
\end{enumerate}


\section*{2 Raising Liability Rule}
	We consider a lender-borrower relationship under moral hazard. The risk-neutral borrower wants to borrow I from a lender to finance a project with safe return $V$ . The project may with probability $1-e$ harm a third-party. The amount of safety care $e$ costs $\psi(e)$ to the borrower with $(\psi' > 0, \psi'' > 0, \psi''' > 0)$. The harm has value $h$. A financial contract is a pair $(\ubar{t}, \lbar{t})$ where $\lbar{t}$ (resp. $\ubar{t}$) is the borrower's reimbursement to the bank if	there is no (resp. one) environmental damage.
	
	\begin{enumerate}
		\item Suppose that $e$ is observable. Compute the first-best level of safety care and assume
		that the project is socially valuable when the interest rate is $r$.
		\item Suppose now that $e$ is not observable. We suppose that the bank is competitive
		and that the borrower has sufficient liability. Show that the first-best is still implementable if the bank must reimburse $h$ to the third-party in case of an accident.
		\item Suppose that the bank must reimburse $c < h$ to the third-party. We denote by $w$
		the initial assets of the borrower. Show that as $w$ diminishes, the first-best level of
		effort can no longer be implemented.
		\item Compute the second-best optimal level of effort maximizing the borrower's expected payoff subject to the bank's zero profit constraint, the borrower's incentive
		constraint and his limited liability constraint.
		\item Show that raising the bank's liability c leads to a lower expected welfare.
		\item Show that this result no longer holds when the bank is a monopoly.
	\end{enumerate}


\textbf{Solution}


\begin{enumerate}
	\item Obviously, if neither bank nor borrower cares about the third-party and safety care is costly, the optimal care is $e = 0$. Denote $r$ the marginal interest rate i.e. if the investor borrows $I$, he should return $(1+r)I$. In this case, the optimal contract is:
	\begin{align*}
	\lbar{t} = \ubar{t} &= V\\
	e &= 0
	\end{align*}
\end{enumerate}
\begin{enumerate}
	\item 
\end{enumerate}
\section*{3 Contracting under Coercion}
	A firm wishes to employ an agent. The agent has limited liability so wages are
	nonnegative, but the firm can punish the agent in a way that is unproductive (e.g. by
	humiliating him). The firm also can lower the worker's outside option by purchasing
	"guns" to harass the agent. More precisely, the timing is:
	\begin{enumerate}
	\item The firm chooses the level of guns, $g$, at cost $k(g)$.
	\item The firm offers a contract $\langle w(y), p(y)\rangle$ to the agent, where $y \in \{L, H\} = \{0, 1\}$ is the output, $w \ge 0$ is the wage, and $p \ge 0$ is the nonpecuniary punishment. If the
	agent rejects the contract, he receives $\ubar{u}-g$.
	\item If the agent accepts the contract, he chooses effort $a \in [0, 1]$ at cost $c(a)$, giving rise
	to output $\prob (y = H) = a$.
	\end{enumerate}


Payoffs for the firm and worker are:
\begin{align*}
\Pi = zy - w - k(g)\\
U = w - p -c(a)
\end{align*}
where $z$ is the price of output. For simplicity, suppose $k(\cdot)$ and $c(\cdot)$ are increasing,
convex and satisfy the appropriate Inada conditions (so we don't have to worry about
boundary problems).
\begin{enumerate}[(a)]
\item Write down the firm's problem subject to the $(PC)$ constraint and the $(IC)$ constraint.
\item Simplify the problem using the first-order approach. Is this approach valid here?
\item Argue that the optimal contract has $p_H = 0$ and $w_L = 0$, and that the $(PC)$ constraint binds.
\item Use the $(PC)$ and $(IC)$ constraints to write profits as follows:
\begin{align*}
\Pi = za - ac(a) + a(1-a)c'(a) - a\ubar{u} + ag-k(g)
\end{align*}
Henceforth, let us assume $\Pi$ is concave in $a$. For $x \in \mathscr{R}^n$, $t \in \mathscr{R}^n$, a function $f(x, t)$ is supermodular in $(x, t)$ if all the cross-partial derivatives are positive. Topkis proved that
if $f(x, t)$ is supermodular then the optimal solution
\begin{align*}
x(t) = argmax_xf(x, t)
\end{align*}
is increasing in the parameter $t$.
\item How does the optimal choice of $a$ and $g$, ($a^*(z, u), g^*(z, \ubar{u})$), vary in the price of
output and the outside option? Provide an intuition.
\end{enumerate}


\textbf{Solution}

\begin{enumerate}[(a)]
	\item 
\end{enumerate}

\section*{4 Moral Hazard in Teams}
	Consider Holmstrom's model of moral hazard in teams. There are $N > 1$ agents that
	work in a team. They exert efforts $e_i \ge 0, i = 1, \dots,N$ to jointly produce the output $x(e_1,\dots, e_N )$, where $x(e_1,\dots,  e_N )$ is increasing differentiable function of each effort $e_i$. For each agent,
	the cost of exerting an effort $e$ is given by an increasing, differentiable and convex function
	$c(e)$. Each agent observes her own effort and total output, but not the efforts of the other
	agents, so they cannot contract on the effort level.
	\begin{enumerate}
	\item Characterize the first-best (i.e., full information) levels of efforts $e_i, i = 1, \dots, N$
	\item Assume that the agents agreed on a differentiable balanced-budget sharing rule
	$\{t_i(x)\}_{i=1, \dots, N}$ . That is, agent $i$ receives $t_i(x)$, which yields her the payoff of
	\begin{align*}
	u_i = t_i(x(e_1, \dots, e_N )) - c(e_i),
	\end{align*}
	and the transfers to all agents exactly exhaust the budget
	\begin{align}
	\sum_{i=1}^N t_i(x) = x
	\end{align}
	Is there a differentiable balanced-budget sharing rule which allows to sustain the
	efficient production level $x^*= x(e_1,\dots,e_N )$? Support your answer analytically.
	Suggest an intuitive explanation.
	\item Assume now that there is an $N + 1$-th actor in this game, a principal, who does
	not contribute to the production but may be part of the sharing rule. Consider the
	following differentiable sharing rule
	\begin{align}
	t_i(x) = x - \frac{N-1}{N}x(e_1,\dots, e_N) \text{ for } i = 1, \dots, N\\
	t_{N+1}(x) = (N-1)(x(e_1, \dots, e_N) - x) 
	\end{align}
	\begin{enumerate}[i.]
	\item Show that this sharing rule is not balanced among agents $1, \dots, N$, but balanced
	among all $N + 1$ agents.
	\item  Does this rule allow to attain the efficient production level? Justify your answer analytically.
	\item  What is the role of the principal here? Does the principal receive any payoff
	\end{enumerate}
	in equilibrium?
	\item Assume that the sharing rule is as in (b), but now the principal can collude with
	one of the agents, say, agent $k$. That is, assume that the principal and agent $k$ can
	write a secret contract maximizing their joint welfare. More specifically, consider a
	contract that makes agent $k$ residual claimant on the joint surplus of her and the
	principal (i.e., she would pay the principal a fixed fee, and maximize their joint
	welfare). The other agents would not know about this collusion.
	Suppose that first, the principal and agent $k$ decide whether to collude, then all
	productive agents $1, \dots, N$ (including agent $k$) choose and exert efforts, and finally
	the transfers are made according to the rule (2)-(3).
	Is the sharing rule (2)-(3) susceptible to collusion (i.e., can the agents still attain
	the efficient production level?) Prove your answer and support it by an intuitive explanation.
	\end{enumerate}
\end{document}