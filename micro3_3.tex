\documentclass[a4paper]{article}
\usepackage[14pt]{extsizes} % 
\usepackage[utf8]{inputenc}
\usepackage{setspace,amsmath}
\usepackage{mathtools}
\usepackage{pgfplots}
\usepackage{titlesec}
\usepackage{pdfpages}
\usepackage[shortlabels]{enumitem}
\usepackage{tikz}
\usetikzlibrary{angles,quotes}
\usepackage{graphicx}
\usepackage{amssymb}
\usepackage{float}
\usepackage[section]{placeins}
\usepackage[makeroom]{cancel}
\usepackage{mathrsfs} % 
\newcommand\numberthis{\addtocounter{equation}{1}\tag{\theequation}}
%\addto\captionsrussian{\renewcommand{\figurename}{Fig.}}
\usepackage{amsmath,amsfonts,amssymb,amsthm,mathtools} 
\newcommand*{\hm}[1]{#1\nobreak\discretionary{}
{\hbox{$\mathsurround=0pt #1$}}{}}
\usepackage{graphicx}  % 
\graphicspath{{images/}{images2/}}  % 
\setlength\fboxsep{3pt} %  \fbox{} 
\setlength\fboxrule{1pt} % \fbox{}
\usepackage{wrapfig} % 
\newcommand{\prob}{\mathbb{P}}
\newcommand{\norma}{\mathscr{N}}
\newcommand{\expect}{\mathbb{E}}
\newcommand{\summa}{\sum_{i=1}^n}
\newcommand{\yrseduc}{\textit{yrseduc}}
\usepackage[left=7mm, top=20mm, right=15mm, bottom=20mm, nohead, footskip=10mm]{geometry} % 
\usepackage{tikz} % 
\def\myrad{2cm}% radius of the circle
\def\myanga{45}% angle for the arc
\def\myangb{195}
\begin{document} % 
	\begin{flushright}
	\begin{tabular}{r}
		Danil Fedchenko, MAE 2020, group A \\
	\end{tabular}
\end{flushright}


\begin{center}
	Microeconomics 3. Problem Set 3.
\end{center}
\section*{Problem 1}
Consider an Edgeworth box exchange economy with two commodities. Consumers'
endowments are given by 
\begin{align*}
\omega_1 = (\frac{\sqrt{2}}{2}, 4),\ \omega_2 = (\frac{\sqrt{2}}{2}, 5)
\end{align*}. 
Assume further that the utility
functions of the two consumers are as follows:
\begin{align*}
u_1(x_1) := x_{11} + \sqrt{x_{21}},\ \ 
u_2(x_2) := \sqrt{8}x_{12} + \sqrt{x_{22}}
\end{align*}
\begin{enumerate}[(i)]
\item Find the set of Pareto efficient allocations parametrically, as a function of the utility level
of consumer 2.
\item For each interior Pareto efficient allocation, find a corresponding equilibrium with transfers.

\end{enumerate}


\textbf{Solution}
\begin{enumerate}[(i)]
	\item To find the set of Pareto efficient allocations one need to solve the following optimization problem
	\begin{align*}
	\underset{x_{11}, x_{21} \ge 0}{\max\ }\ x_{11} + \sqrt{x_{21}}\\
	s.t.\ \sqrt{8}x_{12} + \sqrt{x_{22}} \ge \bar{u}\\
	x_{11} + x_{12} = \sqrt{2}\\
	x_{21} + x_{22} = 9
	\end{align*}
	Or alternatively
	\begin{align*}
	\underset{x_{11}, x_{21} \ge 0}{\max}\ x_{11} + \sqrt{x_{21}}\\
	s.t.\ 4 - \sqrt{8}x_{11} + \sqrt{9 - x_{21}} \ge \bar{u}\\
	x_{11} \le \sqrt{2},\ x_{21} \le 9
	\end{align*}
	constraint is obviously binding because otherwise we can slightly increase $x_{11}$ such that the constraint still will hold but the objective function will increase.
	(It should be borne in mind that $\bar{u}$ cannot exceed 7 because $u_2(\sqrt{2}, 9) = 7$). For the interior solution:
	\begin{align*}
	\text{FOCs}: &\begin{cases}
	1 - \sqrt{8}\lambda = 0\\
	\frac{1}{\sqrt{x_{21}}} - \frac{\lambda}{\sqrt{9 - x_{21}}} = 0
	\end{cases}\\
	\lambda = \frac{1}{\sqrt{8}}\ \to\ 
	\frac{1}{\sqrt{x_{21}}} &- \frac{1}{\sqrt{72 - 8 x_{21}}} = 0 \to\ \begin{cases} x_{21} = 8\\
	x_{11} = \frac{5 - \bar{u}}{\sqrt{8}}
	\end{cases}
	\end{align*}
	Thus the set of Pareto efficient allocation is:
	\begin{align}\label{eq1}
	\begin{cases}
	x^*_1 = 
	\begin{pmatrix}
	\sqrt{2}\\
	9 - \bar{u}^2
	\end{pmatrix},\ x^*_2 = \begin{pmatrix}
	0\\
	\bar{u}^2
	\end{pmatrix}, 0 \le \bar{u} < 1\\
	x^*_1 = 
	\begin{pmatrix}
	\frac{5 - \bar{u}}{\sqrt{8}}\\
	8
	\end{pmatrix},\ x^*_2 = \begin{pmatrix}
	 \frac{\bar{u}- 1}{\sqrt{8}}\\
	 1
	\end{pmatrix}, 1 \le \bar{u} < 5\\
	x^*_1 = 
	\begin{pmatrix}
	0\\
	9 - (\bar{u} - 4)^2\\
	\end{pmatrix},\ x^*_2 = \begin{pmatrix}
	\sqrt{2}\\
	(\bar{u} - 4)^2
	\end{pmatrix}, 5 < \bar{u} \le 7\\
	\end{cases}
	\end{align}
	\item To find euilibria with transfers we need to find wealth levels and prices corresponding to each interior Pareto efficient allocation. That means that we need to find such $w_1, w_2$ and such a price vector $p = (p_1\ p_2)$ that $x^*_1, x^*_2$ from the \eqref{eq1} maximize consumers' utility functions over the following budget sets:
	\begin{align*}
	\left\{(x_{11}, x_{21}) \in \mathbb{R}^2_{+}: p_1x_{11} + p_2x_{21} \le w_1\right\}\\	\left\{(x_{12}, x_{22}) \in \mathbb{R}^2_{+}: p_1x_{12} + p_2x_{22} \le w_2\right\}\\
	\end{align*}
	\begin{align*}
	\underset{x_{11}, x_{21} \ge 0}{\max}\ x_{11} + \sqrt{x_{21}}\\
	 s.t.\ p_1x_{11} + p_2x_{21} \le w_1
	\end{align*}
	In the interior allocations MRS = MRT hence
	\begin{align*}
	2\sqrt{x_{21}} = \frac{p_1}{p_2} \to\ p_1 = 4\sqrt{2}p_2
	\end{align*}
	We can assume $p_2 = 1$ hence $p_1 = 4\sqrt{2}$
	\begin{align*}
	\begin{cases}
	4\sqrt{2} \cdot \frac{5 - \bar{u}}{\sqrt{8}} + 8 = w_1\\
	4\sqrt{2} \cdot \frac{\bar{u} - 1}{\sqrt{8}} + 1 = w_2
	\end{cases}\\
	\begin{cases}
	w_1 = 18 - 2\bar{u}\\
	w_2 = 2\bar{u} - 1
	\end{cases}, 
	\end{align*}
\end{enumerate}
\end{document}